\documentclass[10pt, UTF8]{ctexbook} %% ctexart

\title{\textbf{中国近现代史纲要}

学习参考与复习指导}
\author{钱锋\thanks{Email: strik0r\_qf@mail.nwpu.edu.cn}}


\usepackage{graphicx}
\usepackage[toc]{multitoc}
\usepackage{booktabs}
\usepackage{longtable}
\usepackage{booktabs}
\usepackage{longtable}
\usepackage{amsthm, amssymb, amsmath, mathrsfs, mhchem}
\usepackage{tikz}
\usetikzlibrary{decorations.markings, angles, quotes}
\usepackage{pgfplots}
\usepackage{tikz-3dplot}
\usepackage{extpfeil}
\usepackage{diagbox}
\usepackage{multirow}
\usepackage{float}
\usepackage{hyperref}
\hypersetup{hidelinks,
    colorlinks = true,
    allcolors = black,
    pdfstartview = Fit,
    breaklinks = true}
\usepackage{caption}
\captionsetup[table]{labelsep=space} % 表
\captionsetup[figure]{labelsep=space} % 图
\usepackage{enumitem}
\usepackage{siunitx}
\usepackage{cuted}

\usepackage{mdframed}
\mdfsetup{
  linewidth=0.4pt,
  frametitlebackgroundcolor=white, % 或者 transparent
  frametitlefont=\heiti\bfseries,
  frametitleaboveskip=10pt,
  frametitlebelowskip=5pt,
}

\usepackage{titlesec} % 定义标题样式

\input{_titlesecsettings.tex}

\usepackage{fancyhdr} % 用于自定义页眉页脚


% 设置页眉页脚样式
\fancypagestyle{plain}{%
    \fancyhf{} % 清空页眉页脚
    \fancyhead[RO,LE]{·\thepage·} % 页眉显示页码, RO表示奇数页右侧, LE表示偶数页左侧
    \fancyhead[LO]{\nouppercase{\rightmark}} % 页眉显示小节标题, LO表示奇数页左侧
    \fancyhead[RE]{\nouppercase{\leftmark}} % 页眉显示章节标题, RE表示偶数页右侧
    \renewcommand{\headrulewidth}{0.4pt} % 设置页眉横线的宽度
    \renewcommand{\footrulewidth}{0pt} % 取消页脚横线
}

\renewcommand{\headrule}{\hrule width\textwidth height\headrulewidth\vskip-\headrulewidth}

\input{_geometry_setteings.tex}

% 取消奇偶页的页眉偏移
\fancyhfoffset[RO,LE]{0pt}

% 取消奇偶页的页眉偏移
\fancyhfoffset[RO,LE]{0pt}

% 定义取消页眉的命令
\newcommand{\cancelheader}{%
    \fancyhead{} % 清空页眉
    \renewcommand{\headrulewidth}{0pt} % 取消页眉横线
    \renewcommand{\footrulewidth}{0pt} % 设置页脚横线的宽度
}

\usepackage{fontspec}
% 设置Monaco字体
\setmonofont{Menlo}
\usepackage{fancyvrb}

\usepackage{listings}
\lstset{
    basicstyle=\small\ttfamily,
    numbers=left,
    numberstyle=\scriptsize\fontspec{Menlo}, % 使用 Menlo 字体
    stepnumber=1,
    numbersep=8pt,
    frame=leftline,
    framexleftmargin=0pt, % 调整边框的位置
    breaklines=true,
    postbreak=\mbox{\textcolor{red}{$\hookrightarrow$}\space},
}

\usepackage{smartdiagram}


\begin{document}

\newtheoremstyle{mytheoremstyle}
    {1.5ex}                                         % Space above
    {1.5ex}                                         % Space below
    {}                                              % Font for body
    {}                                              % Indent amount
    {\bfseries}                                     % Font for head
    {}                                              % Punctuation after head
    {0.5em plus 0.2em minus 0.1em}                  % Space after head
    {\thmname{#1}\thmnumber{ #2}.\thmnote{ (#3).}}

\theoremstyle{mytheoremstyle} \newtheorem{example}{例}[section]
\theoremstyle{mytheoremstyle} \newtheorem{key}{核心要点}[section]

\theoremstyle{plain} \newtheorem{thm}{分析论述}

\newtheoremstyle{my3theoremstyle}
    {1.5ex}                                         % Space above
    {1.5ex}                                         % Space below
    {}                                              % Font for body
    {}                                              % Indent amount
    {\kaishu}                                       % Font for head
    {}                                              % Punctuation after head
    {0.5em plus 0.2em minus 0.1em}                  % Space after head
    {\thmname{#1}\thmnumber{ #2}.\thmnote{ (#3).}}

\theoremstyle{my3theoremstyle}
\newtheorem*{remark}{注}
\newtheorem*{sol}{答案要点}
% \maketitle

\begin{titlepage}
    \thispagestyle{empty}
    \centering
        \vspace*{3cm}
        \includegraphics[width=0.5\textwidth]{npu_2.png}\par
        \vspace{1cm}
        \includegraphics[width=0.5\textwidth]{npu_1.png}\par
    \vspace{1cm}
        \begin{center}
            \Huge \heiti \textbf{军事理论 \& 军事技能训练}
        \end{center}
  
    %   \begin{center}
    %       \huge \songti 翼型、叶栅空气动力学国家级重点实验室
    %   \end{center}
        \vspace{6cm}
        \begin{center}
        \songti
    %   \renewcommand\arraystretch{1.5}
    %   \begin{tabular}{p{2cm} c c}
    %       {学\,\,\,\,\,\,\,\,\,\,\,\,号:} & \multicolumn{2}{c}{2022303324}\\
    %       \cline{2-3}
    %       {姓\,\,\,\,\,\,\,\,\,\,\,\,名:} & \multicolumn{2}{c}{钱 \quad\quad 锋}\\
    %       \cline{2-3}
    %       {指导学院:} & \multicolumn{2}{c}{数学与统计学院} \\
    %       \cline{2-3}
    %       {指导教师:} & \multicolumn{2}{c}{魏杰, 范艾利} \\
    %       \cline{2-3}
    %   \end{tabular}
        \kaishu 西北工业大学数学与统计学院 \, \heiti\textbf{钱锋} \quad \songti 编
        \vspace{0.5cm}

    \today
    \end{center}
\end{titlepage}

\newpage
\cleardoublepage

% 使用罗马数字页码
\pagenumbering{roman}

\input{丛书前言.tex}

\chapter*{第一版前言}
\addcontentsline{toc}{chapter}{第一版前言}
\markboth{军事理论}{第一版前言}
% 设置前言标题页的页码格式为empty,即无页眉页脚
\thispagestyle{empty}

\quad\quad
大学生军事理论课是普通高等学校教学的重要组成部分,是学校开展国防教育、实现全民国防教育
的基础,是落实国防法的重要举措。

% 落款
\begin{flushright}
    \kaishu
    钱锋

    西北工业大学软件学院

    2024 年 1 月
\end{flushright}

\newpage
\thispagestyle{empty}

% 设置目录页的页码格式
\pagenumbering{roman} % 切换回罗马数字页码
\addtocontents{toc}{\protect\thispagestyle{empty}}
\pagestyle{plain}
{\small \tableofcontents}
\newpage
\thispagestyle{empty}

% 设置章节标题页的页眉和页脚为空白页样式
\makeatletter
\let\ps@plain\ps@empty
\makeatother


% 设置正文页的页码格式
\cleardoublepage % 确保正文从奇数页开始
\pagenumbering{arabic} % 切换为阿拉伯数字页码
\pagestyle{fancy}
\setcounter{page}{1} % 重置页码计数为1

% 正文从此开始

\chapter{军事思想}

\section{习近平强军思想}

党的十九大报告把党在新时代的强军目标完整表述为“建设一支听党指挥、能打胜仗、作风优良的人民军队,把人民军队建设成为世界一流军队”。
以“强军目标”为牵引,以“十一个明确”为主干,以“五个坚持”为精髓,习近平强军思想涵盖新时代军队建设、改革和军事斗争准备各领域各方面,贯通军事力量建设和运用全过程,形成了一个内涵丰富、结构严谨、与时俱进的思想体系。
“十一个明确” 的内容为:
\begin{enumerate}[label=(\arabic*)]
    \item 明确党对人民军队的绝对领导是人民军队建军之本、强军之魂,必须全面加强军队党的领导和党的建设,贯彻党领导军队的一系列根本原则和制度,确保部队绝对忠诚、绝对纯洁、绝对可靠。
    \item 明确强国必须强军,巩固国防和强大人民军队是新时代坚持和发展中国特色社会主义、实现中华民族伟大复兴的战略支撑,人民军队必须有效履行新时代使命任务。
    \item 明确党在新时代的强军目标是建设一支听党指挥、能打胜仗、作风优良的人民军队,到 2027 年实现建军一百年奋斗目标,到 2035 年基本实现国防和军队现代化,到本世纪中叶把人民军队建成世界一流军队。
    \item 明确军队是要准备打仗的,必须聚焦能打仗、打胜仗,扭住强敌对手,创新军事战略指导,发展人民战争战略战术,全面加强练兵备战,坚定灵活开展军事斗争,有效塑造态势、管控危机、遏制战争、打赢战争。
    \item 明确推进强军事业必须坚持政治建军、改革强军、科技强军、人才强军、依法治军,坚持边斗争、边备战、边建设,更加注重聚焦实战、创新驱动、体系建设、集约高效、军民融合,加强军事治理,推动高质量发展,全面提高革命化现代化正规化水平。
    \item 明确改革是强军的必由之路,必须推进军队组织形态现代化,构建中国特色现代军事力量体系,完善中国特色社会主义军事制度。
    \item 明确科技是核心战斗力,必须坚持自主创新战略基点,推进高水平科技自立自强,统筹推进军事理论、技术、组织、管理、文化等各方面创新,建设创新型人民军队。
    \item 明确强军之道要在得人,必须贯彻新时代军事教育方针,推动军事人员能力素质、结构布局、开发管理全面转型升级,锻造德才兼备的高素质、专业化新型军事人才。
    \item 明确依法治军是我们党建军治军基本方式,必须构建中国特色军事法治体系,推动治军方式根本性转变,提高国防和军队建设法治化水平。
    \item 明确军民融合发展是兴国之举、强军之策,必须巩固提高一体化国家战略体系和能力。
    \item 明确作风优良是我军鲜明特色和政治优势,必须全面从严治党、全面从严治军,全面锻造过硬基层,坚定不移正风肃纪反腐,大力弘扬我党我军光荣传统和优良作风,永葆人民军队性质、宗旨、本色。
\end{enumerate}
“五个坚持”,即坚持政治引领、坚持以武止戈、坚持积极进取、坚持统筹兼顾、坚持敢打必胜。

\newpage
\thispagestyle{empty}
\begin{thebibliography}{1}
    \addcontentsline{toc}{chapter}{参考文献}
    \markboth{中国近现代史纲要}{参考文献}
    \bibitem{2023版}
    《中国近现代史纲要(2023 版)》编写组编,中国近现代史纲要(2023 版),第 9 版,
    北京:高等教育出版社,2023.2
    \bibitem{教学辅导}
    赵海霞主编,西北工业大学马克思主义学院组织编写,中国近现代史纲要教学辅导,第 3 版,
    西安:西北工业大学出版社,2019.8
    \bibitem{两岸晚清}
    王建朗,黄克武主编,两岸新编中国近代史·晚清卷,2016.9
    \bibitem{两岸民国}
    王建朗,黄克武主编,两岸新编中国近代史·民国卷,2016.6
    \bibitem{人类简史}
    (以色列) 尤瓦尔·赫拉利(Yuval Noah Harari)著. 人类简史: 从动物到上帝(Sapiens: A Brief History
    of Humankind). 林俊宏译. 第 2 版.
    北京: 中信出版社.
    % \bibitem{口述史}
    % 钱锋,改革开放口述史:改革开放与党的反腐败工作——
    % 改革开放以来党的有关反腐败的具体司法实践工作的变化,2022.11
    \bibitem{核心考案}
    徐涛主编,考研政治核心考案,北京:中国政法大学出版社,2023.1
    \bibitem{优题库}
    徐涛主编,考研政治通关优题库,北京:中国政法大学出版社,2023.2
    \bibitem{真题库}
    徐涛主编,考研政治必刷真题库,北京:中国政法大学出版社,2023.2
    \bibitem{冲刺背诵笔记}
    徐涛主编,考研政治冲刺背诵笔记,北京:中国政法大学出版社,2023.9
    \bibitem{预测}
    徐涛主编,考研政治预测 6 套卷,北京:中国政法大学出版社,2023.10
    \bibitem{预测}
    徐涛,曲艺主编,考研政治考前预测必备 20 题,北京:中国政法大学出版社,2023.11
\end{thebibliography}



% [序号]主要责任者.电子文献题名.电子文献出处[电子文献及载体类型标识].或可获得地址,发表或更新日期/引用日期。

\newpage
\thispagestyle{empty}

\newpage
\cleardoublepage
\thispagestyle{empty}
\vspace*{3cm}
\begin{center}
    \includegraphics*[width=\textwidth]{pic/Keynote素材库.001.jpeg}
    \large
    公诚勇毅 \quad 永矢毋忘

    中华灿烂 \quad 工大无疆
\end{center}
\vspace*{7cm}
\begin{center}
    \small
    本文档由\textbf{钱锋}编写, 钱锋保留一切权利.

    文档中出现的部分素材来源于网络, 笔者承诺这些素材仅供学习交流之用, 
    它们的原作者保留一切权利.

    2023 年 \quad 西北工业大学 \quad 中国西安 
\end{center}

\end{document}