\documentclass[UTF8,10pt]{ctexbook} %% ctexart

\title{\textbf{习近平新时代中国特色社会主义思想概论}讲义}
\author{钱锋\thanks{Email: strik0r\_qf@mail.nwpu.edu.cn}}


\usepackage{graphicx}
\usepackage[toc]{multitoc}
\usepackage{booktabs}
\usepackage{longtable}
\usepackage{booktabs}
\usepackage{longtable}
\usepackage{amsthm, amssymb, amsmath, mathrsfs, mhchem}
\usepackage{tikz}
\usetikzlibrary{decorations.markings, angles, quotes}
\usepackage{pgfplots}
\usepackage{tikz-3dplot}
\usepackage{extpfeil}
\usepackage{diagbox}
\usepackage{multirow}
\usepackage{float}
\usepackage{hyperref}
\hypersetup{hidelinks,
    colorlinks = true,
    allcolors = black,
    pdfstartview = Fit,
    breaklinks = true}
\usepackage{caption}
\captionsetup[table]{labelsep=space} % 表
\captionsetup[figure]{labelsep=space} % 图
\usepackage{enumitem}
\usepackage{siunitx}
\usepackage{cuted}

\input{_titlesecsettings.tex}
\input{_fancyhdr_settings.tex}
\input{_listings_settings.tex}
\input{_geometry_setteings.tex}
\input{_mdframed_settings.tex}

\usepackage{smartdiagram}
\usepackage{tasks}

\begin{document}

\newtheoremstyle{mytheoremstyle}
    {1.5ex}                                         % Space above
    {1.5ex}                                         % Space below
    {}                                              % Font for body
    {}                                              % Indent amount
    {\bfseries}                                     % Font for head
    {}                                              % Punctuation after head
    {0.5em plus 0.2em minus 0.1em}                  % Space after head
    {\thmname{#1}\thmnumber{ #2}.\thmnote{ (#3).}}

\theoremstyle{mytheoremstyle} \newtheorem{example}{例}[section]
\theoremstyle{mytheoremstyle} \newtheorem{key}{核心要点}[section]

\theoremstyle{plain} \newtheorem{thm}{分析论述}

\newtheoremstyle{my3theoremstyle}
    {1.5ex}                                         % Space above
    {1.5ex}                                         % Space below
    {}                                              % Font for body
    {}                                              % Indent amount
    {\kaishu}                                       % Font for head
    {}                                              % Punctuation after head
    {0.5em plus 0.2em minus 0.1em}                  % Space after head
    {\thmname{#1}\thmnumber{ #2}.\thmnote{ (#3).}}

\theoremstyle{my3theoremstyle}
\newtheorem*{remark}{注}
\newtheorem*{cmt}{评注}
\newtheorem*{sol}{答案要点}

\begin{titlepage}
    \thispagestyle{empty}
    \centering
        \vspace*{2cm}
        \includegraphics[width=0.5\textwidth]{npu_2.png}\par
        \vspace{1cm}
        \includegraphics[width=0.5\textwidth]{npu_1.png}\par
    \vspace{1cm}
        \begin{center}
            \Huge \heiti \textbf{习近平新时代中国特色社会主义思想概论}
        \end{center}
        \vspace{6cm}
        \begin{center}
        \songti
        \kaishu 数学与统计学院 \, \heiti\textbf{钱锋} \quad \songti 编
        \vspace{0.5cm}

    \today
    \end{center}
\end{titlepage}

% 使用罗马数字页码
\cleardoublepage
\pagenumbering{roman}

% \chapter*{第一版前言}
% \addcontentsline{toc}{chapter}{第一版前言}
% \markboth{思想政治理论类课程讲义}{第一版前言}
% % 设置前言标题页的页码格式为empty,即无页眉页脚
% \thispagestyle{empty}

% 2022 年 11 月,西北工业大学数学与统计学院和小 “数” 苗成长支持辅导中心开展学霸直播间朋辈交流活动,
% 学霸直播间活动是我校数学与统计学院基于费曼学习法的,贯彻 “思政内容进课堂”、“我为同学办实事” 的,
% 倡导朋辈交流、互助的一次有力创新,是我院在 2022 级本科大类班与强基班开展本科生三年学业学风建设行动的重要手段。
% 我受聘成为小 “数” 苗成长支持辅导中心朋辈 “小导师”,负责期末复习基础课和学霸直播间的统筹协调工作,
% 以及数学分析、高等代数、空间解析几何、大学美育 (美学原理)、军事理论和中国近现代史纲要等课程基础知识的讲解。

% 这一轮授课形成了若干书面讲义和视频素材,如今我们将其系统地整理成册,供西北工业大学数学与统计学院
% 以及全网有相关课程学习需求的同学参考。
% 在刚开始进行本册资料的编写的时候,我原本只打算制作一份简易的考试复习资料。
% 当时我简单地罗列了《中国近现代史纲要》课程中各个章节中经常考察的知识点,
% 并将中国近现代史纲要课程的授课视频上传至网络,
% 没想到吸引了来自全网的 50 多万人次的浏览和点击。对于这样的流量数据和一个考前突击视频的爆火,
% 我既受宠若惊又诚惶诚恐,一方面,由于这么多年以来不温不火的视频创作,涨粉期的到来让我深感欣慰,
% 自觉自己的努力并没有白费;
% 另一方面,我毕竟只是一个正在本科阶段学习的学生,对相关知识点的了解、掌握和解读都不是那么的全面,
% 巨大的流量数据意味着更多的监督和批评,也意味着对创作内容质量的更高要求,
% 这要求我必须进一步、更深层次地进行相关课程的学习,才能对得起广大老师、同学和网友的期待。

% 在课程速成视频走红后,有许多网友希望我将速成视频中讲义分享出来,以便于复习使用,
% 于是在经过简单的整理后,本书的雏形就这样诞生了。pdf 格式的讲义经历了几次迭代后,
% 有越来越多的网友建议我制作一份能兼顾课前预习、课堂学习、课后知识巩固、课外知识拓展、
% 期末速成和考研复习等多维度学习需求的全面的学习资料,
% 于是本书的内容便不断扩充,字数日益膨胀,最终形成了现在的这样一份资料。

% \subsection*{本书编写的指导思想}

% 在编写本书的过程中,我们遵循以下风格:

% \subsubsection{深入贯彻思政育人理念,将思政课理论知识与中国特色社会主义新时代建设具体实践相结合,
% 与我校三航特色的科学研究与技术创新具体实践相结合。}

% 你瓜是这样的。

% \subsubsection{体现新时代中国青年的时代风貌。}

% 青年是这样的。

% \subsection*{本书的内容组织}
% \setcounter{subsubsection}{0}

% 本书共五个部分,分别对应除形势与政策外的五门必修的思想政治理论类课程。每门思政课按照教材章节顺序组织,
% 每章节分为内容精华、章节自主测试题和拓展提高三个模块。

% \subsubsection{内容精华}

% 内容精华部分罗列了本章节的学习重点、难点以及容易出现的疑难问题,这一部分可作为读者课前预习、课上参考和课后复习
% 的材料使用。其中,\textbf{加粗的内容是重点中的重点,是经常考察的内容,是选择题的选项或分析论述题的关键词},
% 不论是出于何种目的学习,这些加粗内容都应该是读者阅读本书时重点关注的对象。时间实在紧迫的读者,可以只看加粗的内容。

% 同时,我们针对每章节的内容精华部分制作了视频讲解,视频讲解的语速、节奏较快,知识点密度较高且重点突出,
% 非常适合期末速成、突击的同学收看。

% \subsubsection{章节自主测试题}

% 该模块收录了近几年全国硕士研究生入学统一考试的考试真题,以及部分历年期末考试真题。
% 读者可以在完成章节的学习后通过这些习题检测一下对本章知识的掌握情况。需要强调的是,
% \textbf{在学习和复习的过程中不能淡化对分析论述题的训练,要功在平时,
% 养成良好的 “观点+分析” 的分点论述作答习惯},才不至于在考试中面对材料分析题和论述题无从下手。

% \subsubsection{拓展提高}

% 该模块收录了一些课程外阅读材料的精华内容。通过这些材料的阅读,读者能更好地理解本章的重点知识,
% 能将内容精华部分罗列的离散的知识点串成一张网络,形成对知识点的全局视角和更高层次的理解。

% 这一部分的讲解视频将作为我的高级包月充电专属视频发布,高级包月充电费用为 30 元/月。
% 这些内容都不是通过考试或考试拿高分所必须掌握的知识,因此请各位同学不要盲目开通高级包月充电。

% \subsection*{如何使用本书}
% \setcounter{subsubsection}{0}

% 随便用

% % 落款
% \begin{flushright}
%     \kaishu
%     钱锋

%     西北工业大学数学与统计学院

%     2023 年 9 月
%     \songti
% \end{flushright}

\newpage
\thispagestyle{empty}

% 设置目录页的页码格式
\pagenumbering{roman} % 切换回罗马数字页码
\addtocontents{toc}{\protect\thispagestyle{empty}}
\pagestyle{plain}

\tableofcontents
\newpage
\thispagestyle{empty}

% 设置章节标题页的页眉和页脚为空白页样式
\makeatletter
\let\ps@plain\ps@empty
\makeatother



\chapter*{开始之前……}

这本书还没有写完, 且书中的所有内容均为学习交流之用, 请于下载后 24 小时内删除.

% 设置正文页的页码格式
\cleardoublepage % 确保正文从奇数页开始
\pagenumbering{arabic} % 切换为阿拉伯数字页码
\pagestyle{fancy}
\setcounter{page}{1} % 重置页码计数为1
% \chapter{改革开放与中国特色社会主义的开创与发展}

% \section{内容精华}

% \subsection{历史性的伟大转折和改革开放的进步}

% \subsubsection{历史性的伟大转折和中国特色社会主义的开创}

% 十一届三中全会的全部内容
%     \begin{enumerate}[itemsep=0pt]
%         \item 冲破长期“左”的严重束缚;
%         \item 彻底否定“两个凡是”的错误方针;
%         \item 高度评价关于真理标准问题的讨论;
%         \item 果断停止使用“以阶级斗争为纲”的口号;
%         \item 决定把全党的工作重心转移到社会主义现代化建设上来;
%         \item 提出了改革开放的任务。
%     \end{enumerate}

% 十一届三中全会的历史意义
%     \begin{enumerate}[itemsep=0pt]
%         \item 结束了粉碎“四人帮”后党和国家工作在徘徊中前进的局面;
%         \item 标志着中国共产党重新确立了马克思主义的思想路线、政治路线、组织路线;
%         \item 实现了新中国成立以来党的历史上具有深远意义的伟大转折,开启了我国改革开放和社会主义现代化建设新时期。
%     \end{enumerate}

% \begin{thm}
%     邓小平南方谈话的主要内容是什么?有什么重要意义?
%     \begin{sol}
%         \begin{enumerate}[itemsep=0pt, label=(\arabic*)]
%             \item 南方谈话的主要内容是:第一,计划多一点还是市场多一点,不是社会
%             主义与资本主义的本质区别,计划和市场都是经济手段;第二,阐明了社
%             会主义本质;第三,提出了“发展才是硬道理”的重要论断,科学技术是第一生产
%             力;第四,提出判断改革开放和各项工作成败得失的“三个有利于”标准;第五,
%             强调加强党的建设;第六,关于社会主义初级阶段的长期性和前途。
%             \item 南方谈话的重要意义是:邓小平南方谈话,在重大历史关头科学地总结了党的
%             十一届三中全会以来党的基本实践和基本经验,明确回答了长期困扰和束缚人们
%             思想的许多重大认识问题,对整个社会主义现代化建设事业产生了重大而深远的
%             影响。
%         \end{enumerate}
%     \end{sol}
% \end{thm}

% \begin{thm}
%     为什么说党的十一届三中全会是新中国成立以来的伟大转折?
%     \begin{sol}
%         党的十一届三中全会彻底否定了“两个凡是”的错误方针,重新确立了“解放思想、
%         实事求是”的指导思想,实现了思想路线的拨乱反正;果断地停止使用“以阶级斗
%         争纲”的口号,做出工作重点转移的决策,实现了政治路线的拨乱反正;形成了
%         以邓小平为核心的党中央领导集体,取得了组织路线拨乱反正的最重要成果;恢
%         复了党的民主集中制的优良传统,提出使民主制度化、法律化的重要任务;审查
%         和解决历史上遗留的一批重大问题和一些重要领导人的功过是非问题,开始了系
%         统清理重大历史是非的拨乱反正。结束了粉碎“四人帮”后两年在徘徊中前进的局
%         面,揭开了我国改革开放的序幕,开辟了建设中国特色社会主义的新道路,标志
%         着中国从此进入了改革开放和社会主义现代化建设的历史新时期。因此,党的十
%         一届三中全会是新中国成立以来党和国家历史上的伟大转折。
%     \end{sol}
% \end{thm}

% \subsubsection{拨乱反正的完成}

% \begin{thm}
%     关于真理标准问题讨论的意义是什么?
%     \begin{sol}
%         \textbf{实践是检验真理的唯一标准}。真理标准问题的讨论是继五四运动和延安
%         整风运动之后又一场马克思主义思想解放运动,其实质在于是否坚持马列主义、毛
%         泽东思想。这场讨论明确地解决了党的思想路线问题,继而影响了党的政治路线的
%         制定和贯彻,成为拨乱反正和改革开放的思想先导,为党重新确立实事求是思想路
%         线,纠正长期以来的“左”倾错误,实现历史性的转折做了思想理论准备。
%     \end{sol}
% \end{thm}

% 《关于建国以来若干历史问题的决议》的主要内容:
% \begin{enumerate}[itemsep=0pt]
%     \item 确立毛泽东同志的历史地位,坚持和发展毛泽东思想;
%     \item 对建国30年以来的历史大事,要进行实事求是的分析,包括一些负责同志的是非功过,要做出公正的评价;
%     \item 这个总结宜粗不宜细,总结过去是为了引导大家团结一致向前看。
% \end{enumerate}


% \subsubsection{改革开放的起步}

% \subsection{改革开放和社会主义现代化建设新局面}

% \subsubsection{改革开放的展开}
% \subsubsection{加强和完善党的领导}
% \subsubsection{改革开放和现代化建设的深入推进}
% \subsubsection{国防战略的转变、“一国两制” 方针的形成和外交方针政策的调整}
% \subsubsection{中国特色社会主义事业的继续推进}

% \subsection{把中国特色社会主义全面推向 21 世纪}

% “五位一体”战略布局和“四个全面”战略布局
% \begin{enumerate}[itemsep=0pt]
%     \item 五位一体:经济、政治、文化、社会、生态文明
%     \item 四个全面:建成小康社会、深化改革、依法治国、从严治党
% \end{enumerate}
% 中国特色社会主义经济建设、政治建设、文化建设、社会建设、生态文明建设“五位一体”总体布局和全面建成小康社会、全面深化改革、全面依法治国、全面从严治党“四个全面”战略布局统筹联动、相互促进,有力推动了理论创新和实践创新的步伐。

% \subsubsection{社会主义市场经济体制改革目标和基本框架的确立}
% \subsubsection{改革开放和现代化建设的跨世纪发展}
% \subsubsection{香港、澳门回归祖国与两岸交流扩大}
% \subsubsection{推进党的建设新的伟大工程}

% \subsection{在新的形势下坚持和发展中国特色社会主义}

% \subsubsection{全面建设小康社会宏伟目标的提出}
% \subsubsection{全面建设小康社会新部署和改革开放的深化}
% \subsubsection{推进 “一国两制” 实践与祖国和平统一大业}


% \begin{thm}
%     “一国两制”构想的基本内容是什么?这项国策有什么意义?
%     \begin{sol}
%         \begin{enumerate}[itemsep=0pt, label=(\arabic*)]
%             \item “一国两制”构想的基本内容。第一,“一国两制”的基础是“一个中国”。
%             世界上只有一个中国,即台湾、香港、澳门都是领土不可分割的组成部分,在国
%             际上代表中国的只能是中华人民共和国。第二,“一国两制”的主体是社会主义。
%             大陆实行社会主义,是中国的主体。大陆这个主体,坚定不移地实行社会主义。
%             在这个前提下,可以容许在自己身边,在小地区和小范围内实行资本主义。第三,
%             实行“一国两制”。在统一的中国境内,大陆地区坚持社会主义制度,台湾、香港、
%             澳门则保持原有的资本主义制度和生活方式。第四,两种制度长期共存、和平共
%             处、相互支援、共同发展。
%             \item “一国两制”的意义。实行“一国两制”长期不变的方针,在一个中国内,
%             实行和资本主义制度长期共存、和平共处、共同发展,至少五十年不变。这个构
%             想不仅是对马克思主义国家学说的创造性发展,还为和平解决某些相关的历史遗
%             留问题指明了出路。
%         \end{enumerate}
%     \end{sol}
% \end{thm}

% \subsubsection{提高党的科学建设化水平}

% \section{章节自主测试题}

% \chapter{中国特色社会主义进入新时代}

% \section{内容精华}

% \begin{enumerate}[itemsep=0pt]
    
    
%     \item 两个维护
    
%     \begin{enumerate}[itemsep=0pt]
%         \item 坚决维护习近平总书记的核心地位
%         \item 坚决维护党中央权威和集中统一领导
%     \end{enumerate}
    
%     \item 新发展理念: 创新、协调、绿色、开放、共享。
% \end{enumerate}

% \section{章节自主测试题}

% \part{马克思主义基本原理}

% \chapter*{绪论}
% \addcontentsline{toc}{chapter}{绪论}
% \markboth{马克思主义基本原理}{绪论}
% % 这里是绪论的内容

% \part{思想道德与法治}

% \chapter*{绪论}
% \addcontentsline{toc}{chapter}{绪论}
% \markboth{思想道德与法治}{绪论}
% % 这里是绪论的内容



% \part{毛泽东思想与中国特色社会主义理论体系概论}

% \chapter*{绪论}
% \addcontentsline{toc}{chapter}{绪论}
% \markboth{毛泽东思想与中国特色社会主义理论体系概论}{绪论}
% % 这里是绪论的内容



\chapter*{导论}
\thispagestyle{empty}
\addcontentsline{toc}{chapter}{导论}
\markboth{习近平新时代中国特色社会主义思想概论}{导论}
% 这里是绪论的内容
\setcounter{chapter}{0}
\section{习近平新时代中国特色社会主义思想创立的时代背景}


时代是思想之母,实践是理论之源。
这是一个需要思想理论的时代,是一个产生思想理论的时代,
也是一个在伟大变革中不断推动思想理论向前发展的时代。
当前,
\textbf{世界百年未有之大变局加速演进}、\textbf{中华民族伟大复兴进入关键时期}、\textbf{中国式现代化全面推进拓展}、
\textbf{科学社会主义在 21 世纪的中国焕发新的蓬勃生机}、\textbf{中国共产党自我革命开辟新的境界}。
习近平新时代中国特色社会主义思想正是在这个时代中创立并不断丰富发展的。


\textbf{中国共产党第十九次全国代表大会},把习近平新时代中国特色社会主义思想确立为党必须长期坚持的
指导思想并庄严地写入党章,实现了党的指导思想的与时俱进。



\section{习近平新时代中国特色社会主义思想是 “两个结合” 的重大成果}

\setcounter{subsubsection}{0}
\subsubsection{“两个结合” 的内容}

习近平新时代中国特色社会主义思想是马克思主义基本原理同\textbf{中国具体实际}相结合,
同\textbf{中华优秀传统文化}相结合的重大成果。

习近平新时代中国特色社会主义思想强调要提高科学思维能力,
要坚持系统观念,要强化问题导向等,是对\textbf{马克思主义认识论}的新发展。
习近平新时代中国特色社会主义思想提出人与自然是和谐共生的生命共同体,
绿水青山就是金山银山,是对\textbf{马克思主义自然观}的新发展。

习近平新时代中国特色社会主义思想汲取中华优秀传统文化所蕴含的丰富哲学思想、
人文精神、道德理念,有效激活了中华优秀传统文化的生命力,成为中华优秀传统文化
的创造性转化、创新性发展的生动典范。



\subsubsection{“两个结合” 的地位}

“两个结合” 是我们党在探索中国特色社会主义道路中得出的规律性认识,是我们取得成功的最大法宝,
是对坚持和发展马克思主义作出的重大理论贡献,是又一次的思想解放。

\subsubsection{“两个结合” 的伟大意义}

“两个结合” 让中国特色社会主义道路有了更加宏阔深远的历史纵深,
拓展了中国特色社会主义道路的文化根基,
开辟了广阔的理论和实践创新空间,
表明我们党对中国道路、理论、制度的认识达到了新高度,
表明我们党的历史自信、文化自信达到了新高度,
表明我们党在传承中华优秀传统文化中推进理论创新的自觉性达到了新高度。

\section{习近平新时代中国特色社会主义思想是完整的科学体系}
\setcounter{subsubsection}{0}

习近平新时代中国特色社会主义思想,从理论和实践的结合上科学回答了新时代坚持和发展什么样的\textbf{中国特色
社会主义}、怎样坚持和发展中国特色社会主义,建设什么样的\textbf{社会主义现代化强国}、怎样建设社会主义现代
化强国,建设什么样的\textbf{长期执政的马克思主义政党}、怎样建设长期执政的马克思主义政党等重大时代课题,
以崭新的思想内容丰富发展了马克思主义,形成了完整的科学体系。

\subsection{“十个明确” 的内容及其意义}

\begin{enumerate}[itemsep=0pt, label=(\arabic*)]
    \item 明确中国特色社会主义最本质的特征是中国共产党领导,中国特色社会主义制度的最大优势是中国共产党领导,中国共产党是最高政治领导力量,全党必须增强“四个意识”、坚定“四个自信”、做到“两个维护”;
    \item 明确坚持和发展中国特色社会主义,总任务是实现社会主义现代化和中华民族伟大复兴,在全面建成小康社会的基础上,分两步走在本世纪中叶建成富强民主文明和谐美丽的社会主义现代化强国,以中国式现代化推进中华民族伟大复兴;
    \item 明确新时代我国社会主要矛盾是人民日益增长的美好生活需要和不平衡不充分的发展之间的矛盾,必须坚持以人民为中心的发展思想,发展全过程人民民主,推动人的全面发展、全体人民共同富裕取得更为明显的实质性进展;
    \item 明确中国特色社会主义事业总体布局是经济建设、政治建设、文化建设、社会建设、生态文明建设五位一体,战略布局是全面建设社会主义现代化国家、全面深化改革、全面依法治国、全面从严治党四个全面;
    \item 明确全面深化改革总目标是完善和发展中国特色社会主义制度、推进国家治理体系和治理能力现代化;
    \item 明确全面推进依法治国总目标是建设中国特色社会主义法治体系、建设社会主义法治国家;
    \item 明确必须坚持和完善社会主义基本经济制度,使市场在资源配置中起决定性作用,更好发挥政府作用,把握新发展阶段,贯彻创新、协调、绿色、开放、共享的新发展理念,加快构建以国内大循环为主体、国内国际双循环相互促进的新发展格局,推动高质量发展,统筹发展和安全;
    \item 明确党在新时代的强军目标是建设一支听党指挥、能打胜仗、作风优良的人民军队,把人民军队建设成为世界一流军队;
    \item 明确中国特色大国外交要服务民族复兴、促进人类进步,推动建设新型国际关系,推动构建人类命运共同体;
    \item 明确全面从严治党的战略方针,提出新时代党的建设总要求,全面推进党的政治建设、思想建设、组织建设、作风建设、纪律建设,把制度建设贯穿其中,深入推进反腐败斗争,落实管党治党政治责任,以伟大自我革命引领伟大社会革命。
\end{enumerate}

\textbf{“十个明确” 的意义} \quad“十个明确” 是习近平新时代中国特色社会主义思想的\textbf{主体内容},集中体现了这一思想体系的\textbf{主要观点}和\textbf{基本精神},
构成了这一思想体系的四梁八柱,发挥着\textbf{统摄作用}。

\subsection{“十四个坚持” 的内容及其意义}

\textbf{十四个坚持} \quad 坚持党对一切工作的领导;
坚持以人民为中心;
坚持全面深化改革;
坚持新发展理念;
坚持人民当家作主;
坚持全面依法治国;
坚持社会主义核心价值体系;
坚持在发展中保障和改善民生;
坚持人与自然和谐共生;
坚持总体国家安全观;
坚持党对人民军队的绝对领导;
坚持“一国两制”和推进祖国统一;
坚持推动构建人类命运共同体;
坚持全面从严治党。

\textbf{“十四个坚持” 的意义} \quad “十四个坚持” 是习近平新时代中国特色社会主义思想的重要组成部分,
是在治国理政各方面作出的\textbf{理论分析}和\textbf{政策指导},构成了新时代坚持和发展中国特色社会主义的\textbf{基本方略}。

\subsection{“十三个方面成就” 的内容及其意义}

\textbf{十三个方面成就} \quad 在坚持党的全面领导、全面从严治党、经济建设、全面深化改革开放、
政治建设、全面依法治国、文化建设、社会建设、生态文明建设、国防和军队建设、维护国家安全、
坚持“一国两制”和推进祖国统一、外交工作等方面取得的历史性成就和发生的历史性变革。

\textbf{“十三个方面成就” 的意义} \quad “十三个方面成就” 对新时代伟大实践进行了\textbf{科学总结},
全景式地展示了习近平新时代中国特色社会主义思想的\textbf{理论与实践成果}。

\subsection{“六个必须坚持” 的内容及其意义}

\textbf{六个必须坚持} \quad 必须坚持人民至上、必须坚持自信自立、必须坚持守正创新、
必须坚持问题导向、必须坚持系统观念、必须坚持胸怀天下。

\textbf{“六个必须坚持” 的意义} \quad “六个必须坚持” 彰显了习近平新时代中国特色社会主义思想的理论品格和鲜明特征。

\section{习近平新时代中国特色社会主义思想的历史地位}

\begin{enumerate}[itemsep=0pt]
    \item 习近平新时代中国特色社会主义思想是当代中国马克思主义、二十一世纪马克思主义,
    是中华文化和中国精神的时代精华,\textbf{实现了马克思主义中国化时代化新的飞跃}。
    
    \item 习近平新时代中国特色社会主义思想,把马克思主义基本原理同中国具体实际相结合、同中华优秀传统文化相结合,
    使马克思主义这个魂脉和中华优秀传统文化这个根脉内在贯通、相互成就,是中华民族的文化主体性最有力的体现,
    是中华文化和中国精神的时代精华。
    
    \item 习近平新时代中国特色社会主义思想,是全党全国各族人民为实现中华民族伟大复兴而奋斗的行动指南,是新时代
    党和国家事业发展的根本遵循。
\end{enumerate}

\section{深刻领悟 “两个确立” 的决定性意义}
\setcounter{subsubsection}{0}

\subsubsection{“两个确立” 的内容}

\textbf{党的十九届六中全会}指出,党确立\textbf{习近平同志党中央的核心、全党的核心地位},
确立\textbf{习近平新时代中国特色社会主义思想的指导地位},反映了全党全军全国各族人民共同心愿,
对新时代党和国家事业发展,对推进中华民族伟大复兴历史进程具有决定性意义。

\subsubsection{“两个确立” 的决定性意义}

新时代的伟大实践充分证明,“两个确立” 是新时代党和国家事业\textbf{取得历史性成就}、\textbf{发生历史性变革}的决定性
因素,是党和人民应对一切不确定性的最大确定性、最大底气、最大保证。

\newpage
\section*{自测题}
\addcontentsline{toc}{section}{自测题}
\setcounter{example}{0}

\begin{example}
    中国特色社会主义最本质的特征是 \underline{\qquad \qquad},
    中国特色社会主义制度的最大优势是 \underline{\qquad \qquad}。
\end{example}

\begin{example}
    中国共产党是最高政治领导力量,全党必须增强 \underline{\qquad \qquad \qquad},
    坚定 \underline{\qquad \qquad \qquad},做到 \underline{\qquad \qquad \qquad}。
\end{example}

\begin{remark}
    明确中国特色社会主义最本质的特征是中国共产党领导,中国特色社会主义制度的最大优势是中国共产党领导,中国共产党是最高政治领导力量,全党必须增强“四个意识”、坚定“四 个自信”、做到“两个维护”。
\end{remark}

\begin{example}
    坚持和发展社会主义,总任务是 \underline{\qquad \qquad \qquad}
    和 \underline{\qquad \qquad \qquad}。
\end{example}

\begin{example}
    在全面建成小康社会的基础上,分两步走在 \underline{\qquad \qquad \qquad} 建成
    \underline{\qquad \qquad \qquad \qquad \qquad},以 \underline{\qquad \qquad \qquad}
    全面推进中华民族伟大复兴。
\end{example}

\begin{remark}
    明确坚持和发展中国特色社会主义,总任务是实现社会主义现代化和中华民族伟大复兴,在全面建成小康社会的基础上,分两步走在本世纪中叶建成富强民主文明和谐美丽的社会主义现代化强国,以中国式现代化推进中华民族伟大复兴。
\end{remark}

\begin{example}
    新时代我国社会主要矛盾是 \underline{\qquad \qquad \qquad \qquad} 和
    \underline{\qquad \qquad \qquad \qquad} 之间的矛盾。
\end{example}

\begin{example}
    必须坚持以 \underline{\qquad \qquad \qquad} 为中心的发展思想,发展全过程人民民主,
    推动 \underline{\qquad \qquad \qquad}、\underline{\qquad \qquad \qquad} 取得更为明显的实质性进展。
\end{example}

\begin{remark}
    明确新时代我国社会主要矛盾是人民日益增长的美好生活需要和不平衡不充分的发展之间的矛盾,必须坚持以人民为中心的发展思想,发展全过程人民民主,推动人的全面发展、全体人民共同富裕取得更为明显的实质性进展。
\end{remark}

\begin{example}
    (多选)习近平新时代中国特色社会主义思想汲取中华优秀传统文化所蕴含的丰富 
    \underline{\qquad \qquad \qquad \qquad},
    有效激活了中华优秀传统文化的生命力,成为中华优秀传统文化
    的创造性转化、创新性发展的生动典范。
    \begin{tasks}[label={\Alph*. }](4)
        \task 哲学思想
        \task 人文精神
        \task 道德理念
        \task 政治智慧
    \end{tasks}
    \begin{remark}
        中国封建社会政治上实行高度集权的封建君主专制制度,因此不选择政治智慧。故选择 ABC。
    \end{remark}
\end{example}



\newpage
\thispagestyle{empty}

\chapter{新时代坚持和发展中国特色社会主义}

\section{方向决定道路,道路决定命运}

\textbf{道路问题}是关系党的事业成败兴衰\textbf{第一位的问题}。坚持和发展中国特色社会主义是改革开放以来我们党全部理论和实践的鲜明主题。

\subsection{中国特色社会主义是历史和人民的选择}

一个国家实行什么样的主义,关键要看这个主义能否解决这个国家面临的历史性课题。
\textbf{什么是社会主义、怎样建设社会主义},是中国共产党人思考的\textbf{基本问题}。
中国特色社会主义具有深厚的历史渊源和广泛的现实基础,是实现中华民族伟大复兴的正确道路。

\subsection{中国特色社会主义是社会主义而不是其他什么主义}

中国特色社会主义是\textbf{科学社会主义理论逻辑}和\textbf{中国社会发展历史逻辑}的辩证统一。
中国特色社会主义之所以是社会主义,而不是其他什么主义,就是因为我们始终坚持科学社会主义基本原则,
并根据新的时代条件赋予其鲜明的中国特色。

中国特色社会主义写出了科学社会主义的 “新版本”。当代中国的伟大社会变革,不是简单延续我国历史文化的母版,
不是简单套用马克思主义经典作家设想的模板,不是其他国家社会主义实践的再版,也不是国外现代化发展的翻版。

\subsection{坚定道路自信、理论自信、制度自信、文化自信}



\section{中国特色社会主义进入新时代}

\subsection{中国特色社会主义新时代是我国发展新的历史方位}

习近平总书记在中国共产党第十九次全国代表大会上明确提出,
“中国特色社会主义进入了新时代”。
新时代是我国发展新的历史方位,标志着中国特色社会主义事业进入了新的发展阶段。

\subsubsection{中国特色社会主义新时代的内涵}


\begin{enumerate}[itemsep=0pt, label=\arabic*)]
    \item 历史脉络:是承前启后、继往开来,在新的历史条件下继续夺取中国特色社会主义伟大胜利的时代;
    \item 实践主题,是决胜全面建成小康社会,进而全面建设社会主义现代化强国的时代;
    \item 人民性:是全国各族人民团结奋斗、不断创造美好生活,逐步实现全体人民共同富裕的时代;
    \item 民族性:是全体中华儿女戮力同心,奋力实现中华民族伟大复兴的时代;
    \item 世界性:是我国不断为人类做出更大贡献的时代。
\end{enumerate}

\subsubsection{中国特色社会主义进入新时代的历史意义}

中国特色社会主义进入新时代,从历史进程的角度看,意味着近代以来久经磨难的中华民族迎来了从站起来、
富起来到强起来的伟大飞跃,迎来了实现中华民族伟大复兴的光明前景;
从社会主义发展进程的角度看,意味着科学社会主义在 21 世纪的中国焕发出强大生机活力,
在世界上高高举起了中国特色社会主义伟大旗帜;
从人类文明进程的角度看,意味着中国特色社会主义道路、理论、制度、文化不断发展,
拓展了发展中国家走向现代化的途径,给世界上那些既希望加快发展又希望保持自身独
立性的国家和民族提供了选新选择,为解决人类问题贡献了中国智慧和中国方案。

\subsubsection{中国特色社会主义进入新时代是新变化的反映}

中国特色社会主义进入新时代,是我国社会主要矛盾发生新变化的反映。
党的十九大报告指出,我国社会矛盾已由
人民日益增长的物质文化需要同落后的社会生产之间的矛盾,转化为\textbf{人民日益增长的美好生活需要
和不平衡不充分的发展之间的矛盾}。其中,矛盾的主要方面表现在发展的不平衡性和不充分性上。
\textbf{新时代我国社会主要矛盾的变化,是在社会主义初级阶段中发生的变化},没有改变对我国社会所处历史阶段的判断。
我们既要看到我国社会主要矛盾发生的变化,也要看到我国仍处于并将长期处于社会主义初级阶段的基本国情没有变,
我国是世界最大发展中国家的国际地位没有变。

中国特色社会主义进入新时代,是党的主要任务发生新变化的反映。
党的十八大后,党面临的主要任务是,\textbf{实现第一个百年奋斗目标,开启实现第二个百年奋斗目标新征程,
朝着实现中华民族伟大复兴的宏伟目标继续前进}。

中国特色社会主义进入新时代,是中国和世界关系发生新变化的反映。

\begin{example}
    (多选)党的十九大报告指出,中国特色社会主义进入新时代,我国社会的主要矛盾已经转化为
    人民日益增长的美好生活需要和不平衡不充分的发展之间的矛盾。我国社会的主要矛盾的变化
    没有改变 \underline{\qquad \qquad \qquad}。
    \begin{tasks}[label={\Alph*.}](1)
        \task 我们对我国社会主义所处历史阶段的判断
        \task 我国仍处于并将长期处于社会主义初级阶段的基本国情
        \task 我国是世界最大发展中国家的国际地位
        \task 我们对我国发展所处历史方位的判断
    \end{tasks}
    \begin{cmt}
        党的十九大报告指出,“经过长期努力,中国特色社会主义进入了新时代,这是我国发展新的
        历史方位”,因此 D 项表述错误,本题选择 ABC。
    \end{cmt}
\end{example}

\subsection{新时代伟大变革的里程碑意义}

新时代的伟大变革,在党史、新中国史、改革开放史、社会主义发展史、中华民族发展史上具有里程碑意义。
走过百年奋斗历程的中国共产党在革命性锻造中更加坚强有力,在坚持和发展中国特色社会主义的历史进程
中始终成为坚强领导核心;中国人民的前进动力更加强大、奋斗精神更加昂扬、必胜信念更加坚定;改革开
放和社会主义现代化建设深入推进,实现中华民族伟大复兴进入了不可逆转的历史进程;科学社会主义在 21
世纪的中国焕发出新的蓬勃生机,中国式现代化为人类实现现代化提供了新的选择,中国共产党和中国人民为
解决人类面临的共同问题提供更多更好的中国智慧、中国方案、中国力量,为人类和平与发展崇高事业作出
新的更大的贡献!

\section{新时代坚持和发展中国特色社会主义要一以贯之}

\subsection{全面贯彻党的基本理论、基本路线、基本方略}

历史和现实充分证明,我们党团结带领全国各族人民探索形成的党的基本理论、基本路线、基本方略是完全正确的,
必须长期坚持、全面贯彻。

\subsection{统筹推进 “五位一体” 总体布局和协调推进 “四个全面” 战略布局}

中国特色社会主义事业总体布局是\textbf{经济建设、政治建设、文化建设、社会建设、生态文明建设五位一体}的总体布局,
是我们党对社会主义建设规律在实践和认识上不断深化的重要成果。

中国特色社会主义事业战略布局是\textbf{全面建设社会主义现代化国家、全面深化改革、全面依法治国、全面从严治党四个全面}的战略布局,
是新时代条件下坚持和发展中国特色社会主义、推进改革开放和社会主义现代化建设的战略抉择。
全面建设社会主义现代化国家是战略目标,在 “四个全面” 中居于领先地位;全面深化改革、全面依法治国、全面从严治党是三大战略举措,
为全面建设社会主义现代化国家提供重要保障。

\subsection{推动中国特色社会主义不断开拓前进}

社会主义是前无古人的伟大事业,没有现成的方案可以遵循,
必须随着时代、实践和科学的发展不断探索前进。

\section{自测题}
\setcounter{example}{0}

\begin{example}
    (判断题). 新时代是全国各族人民团结奋斗、不断创造美好生活、迅速实现全体人民共同富裕的时代。
    \begin{sol}
        错误。新时代是全国各族人民团结奋斗、不断创造美好生活、逐步实现全体人民共同富裕的时代。
    \end{sol}
\end{example}



\newpage
\thispagestyle{empty}

\chapter{以中国式现代化全面推进中华民族伟大复兴}

新时代新征程中国共产党的\textbf{核心任务},就是团结带领全国各族人民全面建成社会主义现代化强国、
实现第二个百年奋斗目标,以中国式现代化全面推进中华民族伟大复兴。

\section{中华民族近代以来最伟大的梦想}

实现中华民族伟大复兴,是近代以来中国人民的共同梦想,是中国共产党矢志不渝的\textbf{奋斗目标}。

\subsection{实现中华民族伟大复兴的中国梦}

一百年来,中国共产党团结带领中国人民进行的一切奋斗、一切牺牲、一切创造,
归结起来就是\textbf{一个主题}:实现中华民族的伟大复兴。
实现中华民族伟大复兴的中国梦,本质是国家富强、民族振兴、人民幸福。
中国梦就是要让每个人获得发展自我和奉献社会的机会,共同享有人生出彩的机会,共同享有梦想成真的机会,
共同享有同祖国和时代一起成长和进步的机会。

\subsection{在中华大地上全面建成小康社会}

全面建成小康社会是我国社会主义现代化进程中的一座重要里程碑,是实现中华民族伟大复兴中国梦的关键一步。
全面建成小康社会,成效卓著、成果丰硕、成色十足。高质量完成了社会主义现代化建设 “三步走” 战略的第二步,
使中华民族伟大复兴向前迈出了新的一大步。探索了人类解决贫困问题的新路径,为推动人类文明进步作出了重大贡献。

\subsection{全面建成社会主义现代化强国}

党的二十大进一步明确,全面建成社会主义现代化强国总的战略安排是分两步走:从 2020 年到 2035 年
基本实现社会主义现代化;从 2035 年到本世纪中叶把我国建设成富强民主文明和谐美丽的社会主义现代化强国。

\section{中国式现代化是强国建设、民族复兴的唯一正确道路}

\subsection{中国式现代化是中国共产党领导人民长期探索和实践的重大成果}
\subsection{中国式现代化的中国特色}

习近平指出,“中国式现代化,是中国共产党领导的社会主义现代化,
既有各国现代化的共同特征,更有基于自己国情的中国特色。”

(1) 中国式现代化是\textbf{人口规模巨大}的现代化。
(2) 中国式现代化是\textbf{全体人民共同富裕}的现代化。
(3) 中国式现代化是\textbf{物质文明和精神文明相协调}的现代化。
物质富足、精神富有是社会主义现代化的根本要求。
(4) 中国式现代化是\textbf{人与自然和谐共生}的现代化。
(5) 中国式现代化是\textbf{走和平发展道路}的现代化。

\subsection{中国式现代化的本质要求}

中国式现代化的本质要求是:坚持中国共产党领导,坚持中国特色社会主义,
(经济上)实现高质量发展,(政治上)发展全过程人民民主,
(文化上)丰富人民精神世界,(社会上)实现全体人民共同富裕,(生态文明上)促进人与自然和谐共生,
推动构建人类命运共同体,创造人类文明新形态。

中国式现代化是中国共产党领导的现代化,这是对中国式现代化的定性,是管总、管根本的。
党的领导决定中国式现代化的根本性质,
党的领导确保中国式现代化锚定目标奋斗行稳致远,
党的领导凝聚建设中国式现代化的磅礴力量。

\subsection{中国式现代化创造了人类文明新形态}


\begin{enumerate}[itemsep=0pt, label=(\arabic*)]
    \item 中国式现代化,深深植根于中华优秀传统文化,体现科学社会主义的先进本质,借鉴吸收一切人类优秀
    \item 优秀文明成果,代表人类文明进步的发展方向,是一种全新的人类文明形态。中国式现代化打破了 “现代化 = 西方化” 的迷思,展现了不同于西方现代化的新图景。
    \item 中国式现代化是对西方式现代化理论和实践的重大超越。
    \item 中国式现代化为广大发展中国家提供了全新选择。
\end{enumerate}

\section{推进中国式现代化行稳致远}

\subsection{推进中国式现代化需要牢牢把握的重大原则}

\begin{enumerate}[itemsep=0pt, label=(\arabic*)]
    \item 坚持和加强党的全面领导;
    \item 坚持中国特色社会主义道路;
    \item 坚持以人民为中心的发展思想;
    \item 坚持深化改革开放;
    \item 坚持发扬斗争精神。
\end{enumerate}

\subsection{推进中国式现代化需要正确处理的重大关系}

正确处理顶层设计与实践探索的关系、战略与策略的关系、守正与创新的关系、效率与公平的关系、
活力与秩序的关系、自立自强与对外开放的关系\footnote{独立自主是中华民族精神之魂,是我们立党立国的重要原则。对外开放是中国的基本国策,任何时候都不能动摇。}。

\subsection{推进中国式现代化必须坚持团结奋斗}

团结奋斗是中国共产党和中国人民最显著的精神标识,是中国人民创造历史伟业的必由之路。

\section{自测题}
\setcounter{example}{0}

\begin{example}
    党的二十大报告指出,未来五年是全面建设社会主义现代化国家开局起步的关键时期。
    前进道路上,必须牢牢把握五个重大原则。以下表述中不属于五个重大原则的是 \underline{\qquad \qquad \qquad}。
    \begin{tasks}[label={\Alph*. }](2)
        \task 坚持深化改革开放
        \task 坚持以人民为中心的发展思想
        \task 坚持全面依法治国
        \task 坚持发扬斗争精神
    \end{tasks}
    \begin{sol}
    推进中国式现代化需要牢牢把握的重大原则
    \begin{enumerate}[itemsep=0pt, label=(\arabic*)]
        \item 坚持和加强党的全面领导;
        \item 坚持中国特色社会主义道路;
        \item 坚持以人民为中心的发展思想;
        \item 坚持深化改革开放;
        \item 坚持发扬斗争精神。
    \end{enumerate}
    故选择 C。
    \end{sol}
\end{example}

\begin{example}
    以下不属于 “五个必由之路” 的是 \underline{\qquad \qquad \qquad}。
    \newline
    A. 坚持党的全面领导是坚持和发展中国特色社会主义的必由之路;
    \newline
    B. 贯彻科学发展观是新时代我国发展壮大的必由之路;
    \newline
    C. 中国特色社会主义是实现中华民族伟大复兴的必由之路;
    \newline
    D. 团结奋斗是中国人民创造历史伟业的必由之路
    \begin{sol}
        选 B。
    \end{sol}
\end{example}

\begin{example}
    \underline{\qquad \qquad \qquad} 同志首先用 “小康” 来诠释中国式现代化。
    \newline
    \begin{tabular}{p{0.5\textwidth} p{0.5\textwidth}}
    A. 毛泽东 & B. 邓小平; \\
    C. 周恩来; & D. 陈云
    \end{tabular}
\end{example}

\begin{example}
    我们坚持把 \underline{\qquad \qquad \qquad} 作为现代化建设的出发点和落脚点,
    着力维护和促进社会公平正义,着力促进全体人民共同富裕,坚决防止两极分化。
    \begin{enumerate}[itemsep=0pt, label=\Alph*.]
    \item 实现人民对美好生活的向往
    \item 提升人民生活水平
    \item 社会稳定安宁
    \item 建设人类精神家园
    \end{enumerate}
\end{example}

\begin{example}
    改革开放后,面对唱衰中国的舆论在国际上不绝于耳,各式各样针对中国的 “崩溃论”、“失败论”
    从来没有中断过,中国非但没有崩溃,反而综合国力与日俱增,创造了世所罕见的两大奇迹:
    一是经济快速发展奇迹,二是 \underline{\qquad \qquad \qquad} 奇迹。
    \begin{enumerate}[itemsep=0pt, label=\Alph*.]
    \item 物质财富丰富
    \item 平均寿命提高
    \item 人民幸福指数
    \item 社会长期稳定
    \end{enumerate}
\end{example}

\chapter{坚持党的全面领导}

坚持党的全面领导是坚持和发展中国特色社会主义的必由之路。中国特色社会主义最本质的特征是
中国共产党领导, 中国特色社会主义制度的最大优势是中国共产党领导,中国共产党是最高政治领
导力量,坚持党中央集中统一领导是最高政治原则。
\begin{remark}
    \textbf{中国共产党的性质}:中国共产党是中国工人阶级的先锋队,
    同时是中国人民和中华民族的先锋队,是\textbf{中国特色社会主义事业的领导核心},
    代表中国先进生产力的发展要求,代表中国先进文化的前进方向,代表中国最广大人民的根本利益。
\end{remark}
\begin{remark}
    中国共产党领导是中国特色社会主义最本质的特征,要置于科学社会主义基本原理中去认识,
    要置于国际共产朱云洞历史经验教训中去分析,要置于中国共产党领导的中国社会发展中去把握。
\end{remark}

\section{中国共产党领导是中国特色社会主义最本质的特征}

\subsection{中国最大的国情就是中国共产党的领导}

\begin{enumerate}[itemsep=0pt]
    \item 中国共产党的领导地位是\textbf{在历史奋斗中形成的}。
    \item 中国共产党领导是\textbf{人民当家做主的可靠保障}。
    \item 中国共产党领导\textbf{关系中国特色社会主义的性质、方向和命运}。
    中国特色社会主义之所以是社会主义,究其根本就在于坚持科学社会主义基本原则,
    在于坚持中国共产党的领导。
    \item 中国共产党领导是\textbf{实现中华民族伟大复兴的根本保证}。
\end{enumerate}
\begin{remark}
    分析论述题可能会考察 “如何认识中国最大的国情就是中国共产党的领导?”。
\end{remark}

\subsection{中国共产党领导是中国特色社会主义制度的最大优势}

\begin{enumerate}[itemsep=0pt]
    \item 中国共产党以马克思主义作为行动指南,在实践中不断推进马克思注意中国化时代化,
    为坚持和完善中国特色社会主义制度提供强大理论优势。
    \item 中国共产党的自身优势是中国特色社会主义制度优势的主要来源。
    \item 中国共产党能够集中全党全国力量,凝聚全民族共同意志,在各项事业中发挥总揽全局、
    协调各方的作用,确保中国特色社会主义制度的显著优势充分彰显。
\end{enumerate}

\subsection{加强党的全面领导为新时代党和国家事业发展提供了坚强保证}

\begin{enumerate}[itemsep=0pt]
    \item 坚持和加强党的全面领导,使党的领导核心作用充分彰显。
    \item 坚持和加强党的全面领导,使党的政治领导力、思想引领力、群众组织力、社会号召力显著增强。
    \item 坚持和加强党的全面领导,为推进新时代中国特色社会主义事业提供了政治保证,使党成为风雨来袭时中国人民最可靠的主心骨。
\end{enumerate}

\section{坚持党对一切工作的领导}

党的领导是党和国家的根本所在、命脉所在,是全国各族人民的利益所系、命运所系。

\subsection{中国共产党是最高政治领导力量}

\begin{enumerate}[itemsep=0pt]
    \item 党政军民学,东西南北中,\textbf{党是领导一切的},是最高的政治领导力量。
    \item 中国共产党作为最高政治领导力量,\textbf{得到了最广大人民群众的支持和拥护},
    得到了各民主党派、各团体、各阶层、各界人士的支持和拥护。在中国革命、建设、改革的历史进程中,
    中国共产党团结带领全国各族人民为争取民族独立、人民解放和实现国家富强、人民幸福而不懈奋斗,
    从根本上改变了中国人民的前途命运,开辟了实现中华民族伟大复兴的正确道路,深刻影响了世界
    历史进程,成为推动中国历史前进无可替代的领导核心力量。
    \begin{remark}
        近代以来中华民族面临的两大历史任务:争取民族独立、人民解放和实现国家富强、人民幸福。
    \end{remark}
    \item 中国共产党作为最高政治领导力量是由我国国家性质和政治制度体系决定的。
    我国是工人阶级领导的、以工农联盟为基础的人民民主专政的社会主义国家,工人阶级领导
    是通过其先锋队 —— 中国共产党领导实现的。中国共产党是我国最高政治领导力量,党的领导制度在
    我国国家政治制度体系中居于统领地位。
    \item 中国共产党作为最高政治领导力量是由中华民族伟大复兴事业决定的。
\end{enumerate}

\subsection{党的领导是全面的、系统的、整体的}

党的领导是全面的、系统的、整体的,并不是说党组织包揽包办一切、事无巨细什么都去管,而是在各级各种组织中发挥领导核心作用,
既善于总揽全局,又善于协调各方,不断增强党组织的领导统筹能力,充分调动方方面面的工作积极性,
是党的领导作用贯穿于工作全过程。

\subsection{维护党中央权威和集中统一领导}

\begin{enumerate}[itemsep=0pt]
    \item 党中央集中统一领导是党的领导的最高原则,是党保持团结统一和强大生命力、
    不断取得胜利的关键所在。加强和维护党中央权威和集中统一领导,是全党共同的政治责任,
    是党和国家事业发展的必然要求。
    \item 维护党中央权威和集中统一领导,是一个成熟的马克思主义执政党的重大建党原则。
    \item 维护当中言权威和集中统一领导,必须坚决贯彻党的理论、路线、方针政策和党中央决策部署。
    \item 维护党中央权威和集中统一领导,\textbf{最关键的是坚决维护习近平同志党中央的核心、全党的核心地位}。
    确立习近平同志党中央的核心、全党的核心地位,是党的十八大以来的重大政治成果和宝贵经验,
    是历史和人民的共同选择、郑重选择、必然选择。
    \item 维护党中央权威和集中统一领导,同坚持党的民主集中制是完全一致的。中国共产党是以民主集中制
    为根本组织原则和领导制度的马克思主义政党。
\end{enumerate}

\section{健全和完善党的领导制度体系}

\subsection{党的领导制度是我国的根本政治制度}


\begin{enumerate}[itemsep=0pt]
    \item 中国共产党领导是国家治理体系的核心,党的领导制度是中国特色社会主义制度建设的关键。
    党的领导制度是一个系统完备、内涵丰富的制度体系,主要涵盖了六个方面的制度。
    \begin{enumerate}[itemsep=0pt, label=(\arabic*)]
        \item 建立不忘初心、牢记使命的制度,形成长效机制,为坚持和完善党的领导制度体系奠定\textbf{坚实基础};
        \item 完善坚定维护党中央权威和集中统一领导的各项制度,坚决把维护习近平总书记党中央的核心、全党的核心地位落到实处,明确这一制度体系必须坚持的\textbf{最高原则};
        \item 健全党的全面领导制度,确保党在各种组织中发挥领导作用,是这一制度体系的\textbf{主体内容};
        \item 健全为人民执政、靠人民执政各项制度,巩固党执政的阶级基础,厚植党执政的群众基础,反映这一制度体系的\textbf{价值追求};
        \item 健全提高党的执政能力和领导水平制度,提高党把方向、谋大局、定政策、促改革的能力,体现这一制度体系的\textbf{实践要求};
        \item 完善全面从严治党制度,贯彻新时代党的建设总要求,为坚持和完善这一制度体系提供\textbf{坚强保证}。
    \end{enumerate}
    这六个方面的制度彼此支撑、相互联系,共同构筑了党的领导制度体系大厦,是坚持和加强党对一切工作领导的根本制度保障。
    \item 党的领导制度是\textbf{党的领导核心地位的必然反映和内在要求},明确了我国政治生活的领导关系、领导主体、
    领导对象,是中国特色社会主义制度体系的\textbf{核心},是国家治理体系和治理能力现代化的\textbf{关键},发挥着提纲挈领、无可替代的作用。
\end{enumerate}

\subsection{健全党中央对重大工作的领导机制}


\begin{enumerate}[itemsep=0pt]
    \item 加强党中央对重大工作的领导,是我们党的最优良传统和宝贵经验。
    \item 加强党中央对重大工作的领导,必须完善党中央重大决策部署落实机制。
\end{enumerate}


\subsection{健全党的全面领导制度}


\begin{enumerate}[itemsep=0pt]
    \item 必须完善党在各种组织中发挥领导作用的制度。
    \item 必须完善党协调各方的机制。
    \item 必须完善党领导各项事业的具体制度。
\end{enumerate}







% \chapter{坚持以人民为中心}



% \chapter{全面深化改革开放}
% \chapter{推动高质量发展}
% \chapter{社会主义现代化建设的教育、科技、人才战略}
% \chapter{发展全过程人民民主}

% 人民民主是社会主义的生命,全过程人民民主是社会主义民主政治的 \underline{本质属性}。

% \section{坚定中国特色社会主义政治制度自信}

% 制度优势是一个政党、一个国家的最大优势。坚定中国特色社会主义制度自信,
% 首先要坚定对中国特色社会主义政治制度的自信,增强走中国特色社会主义政治发展道路的信心和决心。

% \subsection{人民民主是社会主义的生命(人民民主的重要性)}

% \begin{enumerate}
%     \item 民主是全人类的共同价值,是人类政治文明发展的成果。
%     \item 民主是全人类的共同价值,是人类政治文明发展的成果
%     \item 人民民主建立在社会主义经济基础之上,体现了社会主义国家的性质,反映了社会主义制度的本质要求,是一种新型的社会主义民主。
%     \item 人民民主是全面建设社会主义现代化国家的 \underline{应有之义}。
% \end{enumerate}

% % \subsubsection{民主是全人类的共同价值,是人类政治文明发展的成果}

% % \begin{enumerate}[label=(\arabic*)]
% %     \item \textbf{民主是历史的},一个国家的民主制度是在这个国家的社会历史条件下形成的,植根于
% %     本国的历史文化传统,成长于本国人民的政治实践探索和智慧创造;
% %     \item \textbf{民主是具体的},一个国家实行什么样的民主制度,走什么样的民主发展道路,必须与这个国家的国情相适应;
% %     \item \textbf{实现民主的道路并非只有一条},社会性质不同、现实国情不同,则民主道路不同,民主形态各异;
% %     \item \textbf{民主从来就不是固定不变的,而是不断发展的};
% %     \item 民主没有最好,只有更好,人类对民主的探索和实践永无止境。
% % \end{enumerate}

% % \begin{key}
% %     评价民主与否的 “四个更要”:一个国家民主不民主,\textbf{关键在于是不是真正做到了人民当家做主};
% %     \begin{enumerate}[label=(\arabic*)]
% %         \item 要看人民有没有投票权,更要看人民有没有广泛参与权;
% %         \item 要看人民在选举过程中得到了什么口头许诺,更要看选举后这些承诺实现了多少;
% %         \item 要看制度和法律规定了什么样的政治程序和政治规则,更要看这些制度和法律是不是真正得到了执行;
% %         \item 要看权力运行规则和程序是否民主,更要看权力是否真正受到人民监督和制约。
% %     \end{enumerate}
% % \end{key}

% % \begin{key}
% %     评价世界各国民主的标尺
% %     \begin{enumerate}[label=(\arabic*)]
% %         \item 一个国家是不是民主,应该由这个国家的人民来评判,而不应该由外部少数人指手画脚来评判;
% %         \item 国际社会哪个国家是不是民主的,应该由国际社会共同来评判,而不应该由自以为是的少数国家来评判;
% %         \item 用单一的标尺衡量世界丰富多彩的政治制度,用单调的眼光审视人类五彩缤纷的政治文明,本身就是不民主的。
% %     \end{enumerate}
% % \end{key}

% % \subsubsection{民主是全人类的共同价值,是人类政治文明发展的成果}

% % \subsubsection{人民民主建立在社会主义经济基础之上,体现了社会主义国家的性质,反映了社会主义制度的本质要求,是一种新型的社会主义民主}

% % \begin{key}
% %     如何理解 “国家一切权力属于人民” ?
% %     \begin{enumerate}[label=(\arabic*)]
% %         \item 体现在国家根本性质即国体上,就是工人阶级领导的,以工农联盟为基础的人民民主专政的社会主义国家;
% %         \item 体现在国家政权的组织形式即政体上,就是人民通过各级人民代表大会行使人民权利。
% %     \end{enumerate}
% % \end{key}

% % \subsubsection{人民民主是全面建设社会主义现代化国家的应有之义}


% \subsection{中国特色社会主义政治制度行得通、有生命力、有效率}

% 中国特色社会主义政治制度是由根本政治制度、基本政治制度、重要政治制度等组成的制度体系,
% 具有鲜明的中国特色,在实践中显示出巨大优势。

% \subsection{坚定不移走中国特色社会主义政治发展道路}

% \begin{enumerate}
%     \item \textbf{三方面有机统一}:走中国特色社会主义政治发展道路,必须坚持党的领导、
%     人民当家做主、依法治国有机统一。
%     \begin{enumerate}
%         \item 党的领导是人民当家做主和依法治国的根本保证;
%         \item 人民当家作之时社会主义民主政治的本质特征;
%         \item 依法治国是党领导人民治理国家的基本方式,
%     \end{enumerate}
%     三者统一于我国社会主义民主政治伟大实践。
%     \item 走中国特色社会主义政治发展道路,最根本的是 \underline{坚持党的领导}。
%     \item 走中国特色社会主义政治发展道路,必须始终保持政治定力。
% \end{enumerate}

% \section{全过程人民民主是社会主义民主政治的本质属性}

% \subsection{全过程人民民主是社会主义民主政治的伟大创造}

% 全过程人民民主的提出、实践和发展,鲜明展示了我们党始终高举人民民主旗帜、实现人民当家做主
% 的坚定立场,深刻彰显了我国人民民主的鲜明特色和显著优势,实现了过程民主和成果民主、
% 程序民主和实质民主、直接民主和间接民主、人民民主和国家意志相统一。

% \subsection{全过程人民民主是全链条、全方位、全覆盖的民主}

% 全过程人民民主,既有完整的制度程序,也有完整的参与实践,
% \begin{enumerate}
%     \item 全链条:
%     把民主选举、民主协商、民主决策、民主管理、民主监督贯通起来;
%     \item 全覆盖:涵盖经济、政治、文化、社会、生态文明等各个方面;
%     \begin{remark}
%         回忆 “五位一体” 和 “四个全面” :中国特色社会主义事业总体布局是经济、
%         政治、文化、社会、生态文明建设五位一体,战略布局是全面建设社会
%         主义现代化国家、深化改革、依法治国、从严治党四个全面。
%     \end{remark}
%     \item 全方位:聚焦国家发展大事、社会治理难事、百姓日常琐事;
% \end{enumerate}
% 具有时间上的连续性、内容上的整体性、运行上的协同性、人民参与上的广泛性和持续性。

% \subsection{全过程人民民主是最广泛、最真实、最管用的民主}

% 评判一种民主形式好不好,实践最有说服力,人民最有发言权。
% \begin{enumerate}
%     \item 最广泛:全过程人民民主能确保人民充分享有各方面权力;
%     \item 最管用:能有效促进国家治理高效和社会和谐稳定;
%     \item 最真实:能维护和保障人民群众的根本利益。
% \end{enumerate}

% \section{健全人民当家做主的制度体系}

% \subsection{根本政治制度 —— 人民代表大会制度}

% 人民代表大会制度是我国的根本政治制度,是符合我国国情和实际、体现社会主义国家性质、
% 保证人民当家做主、保障实现中华民族伟大复兴的好制度,是我们党领导人民在人类政治制度
% 史上的伟大创造。人民代表大会制度是实现全过程人民民主的重要制度载体。
% \begin{enumerate}
%     \item 坚持和完善人民代表大会制度,要在党的领导下,不断扩大人民有序政治参与,保证人民依法享有广泛权力和自由;
%     \item 支持和保证人民通过人民代表大会行使国家权力,保证各级人大都由民主选举产生、
%     对人民负责、受人民监督;
%     \item 支持和保证人大及其常委会依法行使立法权、监督权、决定权、任免权,
%     健全人大对行政机关、监察机关、审判机关、检察机关监督制度,维护国家统治统一、
%     尊严、权威;
%     \item 加强人大代表工作能力建设,密切人大代表同人民群众的联系;
%     \item 完善人大的民主民意表达平台和载体。
% \end{enumerate}

% \subsection{基本政治制度}

% 中国共产党领导的多党合作和政治协商制度、民族区域自治制度、基层群众自治制度构成了我国的基本
% 政治制度,反映了我国社会主义民主政治的独特优势,是保障各政党、各阶层、各民族和基层人民群众
% 当家作主的重要基础。

% \subsubsection{中国共产党领导的多党合作和政治协商制度}
% \subsubsection{民族区域自治制度}
% \subsubsection{基层群众自治制度}

% \subsection{重要政治制度}
% \subsection{全面发展协商民主}

% \section{巩固和发展新时代爱国统一战线}

% \chapter{全面依法治国}

% \section{坚持中国特色社会主义法治道路}

% \subsection{全面依法治国是国家治理的一场深刻革命}

% \begin{enumerate}
%     \item 法律是治国之重器, 法治是国家治理体系和治理能力的 \underline{重要依托};
%     \item 全面依法治国, 是完善和发展中国特色社会主义制度、
%     推进国家治理体系和治理能力现代化的重要方面,
%     是坚持和发展中国特色社会主义的 \underline{本质要求和重要保障}.
% \end{enumerate}

% \subsection{全面依法治国的唯一正确道路}

% \begin{enumerate}
%     \item 我们党领导人民长期探索走出的中国特色社会主义法治道路,是全面依法治国的唯一正确道路。
%     这条道路是由我国社会主义国家性质所决定的,本质上是中国特色社会主义道路在法治领域的具体
%     体现。
%     \item \textbf{核心要义:两个坚持、一个贯彻}:中国特色社会主义法治道路的核心要义,
%     是要坚持党的领导、坚持中国特色社会主义制度,贯彻中国特色社会主义法治理论。
%     \begin{enumerate}[label=(\arabic*)]
%         \item 党的领导是中国特色社会主义最本质的特征,是社会主义法制最根本的保证;
%         \item 中国特色社会主义制度是中国特色社会主义法治体系的制度基础,
%         是全面推进依法治国的根本保障;
%         \item 中国特色社会主义法治理论是中国特色社会主义法治体系的理论指导、学理支撑,
%         是全面推进依法治国的行为指南;
%     \end{enumerate}
% \end{enumerate}

% \subsection{把握中国特色社会主义法治道路必须毫不动摇坚持的原则}

% \begin{enumerate}
%     \item 坚持\textbf{中国共产党的领导}:党的领导是中国特色社会主义法治之魂,
%     是我们的法治同西方资本主义国家的法治最大的区别;
%     \item 坚持\textbf{以人民为中心}:推进全面依法治国,根本目的是依法保障人民权益;
%     \item 坚持\textbf{法律面前人人平等}:平等是社会主义法律的基本属性;
%     \item 坚持\textbf{依法治国和以德治国相结合}:
%     \begin{enumerate}[label=(\arabic*)]
%         \item 既重视发挥法律的规范作用,又重视发挥道德的教化作用;
%         \item 以法治体现道德理念、强化法律对道德建设的促进作用;
%         \item 以道德滋养法治精神、强化道德对法治文化的支撑作用。
%     \end{enumerate}
%     \item 坚持\textbf{从中国实际出发}。
% \end{enumerate}

% \chapter{建设社会主义文化强国}
% \chapter{以保障和改善民生为重点加强社会建设}
% \chapter{建设社会主义生态文明}
% \chapter{维护和塑造国家安全}



% \chapter{建设巩固国防和强大人民军队}

% 巩固国防和强大人民军队是新时代坚持和发展中国特色社会主义、实现中华民族伟大复兴的\textbf{战略支撑}。
% \textbf{国防和军队建设是捍卫国家主权、安全、发展利益的坚强后盾}。

% \section{强国必须强军,军强才能国安}

% \subsection{新时代人民军队的历史任务}

% 人民军队是执行党的政治任务的武装集团,党和人民所需就是军队使命任务所系。新时代人民军队的历史任务
% 是为\textbf{巩固中国共产党领导和我国社会主义制度},\textbf{捍卫国家主权、统一和领土完整},
% \textbf{维护我国海外利益},\textbf{促进世界和平与发展}提供战略支撑。

% \section{实现党在新时代的强军目标}

% 党的十九大明确提出,党在新时代的强军目标是建设一支听党指挥、能打胜仗、作风优良的人民军队,
% 把人民军队建设成为世界一流军队。

% \subsection{强军目标的科学内涵}

% \begin{enumerate}[label=(\textup{\arabic*})]
%     \item \textbf{听党指挥是灵魂,决定军队建设的政治方向}。
%     坚持党对人民军队的绝对领导是人民军队永远不变的军魂。
%     实践证明,听党指挥是人民军队建设的首要,是人民军队的命脉所在。
%     \item \textbf{能打胜仗是核心},反映军队的根本职能和军队建设的根本指向。
%     \textbf{人民军队必须牢固树立战斗力这个唯一的根本的标准}。
%     \item \textbf{作风优良是保障},关系军队的性质、宗旨、本色。
% \end{enumerate}

% \subsection{全面推进国防和军队现代化的战略安排}

% 力争到 2035 年基本实现国防和军队现代化,到本世纪中叶把人民军队全面建成世界一流军队。

% \section{加快推进国防和军队现代化}

% 如期实现建军一百年奋斗目标,加快吧人民军队建成世界一流军队,是全面建设社会主义现代化
% 国家的战略要求。

% \subsection{坚持党对人民军队的绝对领导}

% 坚持党对人民军队的绝对领导,是马克思主义建党建军的一条基本原则,
% 是建军之本、强军之魂。坚持党对人民军队的绝对领导必须有一整套制度作保证。
% 军委主席负责制是坚持党对人民军队绝对领导的根本制度和根本实现形式,
% 在党领导军队的一整套制度体系中处于最高层次、居于统领地位。

% \subsection{坚持政治建军、改革强军、科技强军、人才强军、依法治军}

% \begin{enumerate}[label=(\textup{\arabic*})]
%     \item 政治建军是人民军队的立军之本。
%     \item 改革是决定人民军队发展壮大、制胜未来的关键一招。
%     \item 科技是核心战斗力,是军事发展中最活跃、最具改革性的因素。
%     \item 强军之道,要在得人。
%     \item 依法治军是我们党建军治军基本方式。
% \end{enumerate}

% \subsection{巩固提高一体化国家战略体系和能力}

% 军政军民团结是实现富国和强军相统一的重要政治保障。

% % \chapter{坚持“一国两制”和推进祖国完全统一}


% \chapter{中国特色大国外交和推动构建人类命运共同体}


% % % 传统国际关系的核心支撑:“霸权稳定论”。
% % % \begin{enumerate}[label=\textup{\arabic*}${}^\circ$]
% % %      \item 霸权:一个国家单独主宰国际政治和经济关系的规则和安排的能力;
% % %      \item 稳定:不发生大的体系性战争,国际体系保持相对稳定;
% % % \end{enumerate}

% % 习近平外交思想的主要内容集中体现为“十个坚持”:\kaishu
% % 坚持以维护党中央权威为统领加强党对对外工作的集中统一领导;
% % 坚持以实现中华民族伟大复兴为使命推进中国特色大国外交;
% % 坚持以维护世界和平、促进共同发展为宗旨推动构建人类命运共同体;
% % 坚持以中国特色社会主义为根本增强战略自信;
% % 坚持以共商共建共享为原则推动“一带一路”建设;
% % 坚持以相互尊重、合作共赢为基础走和平发展道路;
% % 坚持以深化外交布局为依托打造全球伙伴关系;
% % 坚持以公平正义为理念引领全球治理体系改革;
% % 坚持以国家核心利益为底线维护国家主权、安全、发展利益;
% % 坚持以对外工作优良传统和时代特征相结合为方向塑造中国外交独特风范。\songti

% \section{新时代中国外交在大变局中开创新局}

% \section{自测题}
% \setcounter{example}{0}

% \begin{example}
%     中国特色大国外交要 \underline{\qquad \qquad \qquad},推动建设新型国际关系,推动构建人类命运共同体。
%     \begin{enumerate}[label=\Alph*.]
%     \item 服务民族复兴
%     \item 服务改革开放
%     \item 促进人类进步
%     \item 促进世界大同
%     \end{enumerate}
% \end{example}

% % % \subsection{当前世界正经历百年未有之大变局}
% % % \subsection{中国必须有自己特色的大国外交}
% % % \subsection{我国国际影响力、感召力、塑造力显著提升}
% % % \section{全面推进中国特色大国外交}
% % % \subsection{坚持走和平发展道路}
% % % \subsection{推动构建新型国际关系}
% % % \subsection{坚决维护国家主权、安全、发展利益}
% % % \subsection{坚持外交为民}
% % % \section{推动构建人类命运共同体}
% % % \subsection{推动构建人类命运共同体时世界各国人民前途所在}
% % % \subsection{推动构建人类命运共同体的价值基础和重要依托}
% % % \subsection{积极参与全球治理体系改革和建设}
% % % \subsection{高质量共建“一带一路”}

% % \chapter{全面从严治党}

\input{_appendix.tex}

% \part{附录}

% \chapter{习近平经济思想}



% \chapter{中国近现代史纲要课程期末复习指南和速成复习重点}

% 如果你时间紧张的话,可以结合下表来进行针对性的复习,并且在复习的过程中\textbf{只抓加粗的内容},
% 在保证对加粗内容的熟记和深刻理解后,再结合你的个人兴趣选择性地复习一些下表中未列出且未加粗的知识点。

% \chapter{思想道德与法治课程期末复习速成资料}
% \chapter{马克思主义基本原理课程期末复习速成资料}
% \chapter{毛泽东思想与中国特色社会主义理论体系概论课程期末复习速成资料}
% \chapter{习近平新时代中国特色社会主义思想概论课程期末复习速成资料}


% \begin{enumerate}
%     \item 构建高水平社会主义市场经济体制,关键是 \underline{要处理好政府和市场的关系}。
%     \item 2023年10月12日,中共中央总书记、国家主席、中央军委主席习近平主持召开进一步推动
%     长江经济带高质量发展座谈会并发表重要讲话。他强调,长江经济带发展成就有目共睹,发展质量稳
%     步提升,发展态势日趋向好。同时也要看到,长江流域生态环境保护和高质量发展正处于由量变到质
%     变的关键时期,取得的成效还不稳固。从长远来看,推动长江经济带高质量发展,
%     根本上依赖于 \underline{长江流域高质量的生态环境}。
%     \item 2023年7月4日,国家主席习近平在北京以视频方式出席上海合作组织成员国元首理事会第
%     二十三次会议并发表重要讲话。他指出,我们维护国际公平正义,反对霸权霸道霸凌行径,扩大本
%     组织“朋友圈”,构建 \underline{对话不对抗、结伴不结盟的伙伴关系}。
%     \item \underline{农业强国}是社会主义现代化强国的根基。
%     \item \underline{高质量发展}是新时代我国经济社会发展的鲜明主题。
%     \item 构建 \underline{以国内大循环为主体、国内国际双循环相互促进的新发展格局},
%     是根据我国发展阶段、发展环境、条件变化,特别是基于我国比较优势变化,审时度势作出的重大决策,
%     是立足实现第二个百年奋斗目标、统筹发展和安全作出的战略决策,是把握未来发展主动权的战略部署,
%     是推动高质量发展的战略基点,是关系我国发展全局的重大战略任务。
%     \item 建设数字中国是 \underline{数字时代推进
%     中国式现代化的重要引擎},是 \underline{构筑国家竞争新优势的有力支撑}。加快数字
%     中国建设,对全面建设社会主义现代化国家、全面推进中华民族伟大复兴具有重要意义和深远影响。
%     \item \underline{改革开放} 是当代中国最显著的特征、最壮丽的气象。
%     \item \underline{中国共产党的领导} 是中国特色社会主义的最大优势。
%     \item 高水平科技自强的重大意义:
%     \begin{enumerate}
%         \item 实现高水平科技自立自强是国家强盛和民族复兴的战略基石;
%         \item 实现高水平科技自立自强是应对风险挑战和维护国家利益的必然选择;
%         \item 实现高水平科技自立自强是构建新发展格局、推动高质量发展、满足人民美好生活需要
%         的内在要求。
%     \end{enumerate}
%     \item \underline{人民健康} 是社会文明进步的基础。
%     \item 我国实施优先就业战略的主要措施:
%     \begin{enumerate}
%         \item 强化就业优先政策;
%         \item 健全就业公共服务体系;
%         \item 完善重点群体就业支持体系;
%         \item 统筹城乡就业政策体系;
%         \item 健全终身职业技能培训制度;
%         \item 健全劳动法律法规;
%         \item 完善劳动关系协调协商机制。
%     \end{enumerate}
%     \item 习近平法治思想解释了社会主义法治的生命力和优越性,必将增强广大干部群众走
%     中国特色社会主义法治道路的信心;开辟了马克思主义法治理论新境界,必将引领中国特色
%     社会主义法治理论创新发展;擘画了新时代全面依法治国的宏伟蓝图,必将引领法治中国
%     建设迈向良法善治新境界;凝聚了法治建设的中国经验和中国智慧,必将有力提升中国
%     法治的国际话语权和影响力。
%     \item “第二个结合” 是我们党对马克思主义中国化时代化历史经验的深刻总结,是对
%     中华文明发展规律的深刻把握,表明我们党对中国道路、理论、制度的认识达到了新高度,
%     表明我们党的历史自信、文化自信达到了新高度,表明我们党在传承中华优秀传统文化中
%     推进理论创新的自觉性达到了新高度。
%     \item 马克思主义是不断发展的理论,\underline{本土化和时代化} 是马克思主义创新发
%     展的基本路径,只有将二者相结合,才能全面把握马克思主义创新发展的规律,才能全面把握
%     马克思主义中国化时代化的理论根据。
%     \item \underline{调查研究} 是获得真知灼见的源头活水,是贯彻群众路线的有效途径。
%     \item 作风问题核心是党同人民群众的关系问题。
%     \item 发展基层民主,是实现人民有效政治参与的重要渠道,是人民当家做主的有效途径,
%     是社会主义民主最广泛的实践。
%     \item 基础研究是科技创新的源头。我国面临的很多 “卡脖子” 技术问题,
%     根子是基础理论研究跟不上。
%     \item 建设社会主义文化强国、推动社会主义文化繁荣兴盛,关键在于坚定中国特色社会主义文化自信。
%     \item 中国式现代化,深深植根于中华优秀传统文化,体现科学社会主义的先进本质,借鉴吸收一切人类优秀优秀文明成果,代表人类文明进步的发展方向,是一种全新的人类文明形态。
%     \begin{enumerate}
%         \item 中国式现代化打破了 “现代化 = 西方化” 的迷思,展现了不同于西方现代化的新图景。
%         \item 中国式现代化是对西方式现代化理论和实践的重大超越。
%         \item 中国式现代化为广大发展中国家提供了全新选择。
%     \end{enumerate}
% \end{enumerate}


% \newpage
% \chapter{习近平新时代中国特色社会主义思想概论课程核心要点}
% \setcounter{page}{1}

% \setcounter{section}{-1}
% \section{导论}

% \begin{key}
%     习近平新时代中国特色社会主义思想创立的时代背景
%     \begin{enumerate}
%         \item \underline{世界百年未有之大变局} 加速演进(世界);
%         \item \underline{中华民族伟大复兴} 进入关键时期(民族);
%         \item \underline{中国式现代化} 全面推进拓展(国家);
%         \item \underline{科学社会主义} 在 21 世纪的 \underline{中国} 焕发新的蓬勃生机(理论);
%         \item \underline{中国共产党} 自我革命开辟新的境界(党)。
%     \end{enumerate}
% \end{key}

% \begin{key}
%     \textbf{习近平新时代中国特色社会主义思想是 “两个结合” 的重大成果}
%     \begin{enumerate}
%         \item \textbf{“两个结合” 的重要性}:“两个结合” 是我们\underline{党在探索中国特色社会主义道路中得出的规律性认识},
%         是我们取得成功的最大法宝。
%         \item \textbf{“两个结合” 的意义}:
%         \begin{enumerate}[label=(\arabic*)]
%             \item 让中国特色社会主义道路有了更加宏阔深远的历史纵深,
%             \item 拓展了中国特色社会主义道路的文化根基,
%             \item 开辟了广阔的理论和实践创新空间,
%             \item 表明我们党对中国道路、理论、制度的认识达到了新高度,
%             \item 表明我们党的历史自信、文化自信达到了新高度,
%             \item 表明我们党在传承中华优秀传统文化中推进理论创新的自觉性达到了新高度。
%         \end{enumerate}
%     \end{enumerate}
% \end{key}

% \begin{key}
%     习近平新时代中国特色社会主义思想是完整的科学体系
%     \begin{enumerate}
%         \item \textbf{“十个明确” 是 \underline{主体内容}},
%         集中体现了这一思想体系的主要观点和基本精神,
%         构成了这一思想体系的四梁八柱,发挥着统摄作用。
%         \begin{enumerate}[label=(\arabic*)]
%             \item 明确中国特色社会主义最本质的特征是中国共产党领导,中国特色社会主义制度的最大优势是中国共产党领导,中国共产党是最高政治领导力量,全党必须增强“四个意识”、坚定“四个自信”、做到“两个维护”;
%             \item 明确坚持和发展中国特色社会主义,总任务是实现社会主义现代化和中华民族伟大复兴,在全面建成小康社会的基础上,分两步走在本世纪中叶建成富强民主文明和谐美丽的社会主义现代化强国,以中国式现代化推进中华民族伟大复兴;
%             \item 明确新时代我国社会主要矛盾是人民日益增长的美好生活需要和不平衡不充分的发展之间的矛盾,必须坚持以人民为中心的发展思想,发展全过程人民民主,推动人的全面发展、全体人民共同富裕取得更为明显的实质性进展;
%             \item 明确中国特色社会主义事业总体布局是经济建设、政治建设、文化建设、社会建设、生态文明建设五位一体,战略布局是全面建设社会主义现代化国家、全面深化改革、全面依法治国、全面从严治党四个全面;
%             \item 明确全面深化改革总目标是完善和发展中国特色社会主义制度、推进国家治理体系和治理能力现代化;
%             \item 明确全面推进依法治国总目标是建设中国特色社会主义法治体系、建设社会主义法治国家;
%             \item 明确必须坚持和完善社会主义基本经济制度,使市场在资源配置中起决定性作用,更好发挥政府作用,把握新发展阶段,贯彻创新、协调、绿色、开放、共享的新发展理念,加快构建以国内大循环为主体、国内国际双循环相互促进的新发展格局,推动高质量发展,统筹发展和安全;
%             \item 明确党在新时代的强军目标是建设一支听党指挥、能打胜仗、作风优良的人民军队,把人民军队建设成为世界一流军队;
%             \item 明确中国特色大国外交要服务民族复兴、促进人类进步,推动建设新型国际关系,推动构建人类命运共同体;
%             \item 明确全面从严治党的战略方针,提出新时代党的建设总要求,全面推进党的政治建设、思想建设、组织建设、作风建设、纪律建设,把制度建设贯穿其中,深入推进反腐败斗争,落实管党治党政治责任,以伟大自我革命引领伟大社会革命。
%         \end{enumerate}
%         \item \textbf{“十四个坚持” 是 \underline{重要组成部分}},
%         是在治国理政各方面作出的理论分析和政策指导,构成了新时代坚持和发展中国特色社会主义的基本方略。
%         \begin{enumerate}[label=(\arabic*)]
%             \item 坚持\textbf{党对一切工作的领导};
%             \item 坚持\textbf{以人民为中心};
%             \item 坚持\textbf{全面深化改革};
%             \item 坚持\textbf{新发展理念};
%             \item 坚持\textbf{人民当家作主};
%             \item 坚持\textbf{全面依法治国};
%             \item 坚持\textbf{社会主义核心价值体系};
%             \item 坚持\textbf{在发展中保障和改善民生};
%             \item 坚持\textbf{人与自然和谐共生};
%             \item 坚持\textbf{总体国家安全观};
%             \item 坚持\textbf{党对人民军队的绝对领导};
%             \item 坚持 \textbf{“一国两制”和推进祖国统一};
%             \item 坚持\textbf{推动构建人类命运共同体};
%             \item 坚持\textbf{全面从严治党}。
%         \end{enumerate}
%         \item \textbf{“十三个方面成就”} 就是在坚持党的全面领导、全面从严治党、经济建设、全面深化改革开放、
%         政治建设、全面依法治国、文化建设、社会建设、生态文明建设、国防和军队建设、维护国家安全、
%         坚持“一国两制”和推进祖国统一、外交工作等方面取得的历史性成就和发生的历史性变革。
%         “十三个方面成就” \underline{对新时代伟大实践进行了科学总结},
%         全景式地展示了习近平新时代中国特色社会主义思想的理论与实践成果。
%         \item “六个必须坚持” \underline{赋予了马克思主义世界观和方法论以新的时代内涵},
%         彰显了习近平新时代中国特色社会主义思想的理论品格和鲜明特征。
%         \begin{enumerate}[label=(\arabic*)]
%             \item 必须坚持\textbf{人民至上}:体现了唯物主义群众史观;
%             \item 必须坚持\textbf{自信自立}:体现了客观规律性和主观能动性的有机结合;
%             \item 必须坚持\textbf{守正创新}:体现了变与不变、继承与发展的内在联系;
%             \item 必须坚持\textbf{问题导向}:体现了矛盾的普遍性和客观性;
%             \item 必须坚持\textbf{系统观念}:体现了辩证唯物主义普遍联系的原理;
%             \item 必须坚持\textbf{胸怀天下}:体现了马克思主义追求人类进步和解放的崇高理想。
%         \end{enumerate}
%     \end{enumerate}
% \end{key}

% \begin{key}
%     \textbf{习近平新时代中国特色社会主义思想的历史地位}
%     \begin{enumerate}
%         \item 习近平新时代中国特色社会主义思想继承和发展马克思列宁主义、毛泽东思想、邓小平理论、
%         “三个代表” 重要思想、科学发展观,是马克思主义在当代中国发展的最新理论成果,开辟了
%         马克思主义中国化时代化新境界。
%         \item 习近平新时代中国特色社会主义思想是当代中国马克思主义、二十一世纪马克思主义,
%         是中华文化和中国精神的时代精华,\textbf{实现了马克思主义中国化时代化新的飞跃}。
        
%         \item 习近平新时代中国特色社会主义思想,把马克思主义基本原理同中国具体实际相结合、同中华优秀传统文化相结合,
%         使马克思主义这个魂脉和中华优秀传统文化这个根脉内在贯通、相互成就,是中华民族的文化主体性最有力的体现,
%         是中华文化和中国精神的时代精华。
        
%         \item 习近平新时代中国特色社会主义思想,是全党全国各族人民为实现中华民族伟大复兴而奋斗的行动指南,是新时代
%         党和国家事业发展的根本遵循。
%     \end{enumerate}
% \end{key}

% \begin{key}
%     \textbf{深刻领悟 “两个确立” 的决定性意义}:“两个确立” 是党在新时代取得的重大政治成果、对新时代党和国家事业发展、
%     对推进中华民族伟大复兴历史进程具有决定性意义。
% \end{key}

% \section{新时代坚持和发展中国特色社会主义}

% \subsection{方向决定道路,道路决定命运}

% \begin{key}
%     中国特色社会主义是历史和人民的选择
%     \begin{enumerate}
%         \item 一个国家实行什么样的主义,关键要看这个主义能否解决这个国家面临的历史性课题。
%         \item 中国特色社会主义具有深厚的历史渊源和广泛的现实基础,是实现中华民族伟大复兴的正确道路。
%     \end{enumerate}
% \end{key}

% \begin{key}
%     中国特色社会主义是社会主义而不是其他什么主义
%     \begin{enumerate}
%         \item 中国特色社会主义是 \underline{科学社会主义理论逻辑和中国社会发展历史逻辑的辩证统一}。
%         中国特色社会主义之所以是社会主义,而不是其他什么主义,就是因为我们始终坚持科学社会主义基本原则,
%         并根据新的时代条件赋予其鲜明的中国特色。
%         \item 中国特色社会主义 \underline{写出了科学社会主义的 “新版本”}。当代中国的伟大社会变革,不是简单延续我国历史文化的母版,
%         不是简单套用马克思主义经典作家设想的模板,不是其他国家社会主义实践的再版,也不是国外现代化发展的翻版。
%     \end{enumerate}
% \end{key}

% \begin{key}
%     \textbf{坚定道路自信、理论自信、制度自信、文化自信}:新时代坚持和发展中国特色社会主义,
%     必须 \underline{坚定道路自信、理论自信、制度自信、文化自信}。
%     改革开放以来我们取得一切成绩和进步的根本原因,归结起来就是:开启了 \underline{中国特色社会主义道路},
%     形成了 \underline{中国特色社会主义理论体系},确立了 \underline{中国特色社会主义制度},发展了 \underline{中国特色社会主义文化}。
%     其中,道路是\textbf{实现途径},理论体系是\textbf{行动指南},制度是\textbf{根本保障},
%     文化是\textbf{精神力量}。
% \end{key}

% \subsection{}

% \chapter{习近平经济思想}

% \begin{quote}
%     \kaishu
%     资产阶级,由于一切生产工具的迅速改进,由于交通的极其便利,把一切民族甚至最野蛮的民族
%     都卷到文明中来了。它的商品的低廉价格,是它用来摧毁一切万里长城、征服野蛮人最顽强的仇
%     外心理的重炮。它迫使一切民族——如果它们不想灭亡的话——采用资产阶级的生产方式;它迫使它
%     们在自己那里推行所谓文明,即变成资产者。一句话,它按照自己的面貌为自己创造出一个世界。
%     \songti
%     \begin{flushright}
%         ——卡尔·马克思《共产党宣言》
%     \end{flushright}
% \end{quote}

% 由中共中央宣传部、国家发展和改革委员会组织编写的《习近平经济思想学习纲要》,
% 将习近平经济思想基本内容梳理归纳为十三个方面:
% \begin{enumerate}[label=(\arabic*)]
%     \item 加强党对经济工作的全面领导是我国经济发展的根本保证;
%     \item 坚持以人民为中心的发展思想是我国经济发展的根本立场;
%     \item 进入新发展阶段是我国经济发展的历史方位;
%     \item 坚持新发展理念是我国经济发展的指导原则;
%     \item 构建新发展格局是我国经济发展的路径选择;
%     \item 推动高质量发展是我国经济发展的鲜明主题;
%     \item 坚持和完善社会主义基本经济制度是我国经济发展的制度基础;
%     \item 坚持问题导向部署实施国家重大发展战略是我国经济发展的战略举措;
%     \item 坚持创新驱动发展是我国经济发展的第一动力;
%     \item 大力发展制造业和实体经济是我国经济发展的主要着力点;
%     \item 坚定不移全面扩大开放是我国经济发展的重要法宝;
%     \item 统筹发展和安全是我国经济发展的重要保障;
%     \item 坚持正确工作策略和方法是做好经济工作的方法论。
% \end{enumerate}

% \section{经济新常态}

% \subsubsection{经济新常态}

% 认识新常态、适应新常态、引领新常态,是当前和今后一个时期我国\textbf{经济发展}的\textbf{大逻辑}。
% \begin{remark}
%     单选、多选、判断、论述
% \end{remark}

% 改革开放以后,党扭住经济建设这个中心,领导人民埋头苦干,创造出经济快速发展奇迹,国家经济
% 实力大幅跃升。同时,由于一些地方和部门存在片面追求速度规模、发展方式粗放等问题,加上国际
% 金融危机后世界经济持续低迷影响,经济结构性体制性矛盾不断积累,发展不平衡、不协调、不可持
% 续问题十分突出。党中央提出,我国经济发展进入新常态,已由高速增长阶段转向高质量发展阶段,
% 面临\textbf{增长速度换挡期、结构调整阵痛期、前期刺激政策消化期“三期叠加”}的复杂局面,
% 传统发展模式难以为继。

% \begin{remark}
%     这段内容可以作为习近平经济思想产生的背景来论述。
% \end{remark}
% \begin{remark}
%     “三期叠加” 是考点。
% \end{remark}

% 经济新常态下我国经济\textbf{从高速增长转向中高速增长},发展动力\textbf{从主要依靠资源
% 和低成本劳动力等要素投入驱动转向创新驱动}。
% \begin{remark}
%     从主要依靠资源和低成本劳动力等要素投入的驱动也称为要素驱动,是经济新常态以前的主要经济发展动力。
% \end{remark}

% % \begin{thm}
% %     中国特色社会主义条件下,是否会爆发经济危机?
% %     \begin{sol}
% %         无论你认为会或者不会,答案都是会(存在这种风险与可能性)或者不会(我们能够有效掌握,
% %         前提是坚持党对经济工作的集中统一领导)。

% %         \kaishu 经济危机的前兆表现:
% %         \begin{enumerate}[label=(\textup{\arabic*})]
% %             \item 豆腐渣工程;
% %             \item 假冒伪劣产品: 典型代表 为10 年前仿冒 iPhone 的黑果白果手机等。
% %         \end{enumerate}\songti
% %     \end{sol}
% % \end{thm}

% \subsubsection{四个没有变}

% 新常态下,我国经济发展态势的四个没有变包括
% \begin{enumerate}[label=$\left(\arabic*\right)$]
%     \item 经济发展长期向好的基本面没有变;
%     \item 经济韧性好、潜力足、回旋空间大的基本特质没有变;
%     \item 经济持续增长的良好支撑基础和条件没有变;
%     \item 经济结构调整优化的前进态势没有变。
% \end{enumerate}

% \subsubsection{经济新机遇}

% 经济方面的五个新机遇包括
% \begin{enumerate}[label=$\left(\arabic*\right)$]
%     \item 加快经济结构优化升级带来新机遇;
%     \item 提升科技创新能力带来新机遇;
%     \item 深化改革开放带来新机遇;
%     \item 加快绿色发展带来新机遇;
%     \item 参与全球治理体系变革带来新机遇。
% \end{enumerate}

% \subsubsection{供给侧结构性改革}

% 2015 年,党中央进一步提出实施供给侧结构性改革,明确\textbf{去产能、去库存、去杠杆、降成本、补短板}
% 的 “三去一降一补” 五大重点任务,通过大力推动 “破、立、降”,使供需结构失衡得到矫正,通货紧缩
% 趋向得到遏制。
% \begin{remark}
%     多选题考查
% \end{remark}

% 构建新发展格局的关键是必须坚持深化供给侧结构性改革这条主线,继续完成 “三去一降一补” 的重要任务,
% 实现经济在高水平上的动态平衡。

% 2018 年,党中央进一步提出深化供给侧结构性改革的 “巩固、增强、提升、畅通” 八字方针,具体是指
% \begin{enumerate}[label=$(\arabic*)$]
%     \item 巩固 “三去一降一补” 成果,推动更多产能过剩行业加快出清,降低全社会各类营商成本,加大基础设施等领域补短板力度;
%     \begin{remark}
%         在命题时可能将错误选项表述为 “加大基础学科研究等领域补短板力度”,这里需要辨别。
%     \end{remark}
%     \item 增强微观主体活力,发挥企业和企业家主观能动性,建立公平开放透明的市场。规则化和法治化营商环境,促进正向激励和优胜劣汰,发展更多优质企业;
% \end{enumerate}


% \section{三新一高——新发展阶段、新发展理念、新发展格局、高质量发展}

% 新发展阶段就是全面建设社会主义现代化国家、向第二个百年奋斗目标进军的阶段。进入
% 新发展阶段,是中华民族伟大复兴历史进程的大跨越。
% 坚持创新发展、协调发展、绿色发展、开放发展、共享发展,
% 是关系我国发展全局的一场深刻变革。

% 人才是第一资源。

% \subsection{新发展理念}

% \subsubsection{创新发展理念}

% \begin{itemize}
%     \item 抓住了创新, 就抓住了牵动经济社会发展全局的 “\textbf{牛鼻子}”;
%     \item 不断推进\textbf{理论创新、制度创新、科技创新、文化创新}等各方面创新;
%     \item \textbf{重大科技创新成果}是国之重器、国之利器,必须牢牢掌握在自己手上,必须依靠自力更生、自主创新;
%     \item 关键核心技术是\textbf{要不来、买不来、讨不来}的;
%     \item 突破 \textbf{“卡脖子” 关键核心技术}刻不容缓。
% \end{itemize}

% \subsubsection{绿色发展理念}

% 保护生态环境就是保护生产力,改善生态环境就是发展生产力。

% \subsubsection{开放发展理念}

% 贯彻开放发展理念,建设多元平衡、安全高效的全面开放体系,才能更好提高现代化经济体系的国际竞争力。

% \subsection{高质量发展}

% \subsubsection{坚持和完善社会主义基本经济制度}

% 公有制经济是全体人民的宝贵财富,公有制主体地位不能动摇,国有经济主导作用不能动摇,
% 这是我国各族人民共享发展成果的制度性保证,也是巩固党的执政地位,坚持我国社会主义制度的重要保证。
% \begin{remark}
%     判断题中会表述为 “公有制主体地位不能动摇,民营经济主导作用不能动摇”,注意辨别。
% \end{remark}

% \subsection{关系}

% 历史方位:进入新发展阶段,现实依据
% 指导原则:贯彻新发展理念,行动指南
% 路径选择:构建新发展格局,战略选择

% \section{加强党对经济工作的全面领导、坚持以人民为中心的发展}

% 在当今中国,没有大雨中国共产党的政治力量或其他什么力量。党政军民学,东西南北中,党是领导一切的,
% 是最高的政治领导力量。

% \section{战略选择——数字中国、数字经济: 不断做强做优做大我国数字经济}

% \begin{remark}
%     数字经济不是数字化经济。
% \end{remark}

% \subsection{2522}

% 夯实数字基础设施和数据资源体系“两大基础”、推进数字技术与经济、政治、文化、社会、生态文明建设“五位一体”深度融合,强化数字技术创新体系和数字安全屏障“两大能力”,优化数字化啊发展国内国际“两个环境”。

% \section{自测题}
% \setcounter{example}{0}

% \begin{example}
%     关于习近平经济思想基本内容的十三个方面,以下说法错误的是 \underline{\qquad \qquad \qquad}。
%     \begin{enumerate}[label=\Alph*.]
%         \item 大力发展制造业和实体经济是我国经济发展的主要着力点;
%         \item 坚持目标导向部署实施国家重大发展战略是我国经济发展的战略举措;
%         \item 坚持和完善社会主义基本经济制度是我国经济发展的制度基础;
%         \item 坚定不移全面扩大开放是我国经济发展的重要法宝;
%     \end{enumerate}
%     \begin{sol}
%         明确坚持问题导向部署实施国家重大发展战略是我国经济发展的战略举措,而不是目标导向,
%         故选 B。
%     \end{sol}
% \end{example}

% % \chapter*{后记}
% % \addcontentsline{toc}{chapter}{后记}
% % \markboth{思想政治理论类课程讲义}{后记}
% % % 设置前言标题页的页码格式为empty,即无页眉页脚
% % \thispagestyle{empty}

% % 本文件将长期更新,并接受各位老师、同学的监督和勘误。
% % 本文档中部分图片素材来源于网络,笔者在此承诺这些图片素材仅作为学习交流用途,不涉及任何商业行为。

% % \chapter*{鸣谢}
% % \addcontentsline{toc}{chapter}{鸣谢}
% % \markboth{思想政治理论类课程讲义}{鸣谢}
% % % 设置前言标题页的页码格式为empty,即无页眉页脚
% % \thispagestyle{empty}

% % 感谢我中国近现代史纲要课程的主讲老师王欣瑞副教授,

% % 感谢我的辅导员程姣姣老师,

% % 感谢机智 3581 同学,他为我们总结了中国人民反抗侵略斗争、救亡图存的尝试的历史意义和经验教训和中国人民的四次历史选择的原因、过程、深远的历史意义和深刻内涵。

% \onecolumn

\newpage
\begin{thebibliography}{1}
  \addcontentsline{toc}{chapter}{参考文献}
  \bibitem{习概2023版}
  《习近平新时代中国特色社会主义思想概论》编写组编. 习近平新时代中国特色社会主义思想概论.
  北京: 高等教育出版社, 人民出版社, 2023.8
  \bibitem{冲刺背诵笔记}
  徐涛主编,考研政治冲刺背诵笔记,北京:中国政法大学出版社,2023.9
  \bibitem{预测}
  徐涛主编,考研政治预测 6 套卷,北京:中国政法大学出版社,2023.10
  \bibitem{预测}
  徐涛,曲艺主编,考研政治考前预测必备 20 题,北京:中国政法大学出版社,2023.11
\end{thebibliography}

\input{_endpage.tex}

\end{document}