\documentclass[utf-8, 10pt]{article}

\usepackage{ctex}

\usepackage[
    paperwidth=210mm,
    paperheight=297mm,
    top=31.8mm,
    bottom=31.8mm,
    left=25.4mm,
    right=25.4mm,
    footskip=15mm % 通过这里的值来调整页脚与正文内容的垂直距离
]{geometry}

\usepackage{titlesec} % 定义标题样式

% 定义 section 标题格式
\titleformat{\section}[hang]{\heiti\centering\large\bfseries}{\thesection}{1em}{}

% 定义 subsection 标题格式
\titleformat{\subsection}[hang]{\heiti\bfseries}{\textbf{\thesubsection}}{1em}{}

% 定义 subsubsection 标题格式
\titleformat{\subsubsection}[hang]{\kaishu}{\quad\quad\thesubsubsection\,\,}{0em}{}

\usepackage{mdframed}
\mdfsetup{
  linewidth=0.4pt,
  frametitlebackgroundcolor=white, % 或者 transparent
  frametitlefont=\heiti\bfseries,
  frametitleaboveskip=10pt,
  frametitlebelowskip=5pt,
  frametitlealignment=\raggedright % 新增此行
}
\usepackage{fontspec}
% 设置 Menlo 字体
\setmonofont{Menlo}
\usepackage{fancyvrb}
\usepackage{xcolor}
\usepackage{listings}

% \definecolor{string}{HTML}{067D17}
% \definecolor{comment}{HTML}{8C8C8C}
% \definecolor{keyword}{HTML}{0033B3}
% \definecolor{class_field}{HTML}{871094}

\lstset{breaklines}
%这条命令可以让LaTeX自动将长的代码行换行排版
\lstset{extendedchars=false}
%这一条命令可以解决代码跨页时,章节标题,页眉等汉字不显示的问题
\lstset{escapeinside={(*}{*)}}

\lstset{
    basicstyle=\small\ttfamily\heiti,
    numbers=left,
    numberstyle=\scriptsize\fontspec{Menlo}, % 使用 Menlo 字体
    stepnumber=1,
    numbersep=8pt,
    frame=leftline,
    xleftmargin=2em, % 调整代码块的左边界
    framexleftmargin=0pt, % 调整边框的位置
    breaklines=true,
    % postbreak=\mbox{\textcolor{red}{$\hookrightarrow$}\space},
    % keywordstyle=\bfseries\color{keyword},          % keyword style
    % commentstyle=\heiti\color{comment},       % comment style
    % stringstyle=\color[HTML]{067D17},
    showstringspaces=false,
    % string literal style
    % escapeinside={\%*}{*)},            % if you want to add LaTeX within your code
    % morekeywords={}               % if you want to add more keywords to the set
}

\usepackage{fancyhdr} % 用于自定义页眉页脚


% 设置页眉页脚样式
\fancypagestyle{plain}{%
    \fancyhf{} % 清空页眉页脚
    \fancyhead[L]{·\thepage·} % 页眉显示页码, RO表示奇数页右侧, LE表示偶数页左侧
    \fancyhead[R]{马克思主义基本原理}
    \renewcommand{\headrulewidth}{0.4pt} % 设置页眉横线的宽度
    \renewcommand{\footrulewidth}{0pt} % 取消页脚横线
}

\renewcommand{\headrule}{\hrule width\textwidth height\headrulewidth\vskip-\headrulewidth}

% 定义取消页眉的命令
\newcommand{\cancelheader}{%
    \fancyhead{} % 清空页眉
    \renewcommand{\headrulewidth}{0pt} % 取消页眉横线
    \renewcommand{\footrulewidth}{0pt} % 设置页脚横线的宽度
}

\usepackage{caption, subcaption}
\usepackage{longtable, diagbox, booktabs}
\usepackage{float, graphicx}
\usepackage{amsthm, amssymb, amsmath, mathrsfs, mhchem, siunitx, pgfplots}
\usepackage{tikz, circuitikz, tikz-cd, tikz-3dplot}
\usetikzlibrary{decorations.markings, angles, quotes}
\usepackage{tasks, enumitem}
\usepackage{hyperref}
\hypersetup{hidelinks,
    colorlinks = true,
    allcolors = black,
    pdfstartview = Fit,
    breaklinks = true}
\usepackage[toc]{multitoc}
\usepackage{abstract}
\usepackage{extpfeil}
\usepackage{xcolor}

\everymath{\displaystyle}

\begin{document}
\newtheoremstyle{mytheoremstyle}
    {1.5ex}                                         % Space above
    {1.5ex}                                         % Space below
    {}                                              % Font for body
    {}                                              % Indent amount
    {\bfseries}                                     % Font for head
    {}                                              % Punctuation after head
    {0.5em plus 0.2em minus 0.1em}                  % Space after head
    {\thmname{#1}\thmnumber{ #2}.\thmnote{ (#3).}}

\theoremstyle{mytheoremstyle}
\newtheorem{definition}{定义}[section]
\newtheorem{example}{例}[section]
\newtheorem{exercise}{习题}[section]
\newtheorem{code}{程序清单}[section]
\newtheorem*{result}{运行结果}
\newtheorem*{keywords}{关键词}

\newtheoremstyle{my2theoremstyle}
    {1.5ex}                                         % Space above
    {1.5ex}                                         % Space below
    {\kaishu}                                              % Font for body
    {}                                              % Indent amount
    {\bfseries}                                     % Font for head
    {}                                              % Punctuation after head
    {0.5em plus 0.2em minus 0.1em}                  % Space after head
    {\thmname{#1}\thmnumber{ #2}.\thmnote{ (#3).}}

\theoremstyle{my2theoremstyle}
\newtheorem{thm}{定理}[section]
\newtheorem{law}{定律}[section]
\newtheorem{educt}{推论}
\newtheorem{prop}{命题}
\newtheorem{lemma}{引理}
\newtheorem{axiom}{公理}
\newtheorem{property}{性质}

\newtheoremstyle{my4theoremstyle}
    {1.5ex}                                         % Space above
    {1.5ex}                                         % Space below
    {}                                              % Font for body
    {}                                              % Indent amount
    {\bfseries}                                     % Font for head
    {}                                              % Punctuation after head
    {0.5em plus 0.2em minus 0.1em}                  % Space after head
    {\thmname{#1}.}

\theoremstyle{my4theoremstyle} \newtheorem*{sol}{解}

\newtheoremstyle{my3theoremstyle}
    {1.5ex}                                         % Space above
    {1.5ex}                                         % Space below
    {}                                              % Font for body
    {}                                              % Indent amount
    {\kaishu}                                       % Font for head
    {}                                              % Punctuation after head
    {0.5em plus 0.2em minus 0.1em}                  % Space after head
    {\thmname{#1}\thmnumber{ #2}.\thmnote{ (#3).}}

\theoremstyle{my3theoremstyle} \newtheorem*{remark}{注}
\newtheorem*{cmt}{评注}
% \pagestyle{plain}
\title{实践与认识及发展规律}
\author{钱锋\thanks{电子邮件: strik0r.qf@gmail.com
\newline \indent 西北工业大学软件学院, School of Software, Northwestern Polytechnical University, 西安 710072
}}

\maketitle
\thispagestyle{empty}
% \begin{abstract}
    
%     \begin{keywords}

%     \end{keywords}
% \end{abstract}
{\small \tableofcontents}

马克思主义的辩证唯物论回答了世界的本源是什么的问题,唯物辩证法的观点
则回答了作为世界本源的物质和被物质决定的意识之间关系的问题。现在我们
要聚焦于人类本身,开始探讨马克思主义的认识论,即实践与认识的有关问题。

\section{实践与认识}

\subsection{科学的实践观}

马克思在《关于费尔巴哈的提纲》中,阐明了{\color{red} 实践是感性的、对象性
的物质活动,全部社会生活本质上是实践的}。

\begin{figure}[H]
    \centering


    \tikzset{every picture/.style={line width=0.75pt}} %set default line width to 0.75pt        

    \begin{tikzpicture}[x=0.75pt,y=0.75pt,yscale=-1,xscale=1]
    %uncomment if require: \path (0,300); %set diagram left start at 0, and has height of 300
    
    %Shape: Rectangle [id:dp45009393319866886] 
    \draw   (131.5,172.5) -- (227.5,172.5) -- (227.5,200.5) -- (131.5,200.5) -- cycle ;
    %Shape: Rectangle [id:dp7565604790542991] 
    \draw   (131.5,144.5) -- (227.5,144.5) -- (227.5,172.5) -- (131.5,172.5) -- cycle ;
    %Shape: Rectangle [id:dp07030085015905996] 
    \draw   (131.5,116.5) -- (227.5,116.5) -- (227.5,144.5) -- (131.5,144.5) -- cycle ;
    %Straight Lines [id:da9100248511019812] 
    \draw    (227.5,132) -- (263,158.5) -- (229.07,185.26) ;
    \draw [shift={(227.5,186.5)}, rotate = 321.74] [color={rgb, 255:red, 0; green, 0; blue, 0 }  ][line width=0.75]    (10.93,-3.29) .. controls (6.95,-1.4) and (3.31,-0.3) .. (0,0) .. controls (3.31,0.3) and (6.95,1.4) .. (10.93,3.29)   ;
    
    % Text Node
    \draw (156,178) node [anchor=north west][inner sep=0.75pt]   [align=left] {本质层};
    % Text Node
    \draw (156,151.5) node [anchor=north west][inner sep=0.75pt]   [align=left] {现象层};
    % Text Node
    \draw (156,123.5) node [anchor=north west][inner sep=0.75pt]   [align=left] {观念层};
    % Text Node
    \draw (22,151.5) node [anchor=north west][inner sep=0.75pt]   [align=left] {(1) 感官感知};
    % Text Node
    \draw (22,123.5) node [anchor=north west][inner sep=0.75pt]   [align=left] {(2) 理性建构};
    % Text Node
    \draw (272,150) node [anchor=north west][inner sep=0.75pt]   [align=left] {(3) 通过实践认识};
    
    
    \end{tikzpicture}
    \caption{实践是感性的、对象性的物质活动}
    \label{实践是感性的、对象性的物质活动}
\end{figure}

{\small 
之所以说实践是感性的和对象性的,是因为一切的实践都来源于我们对于客观事物
的感官感知。
当我们的感觉注意到了一个客观事物(或者说客观对象)、它的主观映象在我们的
意识中形成的时候,
这个事物(对象)就成为了实践的客体。如果一个对象不能或暂时不能被我们(直接地或
间接地)感知,那么它是并不存在于我们意识里的观念世界中的,也就是说,
没有被感知到的事物尽管实际地存在着,但并没有观念地存在着。
因此,如果没有一个明确的客观对象,以及没有对它的感官感知,是没有实践的。

另一方面,实践的对象性,决定了实践的物质性。

关于社会生活本质上是实践的这一点讨论,
我们将人类社会历史及其发展规律的有关讨论中展开。}

\subsection{实践的本质与基本结构}

\subsubsection{实践的本质}

实践是人类能动地改造世界的社会性的物质活动,具有以下几个基本特征:
\begin{itemize}[itemsep=0pt]
    \item \textbf{客观实在性}或\textbf{直接现实性},这是实践的本质特征。
    \item \textbf{自觉能动性}。这是因为意识具有能动作用,而意识可以指导
    实践,因此我们可以在能动的意识的指导下能动地实践。
    \item \textbf{社会历史性}:在不同的社会历史发展阶段的实践是有所不同的。
\end{itemize}

\begin{example}
    在中国近代史上,中国社会的主要矛盾是帝国主义与中华民族的矛盾、封建主义与
    人民大众的矛盾,因此中华民族的主要任务是争取国家独立和民族解放。
    在新中国时期,随着中国社会主要矛盾的改变,中华民族的主要任务变成了
    实现国家富强和人民幸福。
\end{example}

{\small 作为对教材知识的补充,我们说实践还具有\textbf{客观规律性},
因此当它在考题中作为选项出现的时候,是可以作为正确选项来选择的。
尊重客观规律是正确发挥主观能动性
来指导实践的前提。不尊重客观规律的实践往往会取得不尽人意的结果,
但遗憾的是,正如我们即将在下一节中提到的那样,不经过实践是不能
认识到客观规律的,因此人类的历史总是在遵从客观规律的实践和违背
客观规律的实践之间来回的摆动,而人类对客观规律的认识,即观念层
对本质层的逼近,就是在这样的螺旋上升中前进的。}

\subsubsection{实践的基本结构}

实践的主体、客体和中介是实践活动的三项基本要素,三者的有机统一
构成实践的基本结构:
\begin{itemize}[itemsep=0pt]
    \item {\kaishu 实践主体是具有一定的主体能力、从事现实社会实践活动的人}。
    \item {\kaishu 实践客体是指实践活动所指向的对象}。实践客体与客观存在的事物
    不完全相同,客观事物只有在被纳入主体实践活动的范围之内、为主体实践活动
    所指向并与主体相互作用时才成为现实的实践客体。这一叙述与前文中
    我们对于实践是感性的、对象性的物质活动的叙述是一致的。
    \item {\kaishu 实践中介是指各种形式的工具、手段以及运用、操作这些
    工具、手段的程序和方法}。实践的终结系统可以分为物质性工具系统和
    语言符号工具系统两个子系统。
\end{itemize}
实践的基本结构是一个动作或者行为构成实践的前提,也就是说,在一个实践中,
主体、客体和中介缺一不可。任何的实践主体和任何的实践客体都必须通过中介相互作用。

实践主体通过中介对实践客体进行实践,这称为实践主体与客体的实践关系,
实践关系是实践主体与实践客体之间最根本的关系。实践主体通过实践去认识客体,
称为主体与客体之间的认识关系;实践的客体在主体的实践之后,要么{\kaishu 被主体基于
的需要所改造}(称为主体的客体化),要么{\kaishu 转化为主体的一部分,强化了主体
的实践能力}(称为客体的主体化),
体现了实践客体对主体的价值关系。

\subsubsection{实践形式的多样性}

实践的形式可以分为三种基本类型:
\begin{itemize}[itemsep=0pt]
    \item {\kaishu 物质生产实践},这是人类最基本的实践活动。
    \item {\kaishu 社会政治实践}。
    \item {\kaishu 科学文化实践}。
\end{itemize}
它们各具不同的社会功能,又密切地联系在一起。其中,{\color{red} 物质生产实践构成
全部社会生活的基础},另外两种实践在它的基础上产生和发展,受它的制约,
并对它产生能动的反作用。

% 随着社会的快速发展,当代人类实践出现了新的变化,其中一个例子是虚拟实践。
% 在过去几年,虚拟实践被认为是伴随着信息化和网络化发展而产生的,其实质是
% 主体和客体之间通过数字化中介系统在虚拟空间内进行的双向对象化的活动。
% 但现在随着互联网经济的深度发展,互联网已经成为了人类社会绝无仅有和最为
% 重要的神经中枢,在虚拟空间中进行的实践已经能够深深地影响到我们的现实生产实践,
% 因此就今天的观点来看,其实可以说我们已经不需要虚拟实践的这个概念了。

{
\small
之前我们说过,实践是具有社会历史性的。我们不妨以科学文化实践这一种实践形式为例
来分析一下传统的实践和今天的实践有何区别。

对于科学研究而言,传统的科学研究方法主要分为三种,分别是
理论推演、实验模拟和数值模拟(这是在计算机和科学计算产生之后才出现的)。
其中:
\begin{itemize}[itemsep=0pt]
    \item 理论推演方法的特点在于科学的抽象(近似),从而能够利用数学方法求出理论
    结果,清晰地、普遍地揭示出物质运动的内在规律。例如在流体力学的有关研究
    中,我们需要利用偏微分方程描述各种流动现象,然后寻找问题的解。
    \item 实验研究方法就是进行(实际的)
    实验,例如在风洞、激波管等实验设备中进行模型实验和实物实验,
    它的优点在于能在与所研究的问题完全相同或大体相同的条件下进行观测。
    \item 由于受到数学发展水平的局限,理论研究方法只能局限于简单的理论模型,
    难以满足生产技术日益提高的要求,而实验研究往往成本高昂,并且一些
    规模巨大或者条件极端的模型,例如大气的流动和海啸,难以在实验室中
    复现。随着计算机的发展,出现了一系列有效的能够近似实际情况的数值方法,
    例如有限差分法、有限元法等,它们的特点是成本低、精度高,但局限性在于需要对所研究
    问题的物理特性具有足够的了解。必须一提的是,由于软件是虚拟实体,
    并不像硬件那样容易逆向研发,因此对未开源的软件的依赖将会成为一个
    国家发展的弱点,在经济发展的关键领域受到技术封锁造成的一系列反应
    将是不可承受的。
\end{itemize}
随着人工智能的发展和应用,现代的科学研究形成了 “传统模型与 AI 驱动相结合”
的新模式,借助 AI,我们可以在已有模型的基础上增加一个由 AI 驱动的摄动项,
来对传统模型中无法建模的部分进行模拟和近似,从而产生精度更高的模型。
例如,在基于强化学习的主动监听
技术中,我们可以借助已有的无线通信模型和博弈论模型,再借助 AI 驱动的强化学习模型,
使得智能体能够随着对可疑通信对(例如敌方的间谍或者情报部门)监听质量的变化
而动态地调整监听策略,从而更有效的获取信息。

接下来我们来看人文学科和艺术创作。对于人文学科来说,现代的一个发展趋势就是
与统计学和数据科学的深度结合。也许在以前,大家还会认为说 “文科生” 只要具有
足够的分析、逻辑能力就足够了,欢迎对数字不敏感的同学选择人文学科,但现在
说不定有的文科生的概率统计和数据分析玩得比我们这些理科的同学还厉害。
借助 AI 的发展,现代艺术创作者可以更尽情地
挥洒自己的创意,并且艺术作品的表现成本也在逐渐降低。在以往,一个视频
特效可能需要大量的特效工作人员加班加点地应用各种效果器、调整各种参数,
但现在,借助 Sora 或者其他类似的大模型,只要有创意,个人作品也能拥有大型制作团队出品
的表现力。}

{\color{red}
\subsubsection{实践对认识的决定作用}

\begin{enumerate}[label={${\arabic*}^\circ$}, itemsep=0pt]
    \item {\kaishu 实践是认识的来源}。人们只有通过实践实际地改造和变革
    对象,才能准确把握对象的属性、本质和规律,形成正确的认识。
    \item {\kaishu 实践是认识发展的动力}。实践的需要推动认识的产生和
    发展,实践为认识的发展提供手段和条件。
    \item {\kaishu 实践是认识的目的}。认识活动的目的在于更好地指导实践
    而非认识活动本身,认识指导实践、为实践服务的过程,就是认识价值
    的实现过程。
    \item {\kaishu 实践是检验认识真理性的唯一标准}。认识是否具有真理性,
    即不能从认识本身得到证实,也不能从认识对象中得到回答,只有在实践中
    才能得到验证。
\end{enumerate}
总而言之,实践是认识的起点、目的和最终归宿,是全部认识的基础,实践的观点
是马克思主义认识论的第一观点、基本观点。}

\begin{thebibliography}{1}
    \addcontentsline{toc}{section}{参考文献}
    \bibitem{2023版教材}
    《马克思主义基本原理 (2023 版)》编写组. 马克思主义基本原理: 2023 版[M].
    北京: 高等教育出版社, 2023.
    \bibitem{核心考案}
    徐涛主编, 考研政治核心考案[M], 北京:中国政法大学出版社, 2023.1
    \bibitem{优题库}
    徐涛主编, 考研政治通关优题库[M], 北京:中国政法大学出版社, 2023.2
    \bibitem{真题库}
    徐涛主编, 考研政治必刷真题库[M], 北京:中国政法大学出版社, 2023.2
    \bibitem{冲刺背诵笔记}
    徐涛主编, 考研政治冲刺背诵笔记[M], 北京:中国政法大学出版社, 2023.9
    \bibitem{预测}
    徐涛主编, 考研政治预测 6 套卷[M], 北京:中国政法大学出版社, 2023.10
    \bibitem{预测}
    徐涛, 曲艺主编, 考研政治考前预测必备 20 题[M], 北京:中国政法大学出版社, 2023.11
\end{thebibliography}

\end{document}