\documentclass[utf-8, 10pt]{article}

\usepackage{ctex}

\usepackage[
    paperwidth=210mm,
    paperheight=297mm,
    top=31.8mm,
    bottom=31.8mm,
    left=25.4mm,
    right=25.4mm,
    footskip=15mm % 通过这里的值来调整页脚与正文内容的垂直距离
]{geometry}

\usepackage{titlesec} % 定义标题样式

% 定义 section 标题格式
\titleformat{\section}[hang]{\heiti\centering\large\bfseries}{\thesection}{1em}{}

% 定义 subsection 标题格式
\titleformat{\subsection}[hang]{\heiti\bfseries}{\textbf{\thesubsection}}{1em}{}

% 定义 subsubsection 标题格式
\titleformat{\subsubsection}[hang]{\kaishu}{\quad\quad\thesubsubsection\,\,}{0em}{}

\usepackage{mdframed}
\mdfsetup{
  linewidth=0.4pt,
  frametitlebackgroundcolor=white, % 或者 transparent
  frametitlefont=\heiti\bfseries,
  frametitleaboveskip=10pt,
  frametitlebelowskip=5pt,
  frametitlealignment=\raggedright % 新增此行
}
\usepackage{fontspec}
% 设置 Menlo 字体
\setmonofont{Menlo}
\usepackage{fancyvrb}
\usepackage{xcolor}
\usepackage{listings}

% \definecolor{string}{HTML}{067D17}
% \definecolor{comment}{HTML}{8C8C8C}
% \definecolor{keyword}{HTML}{0033B3}
% \definecolor{class_field}{HTML}{871094}

\lstset{breaklines}
%这条命令可以让LaTeX自动将长的代码行换行排版
\lstset{extendedchars=false}
%这一条命令可以解决代码跨页时,章节标题,页眉等汉字不显示的问题
\lstset{escapeinside={(*}{*)}}

\lstset{
    basicstyle=\small\ttfamily\heiti,
    numbers=left,
    numberstyle=\scriptsize\fontspec{Menlo}, % 使用 Menlo 字体
    stepnumber=1,
    numbersep=8pt,
    frame=leftline,
    xleftmargin=2em, % 调整代码块的左边界
    framexleftmargin=0pt, % 调整边框的位置
    breaklines=true,
    % postbreak=\mbox{\textcolor{red}{$\hookrightarrow$}\space},
    % keywordstyle=\bfseries\color{keyword},          % keyword style
    % commentstyle=\heiti\color{comment},       % comment style
    % stringstyle=\color[HTML]{067D17},
    showstringspaces=false,
    % string literal style
    % escapeinside={\%*}{*)},            % if you want to add LaTeX within your code
    % morekeywords={}               % if you want to add more keywords to the set
}

\usepackage{fancyhdr} % 用于自定义页眉页脚


% 设置页眉页脚样式
\fancypagestyle{plain}{%
    \fancyhf{} % 清空页眉页脚
    \fancyhead[L]{·\thepage·} % 页眉显示页码, RO表示奇数页右侧, LE表示偶数页左侧
    \fancyhead[R]{马克思主义基本原理}
    \renewcommand{\headrulewidth}{0.4pt} % 设置页眉横线的宽度
    \renewcommand{\footrulewidth}{0pt} % 取消页脚横线
}

\renewcommand{\headrule}{\hrule width\textwidth height\headrulewidth\vskip-\headrulewidth}

% 定义取消页眉的命令
\newcommand{\cancelheader}{%
    \fancyhead{} % 清空页眉
    \renewcommand{\headrulewidth}{0pt} % 取消页眉横线
    \renewcommand{\footrulewidth}{0pt} % 设置页脚横线的宽度
}

\usepackage{caption, subcaption}
\usepackage{longtable, diagbox, booktabs}
\usepackage{float, graphicx}
\usepackage{amsthm, amssymb, amsmath, mathrsfs, mhchem, siunitx, pgfplots}
\usepackage{tikz, circuitikz, tikz-cd, tikz-3dplot}
\usetikzlibrary{decorations.markings, angles, quotes}
\usepackage{tasks, enumitem}
\usepackage{hyperref}
\hypersetup{hidelinks,
    colorlinks = true,
    allcolors = black,
    pdfstartview = Fit,
    breaklinks = true}
\usepackage[toc]{multitoc}
\usepackage{abstract}
\usepackage{extpfeil}
\usepackage{xcolor}
\usepackage{multirow}

\everymath{\displaystyle}

\begin{document}
\newtheoremstyle{mytheoremstyle}
    {1.5ex}                                         % Space above
    {1.5ex}                                         % Space below
    {}                                              % Font for body
    {}                                              % Indent amount
    {\bfseries}                                     % Font for head
    {}                                              % Punctuation after head
    {0.5em plus 0.2em minus 0.1em}                  % Space after head
    {\thmname{#1}\thmnumber{ #2}.\thmnote{ (#3).}}

\theoremstyle{mytheoremstyle}
\newtheorem{definition}{定义}[section]
\newtheorem{example}{例}[section]
\newtheorem{exercise}{习题}[section]
\newtheorem{code}{程序清单}[section]
\newtheorem*{result}{运行结果}
\newtheorem*{keywords}{关键词}

\newtheoremstyle{my2theoremstyle}
    {1.5ex}                                         % Space above
    {1.5ex}                                         % Space below
    {\kaishu}                                              % Font for body
    {}                                              % Indent amount
    {\bfseries}                                     % Font for head
    {}                                              % Punctuation after head
    {0.5em plus 0.2em minus 0.1em}                  % Space after head
    {\thmname{#1}\thmnumber{ #2}.\thmnote{ (#3).}}

\theoremstyle{my2theoremstyle}
\newtheorem{thm}{定理}[section]
\newtheorem{law}{定律}[section]
\newtheorem{educt}{推论}
\newtheorem{prop}{命题}
\newtheorem{lemma}{引理}
\newtheorem{axiom}{公理}
\newtheorem{property}{性质}

\newtheoremstyle{my4theoremstyle}
    {1.5ex}                                         % Space above
    {1.5ex}                                         % Space below
    {}                                              % Font for body
    {}                                              % Indent amount
    {\bfseries}                                     % Font for head
    {}                                              % Punctuation after head
    {0.5em plus 0.2em minus 0.1em}                  % Space after head
    {\thmname{#1}.}

\theoremstyle{my4theoremstyle} \newtheorem*{sol}{解}

\newtheoremstyle{my3theoremstyle}
    {1.5ex}                                         % Space above
    {1.5ex}                                         % Space below
    {}                                              % Font for body
    {}                                              % Indent amount
    {\kaishu}                                       % Font for head
    {}                                              % Punctuation after head
    {0.5em plus 0.2em minus 0.1em}                  % Space after head
    {\thmname{#1}\thmnumber{ #2}.\thmnote{ (#3).}}

\theoremstyle{my3theoremstyle} \newtheorem*{remark}{注}
\newtheorem*{cmt}{评注}
% \pagestyle{plain}
\title{进入近代后中华民族的磨难与抗争}
\author{钱锋\thanks{电子邮件: strik0r.qf@gmail.com
\newline \indent 西北工业大学软件学院, School of Software, Northwestern Polytechnical University, 西安 710072
}}

\maketitle
\thispagestyle{empty}
% \begin{abstract}
    
%     \begin{keywords}

%     \end{keywords}
% \end{abstract}
{\small \tableofcontents}

本章的内容概括来讲就是 “落后就要挨打”、怎么挨打? 怎么反抗、反抗失败的经验和教训, 以及
我们反抗本身和反抗失败所导致的民族意识的觉醒. 本章的主要知识结构如下:
\begin{enumerate}[label=1.\arabic*, itemsep=0pt]
    \item 鸦片战争前后的中国与世界
    \item 西方列强对中国的侵略
    \item 外国武装侵略的斗争
    \item 反侵略战争的失败与民族意识的觉醒
\end{enumerate}
本章的学习目的是了解鸦片战争前后的中国与世界,认识近代中国社会的半殖民地半封建
社会的性质、主要社会矛盾和基本特征,并了解因近代中国社会主要矛盾的转变而产生的中华民族在近代
的两大历史任务和它们之间的相互关系,认识造成近代中国社会贫困落后的根本原因,并通过对近代中
国人民抵御外国侵略的斗争历史认识近代中国历次反侵略战争失败的根本原因,继承和发扬以爱国主义为
核心的民族精神。

其中,\textbf{中华民族在近代面临的两大历史任务及其相互关系和近代中国反侵略战争失败的原因是本章学习的重点};
\textbf{对鸦片战争前后的中国与世界的认知、理解和近代中国历次反侵略战争失败的原因和教训是本章学习的难点};
各种历史事件的意义、教训和成因是考察重点,对进入近代后的代表性历史事件的时间节点在考察中一般不作要求。

\section{鸦片战争前后的中国与世界}

这一块的内容可以总结为世界走向中国与中国走向世界。随着外国资本主义的入侵,
中国的封建社会逐步变成了半殖民地半封建社会,中国人民逐渐开始了反帝反封建的
资产阶级民主革命。虽然中国走向世界和世界走向中国的时间不同、方式不同、性质
不同、目的不同,但是从全球的观点来看,这个历史阶段最显著的变化是中国成为了
全球性国际社会中的一员,成为了世界的中国。中国外交从单向发展为双向,中国也
基本走向了以欧洲为主体的资本主义世界,成为了全球性国际社会的一员。

那么,中国的封建社会为什么会走向衰落?西方的资本—帝国主义为什么要来到东方?
鸦片战争对近代的中国社会造成了什么样的的改变?为什么鸦片战争是中国近代史
的起点呢?本节我们就要探讨这些问题。

\subsection{中国封建社会的衰落}

% \subsubsection{中国封建社会的主要特点}

自公元前 5 世纪的战国时代到 1840 年鸦片战争,中国的封建社会前后延续了两千多年。
中国封建社会的主要特点是:
\begin{itemize}[itemsep=0pt]
    \item 经济上,封建地主土地所有制经济占主导地位;
    \item 政治上,实行高度集权的封建君主专制制度;
    \item 文化上,中国封建社会的文化思想体系以儒家思想为核心;
    \item 中国封建社会的社会结构特点是族权和政权相结合的封建宗法等级制度,其核心是宗族家长制。
\end{itemize}
中国封建社会的经济、政治、文化、社会结构,一方面巩固和维系了中国封建制度的稳定和延续,
另一方面也使其前进缓慢甚至迟滞,并造成不可克服的周期性的政治经济危机。
\begin{remark}
    在这里需要简单说明的是,中国古代的封建君主制、封建君主专制和封建制度是不能划等号的。
    封建制度是一个很大的范畴,在政治方面,它的表现是封建君主专制;经济方面,它的表现是
    封建地主土地所有制,本质上就是土地剥削;而在文化方面,封建制度的体现是一整套的伦理纲常。
    中国的封建社会,就是一种处在这样的封建制度中的社会。
\end{remark}
\begin{remark}
    注意这里的这个 “周期性的政治危机”,以后还会出现。
\end{remark}

\subsection{世界资本主义的发展与殖民扩张}

\begin{itemize}[itemsep=0pt]
    \item 14 世纪至 15 世纪,在欧洲地中海沿岸的城市里,最早出现了资本主义的萌芽。
    \item 欧洲的文艺复兴冲破了中世纪神学蒙昧主义的精神束缚,
    为欧洲资本主义市场的产生做了思想上的准备;
    \item 而 15 世纪以来的地理大发现,更为欧洲开拓世界市场、发展海外贸易、
    推动殖民扩张提供了条件,加速了欧洲资本主义的兴起;
    \item 1640 年的英国资产阶级革命,标志着世界历史开始进入资本主义时代。
    \item 18 世纪,英国、美国、法国等先后通过资产阶级革命,建立了资产阶级政权,
    为资本主义的发展提供了政治上的前提和保证。
    \item 18 世纪中叶至 19 世纪中叶,从英国开始然后迅速推广到欧美各国的工业革命,
    使大机器生产取代了工场手工业,资本主义经济得到迅速发展。
    \item 19 世纪末,资本主义进入帝国主义阶段,资本输出成为殖民剥削的重要形式,
    并出现瓜分世界的狂潮。西方资本主义的发展及其向东方的殖民扩张,
    使古老的中国遇到了空前严重的挑战。
\end{itemize}

\subsection{鸦片战争的爆发}

\subsubsection{西方学界的主流观点:文化价值冲突论}

1793 年,George Macartney 率英国使团来华,当时的清朝并没有近代的外交理念,只有
宗藩观念。Macartney 抵华后,清廷一如既往地视为 “贡使”,朝见中国皇帝依例当行三跪九叩大礼,
Macartney 抗不遵从,经反复磋商,清廷同意屈一膝以为礼,并拒绝了 Macartney 提出的交涉要求。

Macartney 使团的来华经历成为了西方为发动侵略战争狡辩的借口,
1841 年 12 月,美国众议院外交委员会主席 J.Q.Adams 发表演说,“一般的看法都以为争执不过是
为了英国商人输入几箱鸦片,中国政府因其违法输入而予以查抄,但是我却认为这完全是错误的看法。
这只不过是争端中的一个偶然事故,而并不是战争的原因……战争的原因是磕头!”
Adams 的解释是:通过正常的外交途径无法构建平等的国家关系,只有诉诸战争。\cite{两岸晚清}

西方学界流行的 “文化价值冲突论”,虽有些道理,但不全面,没有把握到战争爆发最主要的根源,
反而更像是侵略者为自身邪恶行径所做的辩解。

\subsubsection{鸦片战争}

鸦片战争的发生是综合因素所致,但其中最重要的不是文化因素,而是经济原因,在于茶叶、白银、
鸦片等。
西方的资本主义社会具有一个主要矛盾,那就是生产的无限扩大于劳动人民具有支付能力的需求之间的
矛盾,这个矛盾决定了西方资本主义有向外扩张的趋势,他们需要找到一片新的市场作为商品的倾销地
和原料的供应地。因此,西方殖民主义势力来到东方,不是使包括中国在内的东方国家为了成为独立的
资本主义国家,而是为了把东方世界纳入资本主义的世界体系,成为殖民地、半殖民地,成为西方世界
经济上、政治上、文化上的附庸。

英国殖民者以走私毒品鸦片作为牟取暴利及改变贸易逆差的手段,强迫其殖民地印度种植鸦片,再由
东印度公司垄断收购、加工,然后走私到中国贩卖。
钦差大臣林则徐于 1839 年 6 月在广东虎门销毁所收缴鸦片的行动,完全是维护国家利益和民族尊严
的正义行动。

为了解决英国对华长期处于贸易入超状态的问题,和 1825 年、1847 年英国发生的两次资本主义经济危机,
并转移国内人民的视线,英国政府迫不及待地要发动一场侵略战争。
1840 年下午 4 月 4 日,英国政府将对华战争议案提交国会,遭
反对党质询,9 日下午,在辩论无果的情况下就反战质询案表决,赞成 262 人,反对 271 人。
“托利党的反战决议案只以五票之差被否决”。一个相当微弱的多数,却在某种程度上决定了对
世界上人口最多的国家的战争的动议得到通过。

鸦片战争以清政府的失败而告终,此后,在这段屈辱的历史时期内,中国与列强签订了一系列的不平等
条约:
\begin{table}[H]
    \centering
    \caption{中国与列强签订的不平等条约(不完全)}
    \label{table: 中国与列强签订的不平等条约(不完全)}
    \begin{tabular}{c c c p{6cm}}
        \hline
        条约名 & 签署国 & 签署时间 & 备注 \\ 
        \hline 
        《南京条约》 & 英国 & 1842 & 近代史上第一个不平等条约,标志着中国开始沦为半殖民地半封建社会 \\
        《虎门条约》 & 英国 & 1843 & \\ 
        《望厦条约》 & 美国 & 1844.7 & \\ 
        《黄埔条约》 & 法国 & 1844.10 & \\ 
        《马关条约》 & 日本 & 1895 & 中国沦为半殖民地半封建社会的程度大大加深了 \\ 
        《辛丑条约》 & 八国联军 & 1901& 标志着以慈禧太后为首的清政府已经彻底放弃
        了抵抗外国侵略者的念头,甘当 “洋人的朝廷”。 \\
        \hline
    \end{tabular}
\end{table}

通过这一系列的不平等条约,英国等西方列强在中国攫取了大量侵略特权。破坏了中国的主权的领土完整,
破坏了中国的领海、司法、关税主权……

随着外国资本主义的入侵,中国的社会性质开始发生质的变化。中国逐步沦为
半殖民地半封建国家。随着社会主要矛盾的变化,中国逐渐开始了
反帝反封建的资产阶级民主革命。正因为如此,鸦片战争成为中国近代史的起点。

\subsubsection{近代中国社会的半殖民地半封建性质}

从鸦片战争直到中华人民共和国成立前夕的中国社会,处于一种称为\textbf{半殖民地半封建社会}
的畸形社会形态。社会性质就是最大的国情,今后在讨论任何中国近代史的社会问题
的时候,都要从这个大前提、大背景出发。
中国半殖民地半封建社会及其特征,是随着资本—帝国主义侵略的扩大、资本—帝国主义与
中国封建势力结合的加深而逐渐形成的。
那么,为什么是半殖民地半封建社会呢?主要是两点原因:
\begin{itemize}[itemsep=0pt]
    \item \textbf{首先,独立的逐步中国变成了半殖民地的中国。}
    鸦片战争以后,中国已经丧失了完全独立的地位,在相当程度上被殖民地化了,
    但近代中国虽然在十几岁已经丧失拥有完整主权的独立国家的地位,但仍然维持着
    独立国家和政府的名义,还有一定的主权。由于它与连名义上的独立也没有而由
    殖民主义宗主国直接统治的殖民地尚有区别,因此被称作半殖民地。
    
    \item \textbf{其次,封建的中国逐步变成了半封建的中国。}
    资本—帝国主义列强用武力打开中国的门户,把中国卷入世界资本主义经济体系和世界市场之中,
    但资本—帝国主义列强并不愿意中国成为独立的资本主义国家。这样,中国的经济既不再是完全的
    封建经济,也不是完全的资本主义经济,成为了半殖民地半封建的经济。
\end{itemize}
简单地来说就是,列强用武力打开了中国的大门以后,我们的封建社会开始瓦解了,
我们的资本主义社会开始形成但没有完全形成,就这样 “卡” 在了中间,变成了
半殖民地半封建社会这种畸形的社会形态。

\begin{mdframed}
    \begin{thm}
        请论述中国半殖民地半封建社会的基本特征。(政治、经济、社会三个方面)
    \end{thm}
    \begin{enumerate}[label=(\arabic*), itemsep=0pt]
        \item 政治方面:
        \begin{enumerate}[label=$\textup{\alph*}$., itemsep=0pt]
            \item 资本—帝国主义日益成为支配中国的决定性力量;
            \item 封建势力日益成为资本—帝国主义压迫、奴役中国人民的社会基础和统治支柱;
        \end{enumerate}
        \item 经济方面:
        \begin{enumerate}[label=$\textup{\alph*}$., itemsep=0pt]
            \item 封建剥削制度的根基即封建地主的土地所有制依然在广大地区内保持着;
            \item 民族资本主义经济虽然已经产生,但发展很缓慢,力量很微弱;
        \end{enumerate}
        \item 社会方面:
        \begin{enumerate}[label=$\textup{\alph*}$., itemsep=0pt]
            \item \textbf{近代中国各地区经济、政治和文化的发展极不平衡};
            \item 广大人民尤其是农民日益贫困化以至大批破产,
            过着饥寒交迫和毫无政治权利的生活。
        \end{enumerate}
    \end{enumerate}
\end{mdframed}

\subsubsection{近代中国社会阶级关系的变动}

鸦片战争前,中国社会只有地主阶级和农民阶级两大阶级,鸦片战争后,中国的大门被迫打开,
中国社会的新阶级开始产生。19 世纪四五十年代,中国工人阶级最早出现于外国资本主义的在华企业中。

两个传统阶级:地主阶级、农民阶级;

两个新兴阶级:工人阶级、资产阶级。

% \begin{cmt}
%     各个新阶级的产生时间:
%     \begin{enumerate}[label=\textup{\arabic*}${}^\circ$]
%         \item 工人阶级:19 世纪四五十年代;
%         \item 民族资产阶级:19 世纪六七十年代;
%         \item 城市小资产阶级:19 世纪末;
%         \item 官僚买办资产阶级:20 世纪 20 年代末。
%     \end{enumerate}
% \end{cmt}

\textbf{近代中国诞生的工人阶级}是中国新生产力的代表。它深受帝国主义、封建势力、资产阶级三重压迫,
受剥削最深、革命性最强,\textbf{是近代中国最革命的阶级}。

\subsubsection{近代中国社会的主要矛盾和两大历史任务}

首先我们要讲的是国情、社会矛盾和历史任务的关系。一切从国情出发,而社会性质是最大的国情。
在讨论中国近代史的任何社会问题的时候,都要从半殖民地半封建社会这个大前提、大背景出发。
当时的我们处于半殖民地半封建社会,而造成这一社会性质产生的始作俑者是帝国主义侵略者和
长期封建统治中国的封建地主阶级。\textbf{国情决定社会矛盾,社会矛盾决定历史任务}。
在近代中国半殖民地半封建的错综复杂的社会矛盾中,占支配地位的主要矛盾,
是\textbf{帝国主义和中华民族的矛盾},\textbf{封建主义和人民大众的矛盾}。
中国近代社会的发展和演变,是上述两队主要矛盾相互交织和交替作用的结果。近代以来伟大的中国
革命,是在这些主要矛盾及其激化的基础知识发生和发展起来的。

中国近代社会的主要矛盾决定了,为了使中国在世界上站起来,
为了使中国人民过上幸福、富裕的生活,就必须推翻帝国主义、封建主义联合统治的半殖民地半封建
的社会制度,\textbf{争取民族独立、人民解放};就必须改变中国经济技术落后的面貌,
\textbf{实现国家富强、人民幸福}。这两个历史任务相互区别又相互紧密联系的。
完成前一项任务是为后一项任务扫清障碍、创造必要的前提。
只有通过革命取得民族独立、人民解放以后,中国人民才有可能集中力量进行现代化建设,
实现国家富强和人民富裕,
所以说,前一个任务是后一个任务的必要条件,后一个任务是前一个任务的最终目的和必然要求。
试想一下,不通过革命战争赶走侵略者、解放中国人民、创造一个好的执政环境和政治舞台,
怎么通过发展来实现国家的富强和人民的幸福呢?所以这两大历史任务的关系是不难理解的。

\begin{mdframed}
    \begin{cmt}
        鸦片战争对近代的中国社会造成了深远的改变,在鸦片战争后,中国社
        会的以下各个方面发生了巨大的变化:
        \begin{enumerate}[label=(\arabic*), itemsep=0pt]
            \item \textbf{中国的社会性质发生根本性变化},
            在鸦片战争后,独立的中国逐步变成了半殖民地的中国,
            封建的中国逐步变成了半封建的中国。
            中国由一个落后封闭的封建国家沦为一个\textbf{半殖民地半封建社会}。
            \item 中国社会阶级结构发生变化,
            除了原有的农民阶级和地主阶级外,出现了\textbf{两大新兴的阶级——资产
            阶级与无产阶级}。近代中国诞生的工人阶级是中国新生产力的代表,
            是近代中国最革命的阶级。
            \item \textbf{社会主要矛盾发生变化},
            主要矛盾由封建主义和人民大众的矛盾转化为\textbf{封建主义和人民大众
            的矛盾以及帝国主义与中华民族的矛盾}。其中,帝国主义与中华民
            族的矛盾是最主要的矛盾。
            \item \textbf{社会任务发生变化},
            近代以来中华民族面临的两大历史任务,就是\textbf{争取民族独立、人民解放
            和实现国家富强、人民幸福}。其中,前一个任务为后一个任务扫除障碍,创造必要
            的前提,后一个任务是前一个任务的最终目的和必然要求。
        \end{enumerate}
        随着社会主要矛盾的变化,\textbf{中国逐步开始了反帝反封建的资产
        阶级民主革命},正因如此,鸦片战争成为中国近代史的起点。
    \end{cmt}
\end{mdframed}

在这里扯一个题外话说明一下,本书勾划重点的方式主要有粗体、打框和标记 “核心要点” 三种方式。
其中,粗体的内容是一些关键词,大家只需要记住这些关键词然后能用自己的语言串联出这些关键词
之间的关系就可以了。打框的内容是选择题考点,这意味着你一定要知道有这样的表述,当这些内容
在选项中出现的时候你要能想得起来并且识别出来可能被偷换的用词,这些往往是命题的陷阱。打框
的内容会在行文过程中反复出现,因为优秀的课程设计在于重复,所以我在写书的时候决定用这种方式
来强迫你把一些重点的内容多重复几遍,不要嫌烦哦同学(嫌烦就对了!嫌烦意味着你已经熟练到一定
程度了)。“核心要点” 部分是我本人根据学习和做题的经验总结出来的一些内容,
这些内容往往在书上没有原文并且高度凝练,是一些在纵向上联系了本课程的各个章节,
又在横向上联系了其他四门思想政治理论课的综合性的表述,
对你的理解很有帮助。而\textbf{既打了框,又标注了 “核心要点” 的内容,就是全书的重中之重,
是你在复习备考的时候一定要过一遍的内容,记得越熟越好}。所以,请你再折回去,把上面既打了框
又是 “核心要点” 的这一段内容再看一遍。

\subsection{复习参考题}

\begin{example}
    中国从封建社会逐步沦为半殖民地半封建社会,中国社会的阶级关系也发生了深刻的变动。
    近代中国最早产生的新的阶级是 \underline{\qquad \qquad \qquad}。
    \begin{tasks}[label={\Alph*. }](4)
        \task 民族资产阶级
        \task 官僚买办资产阶级
        \task 工人阶级
        \task 城市小资产阶级
    \end{tasks}
\end{example}

\begin{mdframed}[frametitle={鸦片战争的影响:中国社会的阶级变化}]
    \begin{enumerate}[itemsep=0pt]
        \item 两个传统阶级:封建阶级和农民阶级。
        \item \textbf{两大新兴阶级}:资产阶级与无产阶级。近代中国诞生的工人阶级是中国新生产力的代表,
        是近代中国最革命的阶级。
    \end{enumerate}
\end{mdframed}

\begin{example}
    1840 年 6 月,英国舰队封锁了珠江海口和广州海面,鸦片战争正式爆发。下列关于
    鸦片战争的后果,说法错误的是:
    \begin{tasks}[label={\Alph*. }](1)
        \task 签订了中国近代史上第一个不平等条约
        \task 外国人在华不受中国法律管束
        \task 允许外国人在华开办工厂
        \task 帝国主义与中华民族的矛盾成为中国社会最主要的矛盾
    \end{tasks}
\end{example}

\begin{example}
    (多选)鸦片战争以后,西方列强通过发动侵略战争,强迫中国签订了一系列不平等条约,
    使中国沦为半殖民地半封建社会。中国逐步沦为半殖民地社会的原因包括:
    \begin{tasks}[label={\Alph*. }](1)
        \task 中国已经丧失了完全独立的地位
        \task 列强把中国卷入世界资本主义经济体系和世界市场之中
        \task 中国仍然维持着独立国家和政府的名义
        \task 西方列强不愿意中国成为独立的资本主义国家
    \end{tasks}
\end{example}

\begin{mdframed}[frametitle={为什么是半殖民地半封建社会?}]
    \begin{enumerate}[itemsep=0pt]
        \item 独立的中国逐步变成了半殖民地的中国:中国已经方式了完全独立的地位,但
        仍然维持着独立国家和政府的名义;
        \item 封建的中国逐步变成了半封建的中国:列强把中国卷入世界资本主义经济体系
        和世界市场之中,但不愿意中国成为独立的资本主义国家。
    \end{enumerate}
\end{mdframed}

\section{西方列强对中国的侵略}

本节的主要考点在于考察西方列强对中国侵略的具体表现,一般来讲,在选项中我们是能够
识别出具体侵略罪行所属的领域的。总而言之,在一般情况下,与领土主权有关的都是军事侵略,
与行政权力有关的都是政治控制,与钱有关的都是经济掠夺,与思想文化有关的都是文化渗透。
少数的特例(或者不易识别的选项)需要在平时解题过程中注意积累。

接下来我们把西方列强对中国的侵略行径不完全地罗列一下,你可以在闲着没事干的时候翻翻看看,
你没必要背,也可以为了备考而先跳过这张表格,但是请你一定要抽空回来看,你会直观的感受到
什么叫落后就要挨打,什么叫罄竹难书。
\begin{longtable*}{p{0.12\textwidth}|p{0.08\textwidth}|p{0.08\textwidth}|p{0.14\textwidth}|p{0.45\textwidth}}
    \toprule
    \textbf{侵略领域} & \textbf{列强} & \textbf{时间} 
    & \textbf{侵略行径} & \textbf{史实} \\
    \midrule
    \endhead
    军事侵略 & 英国 & 1840 & 发动侵略战争 & 鸦片战争 \\
    \hline
    军事侵略 & 英国 & 1842 & 侵占中国领土 & 强迫清政府签订《南京条约》,
    把香港岛割让给英国。 \\
    \hline
    军事侵略 & 英国 & 1845 & 划分势力范围 & 
    租得上海外滩附近 837 亩土地,设立上海英租界。 \\
    \hline
    军事侵略 & 英、法、美、德、日、俄、意、比、奥等国 & 1845 至 1911 & 划分势力范围 & 
    先后在上海、天津、汉口、广州、福州、重庆等 16 个城市,设立了 30 多个租界。 \\
    \hline
    军事侵略 & 葡萄牙 & 1849 & 侵占中国领土 & 武力强占澳门半岛。 \\
    \hline
    军事侵略 & 俄国 & 1858 & 侵占中国领土 & 
    利用英、法发动第二次鸦片战争之机,胁迫黑龙江将军奕山签订《瑷珲条约》,
    割去黑龙江以北 60 万平方公里领土。 \\
    \hline
    军事侵略 & 英国 & 1860 & 侵占中国领土 & 通过中英《北京条约》,
    割去香港岛对岸九龙半岛南端和昂船洲。 \\
    \hline
    军事侵略 & 俄国 & 1860 & 侵占中国领土 & 
    通过签订中俄《北京条约》,割去乌苏里江以东 40 万平方公里领土。 \\
    \hline
    军事侵略 & 俄国 & 1864 & 侵占中国领土 & 
    强迫清政府签订《勘分西北界约记》,割去中国西北 44 万平方公里领土。 \\
    \hline
    军事侵略 & 俄国 & 1881 & 侵占中国领土 & 
    通过《改订伊犁条约》和 5 个勘界议定书,割去中国西北 7 万多平方公里领土。
    通过这一系列不平等条约,俄国共侵占中国领土 150 多万平方公里。 \\
    \hline
    军事侵略 & 葡萄牙 & 1887 & 侵占中国领土 & 
    胁迫清政府订立《中葡和好通商条约》,允许葡萄牙 “永居管理澳门”。 \\
    \hline
    军事侵略 & 日本 & 1894.11 & 屠杀中国人民 
    & 日军在甲午战争中制造了旅顺大屠杀惨案,在 4 天内连续屠杀中国居民 2 万余人。 \\
    \hline
    军事侵略 & 日本 & 1895 & 侵占中国领土 
    & 日本强迫清政府签订《马关条约》,割去中国台湾全岛及其所有附属岛屿和澎湖列岛。 \\
    \hline
    军事侵略 & 德国 & 1898 & 划分势力范围 
    & 强租山东的胶州湾,把山东划为其势力范围。 \\
    \hline
    军事侵略 & 沙俄 & 1898 & 划分势力范围 
    & 强租辽东半岛的旅顺口、大连湾及其附近海面,以长城以北为其势力范围。 \\
    \hline
    军事侵略 & 英国 & 1898 & 划分势力范围 
    & 强租山东的威海卫和香港岛对岸的九龙半岛界限街以北、深圳河以南及附近的岛屿(新界),
    以长江流域为其势力范围。 \\
    \hline
    军事侵略 & 法国 & 1899 & 划分势力范围 
    & 法国强租广东的广州湾及其附近水面,把广东、广西、云南作为其势力范围。 \\
    \hline
    军事侵略 & 日本 & 1899 & 划分势力范围 
    & 声明把福建作为其势力范围。 \\
    \hline
    军事侵略 & 俄国 & 1900.7 & 屠杀中国人民 
    & 俄国入侵中国东北时,先后制造了海兰泡惨案和江东六十四屯惨案。沙俄军警
    把中国人居住的村庄烧光,把数千居民枪杀,或驱入黑龙江中活活淹死。 \\
    \hline
    军事侵略 & 八国联军 & 1900.8 & 屠杀中国人民 
    & 八国联军侵占北京后,仅在庄王府一处,就烧死和杀死义和团团民与平民 1700 多人。 \\
    \hline
    军事侵略 & 八国联军 & 1901 & 划分势力范围 
    & 《辛丑条约》规定,外国军队有权在北京使馆区和北京至大沽、山海关一线
    包括天津、唐山等 12 处 “留兵驻守”。 \\
    \hline
    军事侵略 & 日本 & 1901 & 划分势力范围 
    & 日俄战争后,日本从俄国手中攫得租自中国的旅顺口的大连湾、长春至旅顺口的铁路及
    其他有关权益,在旅顺设置 “关东总督府”,并派兵驻守上述地区及 “南满铁路” 沿线。
    这支军队后来被称作 “关东军”,成了日本侵略中国的突击队。 \\
    \hline
    \midrule
    政治控制 \\ 
    \midrule 
    经济掠夺 \\ 
    \midrule
    文化渗透 \\
    \bottomrule
\end{longtable*}

(未完待续,现在只列了不完全统计的 1/10 左右,任重而道远啊,罄竹难书,罄竹难书)

% \begin{table}[H]
%     \centering
%     \caption{细数西方列强的罪恶}
%     \begin{tabular}{c p{0.5\textwidth}}
%         \toprule
%         领域 & 列强国家 & 具体表现 \\ 
%         \midrule
%         军事侵略 & 英国 & 发动侵略战争,划分势力范围,勒索赔款\\
%         政治控制 &  & 控制内政外交,镇压人民反抗,扶植、收买代理人,把持海关行政权\\ 
%         经济掠夺 &  & 控制通商口岸,剥夺关税自主,商品倾销,资本输出,操控中国经济\\ 
%         文化渗透 &  & 假传教真侵略,炮制中国威胁论,办报纸,庚款兴学 \\ 
%         \bottomrule
%     \end{tabular}
% \end{table}

我们在这里针对一些容易混淆的问题简单说明一下,
勒索赔款的重点在勒索,试想一下,我拿一把刀架在你脖子上要你给我钱,你敢不给吗?
所以,西方列强在战后勒索一笔赔款,本质上还是对我们的军事侵略。

除此之外,对于海关权和关税的领域划分是一个容易混淆的问题。按照我们的原则,
与行政权力有关的都是政治控制,所以把持海关行政权本质上是政治控制,而与钱有关的都是
经济掠夺,所以剥夺关税自主本质上是经历掠夺。所以在海关问题上,西方列强的政治控制和经济掠夺是同时发生的。

还有一个,也是西工大期末考试比较爱考的一个,是庚款兴学。这里的 “庚款” 指的其实
就是《辛丑条约》签订以后的庚子赔款。1907 年,时任美国总统西奥多·罗斯福宣布将退还部分庚子赔款,
用于资助中国政府选派留学生赴美留学,史称 “庚款兴学”。从 1909 年第一批庚款留美学生至 1929 年,
整整持续 20 年之久,很多留学生成为美国思想的忠实信徒,甚至站在美国立场上为美国侵华行径做辩护。
毛泽东在《“友谊”,还是侵略?》一文中指出,\textbf{“庚款兴学” 的实质是帝国主义的文化侵略}。

总之,这一节的考察方法比较固定,但是出题点比较琐碎和细节,也有可能会考察一些教材上没有
出现过的史实。但总而言之,识别出西方列强到底在侵略我们的什么其实是不难的。

\begin{example}
    (多选)资本—帝国主义列强在对中国实行军事侵略、政治控制、经济掠夺的同时,还对中国进行
    文化渗透,具体包括:
    \begin{tasks}[label={\Alph*. }](2)
        \task 以传教为掩护进行间谍活动
        \task 创办《万国公报》
        \task 炮制了 “中国威胁论”
        \task 允许外国公使常驻北京
    \end{tasks}
\end{example}



% \subsection{军事侵略}
% \subsection{政治控制}
% \subsection{经济掠夺}
% \subsection{文化侵略}

\section{反抗外国武装侵略的斗争}

\subsection{抵御外来侵略的斗争历程}

这一块的内容在考题中多是以一些琐碎的史实作为选项出现。笔者建议各位读者在复习备考过程中
注意有关知识和有关内容的积累。一般来说,在选项中能够明显地看出这个人、或这群人是属于
人民群众还是爱国官兵。

然后在这里需要强调的是,阶级属性,或者说立场,和爱国是不一样的。例如我们即将谈到的
以邓世昌为代表的清朝的爱国将领,他们虽然属于腐朽的封建地主阶级,不是先进生产力和先进
生产关系的代表,但他们不可谓不爱国。
还有我们即将讲到的义和团运动,虽然他们有着笼统的排外主义错误,还存在着迷信、落后的倾向,
属于农民阶级,但他们也不可谓不爱国。
关于这一点,是各位读者需要了解的,所以在设问中
如果说他们是爱国的,这并不错,不能作为错误选项排除。

\subsubsection{人民群众的反侵略斗争}

1841 年 5 月,广州三元里人民的抗英斗争,是中国近代史上中国人民第一次大规模的反侵略武装
斗争,显示了中国人民不甘屈服和敢于斗争的英雄气概。

\subsubsection{爱国官兵的反侵略斗争}

中日甲午战争中,“致远” 舰受伤下沉,管带邓世昌下令向日军 “吉野” 号猛撞,不幸被鱼雷击中,
船体炸裂沉没,全舰 250 名官兵壮烈牺牲。

\subsection{义和团运动与列强瓜分中国图谋的破产}

\begin{mdframed}[frametitle={义和团运动失败的原因:爱国切不可学义和团}]
    \begin{enumerate}[label=\textup{\arabic*}${}^\circ$, itemsep=0pt]
        \item 对帝国主义的认识还停留在感性认识的阶段,存在着\textbf{笼统的排外主义错误};
        \item 认识不到帝国主义联合中国封建地主阶级压迫中国人民的实质,
        曾经\textbf{蒙受封建统治者的欺骗};
        \item 由于小生产者的局限性,义和团运动中还存在着\textbf{迷信、落后}的倾向。
    \end{enumerate}
\end{mdframed}

\subsubsection{边疆危机和瓜分危机}

帝国主义列强对中国的争夺和瓜分的图谋,在 1894 年中日甲午战争爆发后达到高潮。
中日《马关条约》签订后,俄国、法国、德国三国干涉还辽,迫使日本放弃了割占辽东半岛的
要求。1898 年至 1899 年,帝国主义列强竞相租借港湾和划分势力范围,掀起了瓜分中国的
狂潮。

\subsubsection{列强瓜分中国图谋的破产}

正是包括义和团在内的中华民族为反抗侵略所进行的前赴后继、视死如归的战斗,才粉碎了帝国主义
列强瓜分和灭亡中国的图谋。
\begin{remark}
    这里需要强调的是,义和团运动在粉碎帝国主义列强瓜分中国的斗争中发挥了重大的历史作用,
    \textbf{并不是说义和团运动粉碎了帝国主义列强瓜分中国的图谋}。所以,当选项中出现
    类似的错误表述的时候,千万不能选。不过一般来说,像这种错误,没复习过的同学不会犯,
    复习得比较全面的同学也不会犯,就是那些复习了但复习的不彻底、不仔细的同学,容易看到
    书上的那一句话然后就错误地把帝国主义列强瓜分中国图谋破产归功于义和团运动了。
\end{remark}

\begin{mdframed}
    \begin{cmt}
        瓜分中国图谋破产的原因:
        帝国主义列强之所以没有能够实现瓜分中国的图谋,
        一个重要的原因是\textbf{帝国主义列强之间的矛盾和相互
        制约}。而帝国主义列强不能灭亡和瓜分中国,最根本的原因在
        于\textbf{中华民族的不屈不挠的反侵略斗争}。
    \end{cmt}
\end{mdframed}


\subsection{复习参考题}

\begin{example}
    (多选)资本—帝国主义侵略、压迫中国人民的过程,同时也是中国人民反抗它们的侵略、压迫的过程。
    救亡图存,成了一代又一代中国人面临的神圣使命。其中,爱国官兵反抗外来侵略的斗争有:
    \begin{tasks}[label={\Alph*. }](2)
        \task 广州三元里抗英斗争
        \task 镇南关大捷
        \task 义和团反抗八国联军侵华
        \task 中日甲午海战
    \end{tasks}
\end{example}

\begin{example}
    (多选)1895 年签订的《马关条约》不仅割让台湾岛及其附属岛屿、澎湖列岛和辽东半岛
    给日本,使日本得到巨大的利益,还顺应了帝国主义各国向中国输出资本的愿望,标志着
    中日甲午战争的结束和洋务运动的破产。《马关条约》签订后,逼迫日本归还辽东半岛的
    国家有 \underline{\qquad \qquad \qquad}。
    \begin{tasks}[label={\Alph*. }](4)
        \task 俄国
        \task 德国
        \task 英国
        \task 法国
    \end{tasks}
\end{example}

\section{反侵略战争的失败与民族意识的觉醒}


\begin{mdframed}
    \subsection{反侵略战争的失败及其原因}
    
    \subsubsection{社会制度的腐败(根本原因)}

    正是社会制度的腐败,才使得经济技术落后的状况长期得不到改变。

    \subsubsection{经济技术的落后}
    19 世纪中叶,西方资本主义强国经过工业革命,经济和技术飞速发展,
    中国的封建统治者夜郎自大,闭关锁国,导致中国落后于时代发展步伐。

\end{mdframed}

\subsection{民族意识的觉醒}

% 这又是一个琐碎的考点,建议各位读者在复习备考过程中自行积累。

% \subsubsection{“师夷长技以制夷” 的主张和早期的维新思想}

% 林则徐是近代中国睁眼看世界的第一人。

% 在《海国图志》中,魏源提出了 “师夷长技以制夷” 的思想,主张学习外国先进的军事和科学技术,
% 以期富国强兵,抵御外国侵略。



% \subsubsection{救亡图存和振兴中华}

% 1895 年,严复写了《救亡决论》一文,响亮地喊出了 “救亡” 的口号。在甲午战争后,严复
% 翻译了《天演论》(1898 年正式出版)。他用 “物竞天择”、“适者生存” 的社会进化论思想,
% 为这种危机意识和民族意识提供了理论根据。《天演论》对中国人无疑是振聋发聩的警世钟。

% 绘制于 1898 年的《时局图》,形象地表现了当时中国面临的瓜分危局。
% 关于《时局图》,我们说明一下,根据有关的 Wikipedia 词条,
% 一般的说法认为《时局图》的作者是兴中会会员谢缵泰(1872年-1939年)。
% 其题词作者,则有三种不同说法:一是不知姓名的广东人;二是作者谢缵泰自己;
% 三是晚清政治家黄遵宪(1848年-1905年)。

% 孙中山 1894 年 11 月在创立革命团体兴中会时,喊出了 “振兴中华” 这个时代的最强音。

\begin{table}[H]
    \centering
    \caption{标志着中国人民的民族意识逐渐觉醒的人物、思想或事件}
    \begin{tabular}{p{0.2\textwidth} p{0.3\textwidth} p{0.4\textwidth}}
        \toprule
        人物 & 事件/作品 & 思想/警示 \\ 
        \midrule
        \multirow{2}{*}{林则徐} & 近代中国睁眼看世界的第一人 \\ 
        & 《四洲志》 \\
        魏源 & 《海国图志》 & 师夷长技以制夷 \\
        早期的维新派 & & 不仅要学习西方的科学技术,同时要求也要吸纳西方的政治、经济学说,具有一定程度的反对封建专制的民主思想 \\
        \multirow{2}{*}{严复} & 《救亡决论》 & 救亡 \\ 
        & 《天演论》 & 物竞天择、适者生存 \\
        谢缵泰 & 《时局图》 & 瓜分危局 \\
        孙中山 & 创立革命团体兴中会 & “振兴中华” 的时代最强音 \\
        \bottomrule
    \end{tabular}    
\end{table}

\textbf{中日甲午战争以后,当中华民族面临危机存亡的关头时,中国人才开始有了普
遍的民族意识的觉醒}。

\subsection{复习参考题}

\begin{example}
    (多选)魏源在其所著的《海国图志》一书中提出了 “师夷长技以制夷” 的思想,
    19 世纪 60 年代开始的洋务运动则提出了 “自强” “求富” 的口号。
    二者的相同点包括:
    \begin{tasks}[label={\Alph*. }](2)
        \task 有内在的一致性和继承性
        \task 都有抵御外来侵略的意图
        \task 主要体现了地主阶级的要求
        \task 意识到了中国落后挨打的根本原因
    \end{tasks}
\end{example}

\begin{example}
    1840 年鸦片战争以后,中国遭受西方列强“船坚炮利”的欺凌不断加深,中华民族面临生死存亡的形势也日益严峻,
    中国“睡狮”在西方列强的隆隆炮声中逐渐苏醒。促使中国人民的民族意识开始普遍觉醒的重大事件是 \underline{\qquad \qquad \qquad}。
    \begin{tasks}[label={\Alph*. }](2)
        \task 中法战争
        \task 中日甲午战争
        \task 八国联军侵华战争
        \task 日本全面侵华战争
    \end{tasks}
\end{example}

\begin{example}
    (多选)1894 年 7 月丰岛海战后,中日两国相互宣战,战至 1895 年 2 月北洋海军全军覆没,
    中日两国各自改革 30 年后的决战以清政府的惨败而告终,1895 年 4 月清政府被迫签署
    《马关条约》。这场战争对中国产生了极其严重的后果,表现在:
    \begin{tasks}[label={\Alph*. }](1)
        \task 清政府已经彻底沦为 “洋人的朝廷”;
        \task 中断了清政府通过洋务运动向近代化转型的努力;
        \task 中国付出了丧失领土主权的极大代价;
        \task 清政府损失了海军主力
    \end{tasks}
\end{example}

\begin{example}
    王韬在近代首次提出 “变法” 的主张,他在介绍西方国家的 “君主”、“民主”、“君民共主”
    这三种制度时,最早提出废除封建君主专制,建立 “与民众共政事并治天下” 的君主立宪制。
    该思想:
    \begin{tasks}[label={\Alph*. }](2)
        \task 最早提出发展资本主义
        \task 推动了维新变法的兴起
        \task 反思了当时中国近代化的问题
        \task 引入了社会进化论的思想
    \end{tasks}
    \begin{sol}
        19 世纪 70 年代以后,王韬、薛福成、马建忠、郑观应等人不仅主张学习西方的科学技术,
        同时也要求吸纳西方的政治、经济学说。它们的共同特点,就是具有比较强烈的反对外国侵略、
        追求中国独立富强的爱国思想,以及具有一定程度反对封建专制的民主思想。
        这些主张具有重要的思想启蒙的意义,我们现在称他们为 “早期的维新派”。
    \end{sol}
\end{example}

\begin{mdframed}[frametitle={救亡图存的各种尝试}]
    \begin{table}[H]
        \centering
        \begin{tabular}{c c c c}
            \toprule
            人物/历史事件 & 阶级属性 & 学习军事、科学、技术 & 学习思想、制度、文化 \\
            \midrule
            魏源、林则徐等人 & 封建地主阶级 & T & F \\
            洋务派/洋务运动 & 封建地主阶级 & T & F \\
            王韬/早期维新派 & 封建地主阶级 & T & T \\
            \bottomrule
        \end{tabular}        
    \end{table}
\end{mdframed}

% \subsubsection{中国历代历次反侵略战争失败的原因和教训是什么?}

% 中国近代历次反侵略战争失败的原因从中国内部因素来分析,主要有两个方面,
% 即\textbf{社会制度的腐败}和\textbf{经济技术的落后},而社会制度的腐败是更根本的原因。
% 正是由于社会制度的腐败,经济技术落后的状况才长期得不到改变。

% 中国近代历次反侵略战争失败的教训是:
% (1) 社会制度的腐败导致反侵略战争失败,落后就要挨打;
% (2) 必须建立先进的社会制度,要进行制度创新;
% (3) 必须大力发展先进生产力、富国强兵等。


% \subsubsection{民族意识的觉醒}

% 近代中国睁眼看世界的第一人是\textbf{林则徐}。
% 中国人民的民族意识开始普遍觉醒是在\textbf{中日甲午战争以后},当中华民族面临生死存亡的关头时。




% \newpage
% \section{拓展提高}

% \begin{cmt}
%     这一章中出现了许多琐碎的历史人物。在此总结如下:
%     \begin{enumerate}[itemsep=0pt]
%         \item 王韬,曾参与洋务运动,该经历及其对洋务运动的认识使他深刻体会到洋务运动的不足,
%         从而萌发出学习西方政治制度的思想主张。在近代首次提出 “变法” 的主张,在介绍西方国家
%         的 “民主”、“君主”、“君民共主” 这三种制度时,最早提出废除封建君主制,建立 “与民众共
%         政事并治天下” 的君主立宪制,该思想反思了当时中国近代化的问题。
%         \item 洪仁玕,近代中国最早提出发展资本主义主张。
%         \item 康有为、梁启超的维新思想推动了维新变法运动的兴起。
%         \item 西方社会进化论思想传入中国始于严复。
%     \end{enumerate}
% \end{cmt}

\begin{thebibliography}{1}
    \addcontentsline{toc}{section}{参考文献}
    \bibitem{2023版教材}
    《马克思主义基本原理 (2023 版)》编写组. 马克思主义基本原理: 2023 版[M].
    北京: 高等教育出版社, 2023.
    \bibitem{核心考案}
    徐涛主编, 考研政治核心考案[M], 北京:中国政法大学出版社, 2023.1
    \bibitem{优题库}
    徐涛主编, 考研政治通关优题库[M], 北京:中国政法大学出版社, 2023.2
    \bibitem{真题库}
    徐涛主编, 考研政治必刷真题库[M], 北京:中国政法大学出版社, 2023.2
    \bibitem{冲刺背诵笔记}
    徐涛主编, 考研政治冲刺背诵笔记[M], 北京:中国政法大学出版社, 2023.9
    \bibitem{预测}
    徐涛主编, 考研政治预测 6 套卷[M], 北京:中国政法大学出版社, 2023.10
    \bibitem{预测}
    徐涛, 曲艺主编, 考研政治考前预测必备 20 题[M], 北京:中国政法大学出版社, 2023.11
\end{thebibliography}

\end{document}