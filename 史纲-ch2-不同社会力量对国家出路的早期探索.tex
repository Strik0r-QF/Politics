\documentclass[utf-8, 10pt]{article}

\usepackage{ctex}

\usepackage[
    paperwidth=210mm,
    paperheight=297mm,
    top=31.8mm,
    bottom=31.8mm,
    left=25.4mm,
    right=25.4mm,
    footskip=15mm % 通过这里的值来调整页脚与正文内容的垂直距离
]{geometry}

\usepackage{titlesec} % 定义标题样式

% 定义 section 标题格式
\titleformat{\section}[hang]{\heiti\centering\large\bfseries}{\thesection}{1em}{}

% 定义 subsection 标题格式
\titleformat{\subsection}[hang]{\heiti\bfseries}{\textbf{\thesubsection}}{1em}{}

% 定义 subsubsection 标题格式
\titleformat{\subsubsection}[hang]{\kaishu}{\quad\quad\thesubsubsection\,\,}{0em}{}

\usepackage{mdframed}
\mdfsetup{
  linewidth=0.4pt,
  frametitlebackgroundcolor=white, % 或者 transparent
  frametitlefont=\heiti\bfseries,
  frametitleaboveskip=10pt,
  frametitlebelowskip=5pt,
  frametitlealignment=\raggedright % 新增此行
}
\usepackage{fontspec}
% 设置 Menlo 字体
\setmonofont{Menlo}
\usepackage{fancyvrb}
\usepackage{xcolor}
\usepackage{listings}

% \definecolor{string}{HTML}{067D17}
% \definecolor{comment}{HTML}{8C8C8C}
% \definecolor{keyword}{HTML}{0033B3}
% \definecolor{class_field}{HTML}{871094}

\lstset{breaklines}
%这条命令可以让LaTeX自动将长的代码行换行排版
\lstset{extendedchars=false}
%这一条命令可以解决代码跨页时,章节标题,页眉等汉字不显示的问题
\lstset{escapeinside={(*}{*)}}

\lstset{
    basicstyle=\small\ttfamily\heiti,
    numbers=left,
    numberstyle=\scriptsize\fontspec{Menlo}, % 使用 Menlo 字体
    stepnumber=1,
    numbersep=8pt,
    frame=leftline,
    xleftmargin=2em, % 调整代码块的左边界
    framexleftmargin=0pt, % 调整边框的位置
    breaklines=true,
    % postbreak=\mbox{\textcolor{red}{$\hookrightarrow$}\space},
    % keywordstyle=\bfseries\color{keyword},          % keyword style
    % commentstyle=\heiti\color{comment},       % comment style
    % stringstyle=\color[HTML]{067D17},
    showstringspaces=false,
    % string literal style
    % escapeinside={\%*}{*)},            % if you want to add LaTeX within your code
    % morekeywords={}               % if you want to add more keywords to the set
}

\usepackage{fancyhdr} % 用于自定义页眉页脚


% 设置页眉页脚样式
\fancypagestyle{plain}{%
    \fancyhf{} % 清空页眉页脚
    \fancyhead[L]{·\thepage·} % 页眉显示页码, RO表示奇数页右侧, LE表示偶数页左侧
    \fancyhead[R]{中国近现代史纲要}
    \renewcommand{\headrulewidth}{0.4pt} % 设置页眉横线的宽度
    \renewcommand{\footrulewidth}{0pt} % 取消页脚横线
}

\renewcommand{\headrule}{\hrule width\textwidth height\headrulewidth\vskip-\headrulewidth}

% 定义取消页眉的命令
\newcommand{\cancelheader}{%
    \fancyhead{} % 清空页眉
    \renewcommand{\headrulewidth}{0pt} % 取消页眉横线
    \renewcommand{\footrulewidth}{0pt} % 设置页脚横线的宽度
}

\usepackage{caption, subcaption}
\usepackage{longtable, diagbox, booktabs}
\usepackage{float, graphicx}
\usepackage{amsthm, amssymb, amsmath, mathrsfs, mhchem, siunitx, pgfplots}
\usepackage{tikz, circuitikz, tikz-cd, tikz-3dplot}
\usetikzlibrary{decorations.markings, angles, quotes}
\usepackage{tasks, enumitem}
\usepackage{hyperref}
\hypersetup{hidelinks,
    colorlinks = true,
    allcolors = black,
    pdfstartview = Fit,
    breaklinks = true}
\usepackage[toc]{multitoc}
\usepackage{abstract}
\usepackage{extpfeil}
\usepackage{xcolor}
\usepackage{multirow}

\everymath{\displaystyle}

\begin{document}
\newtheoremstyle{mytheoremstyle}
    {1.5ex}                                         % Space above
    {1.5ex}                                         % Space below
    {}                                              % Font for body
    {}                                              % Indent amount
    {\bfseries}                                     % Font for head
    {}                                              % Punctuation after head
    {0.5em plus 0.2em minus 0.1em}                  % Space after head
    {\thmname{#1}\thmnumber{ #2}.\thmnote{ (#3).}}

\theoremstyle{mytheoremstyle}
\newtheorem{definition}{定义}[section]
\newtheorem{example}{例}[section]
\newtheorem{exercise}{习题}[section]
\newtheorem{code}{程序清单}[section]
\newtheorem*{result}{运行结果}

\newtheoremstyle{my2theoremstyle}
    {1.5ex}                                         % Space above
    {1.5ex}                                         % Space below
    {\kaishu}                                              % Font for body
    {}                                              % Indent amount
    {\bfseries}                                     % Font for head
    {}                                              % Punctuation after head
    {0.5em plus 0.2em minus 0.1em}                  % Space after head
    {\thmname{#1}\thmnumber{ #2}.\thmnote{ (#3).}}

\theoremstyle{my2theoremstyle}
\newtheorem{thm}{定理}[section]
\newtheorem{law}{定律}[section]
\newtheorem{educt}{推论}
\newtheorem{prop}{命题}
\newtheorem{lemma}{引理}
\newtheorem{axiom}{公理}
\newtheorem{property}{性质}

\newtheoremstyle{my4theoremstyle}
    {1.5ex}                                         % Space above
    {1.5ex}                                         % Space below
    {}                                              % Font for body
    {}                                              % Indent amount
    {\bfseries}                                     % Font for head
    {}                                              % Punctuation after head
    {0.5em plus 0.2em minus 0.1em}                  % Space after head
    {\thmname{#1}.}

\theoremstyle{my4theoremstyle} \newtheorem*{sol}{解}

\newtheoremstyle{my3theoremstyle}
    {1.5ex}                                         % Space above
    {1.5ex}                                         % Space below
    {}                                              % Font for body
    {}                                              % Indent amount
    {\kaishu}                                       % Font for head
    {}                                              % Punctuation after head
    {0.5em plus 0.2em minus 0.1em}                  % Space after head
    {\thmname{#1}\thmnumber{ #2}.\thmnote{ (#3).}}

\theoremstyle{my3theoremstyle} \newtheorem*{remark}{注}
\newtheorem*{cmt}{评注}
% \pagestyle{plain}
\title{不同社会力量对国家出路的早期探索}
\author{钱锋\thanks{电子邮件: strik0r.qf@gmail.com
\newline \indent 西北工业大学软件学院, School of Software, Northwestern Polytechnical University, 西安 710072
}}

\maketitle
\thispagestyle{empty}
% \begin{abstract}
    
%     \begin{keywords}

%     \end{keywords}
% \end{abstract}
{\small \tableofcontents}

这是一段近代中国仁人志士扛起救亡图存、振兴中华的爱国主义大旗,历经艰辛,
多方求索,挽救中华民族危亡的历史。
太平天国运动、洋务运动和维新运动的兴起和失败,说明了农民阶级、地主阶级洋务
派和资产阶级维新派都不可能是中国真正实现民族独立和国家富强,深刻地揭示了
无产阶级领导中国革命走向胜利的必然性。
本章的章节结构如下:
\begin{enumerate}[label=2.\arabic*,itemsep=0pt]
    \item \textbf{太平天国运动的起落}:介绍太平天国农民战争爆发的原因、经过,
    和这次农民起义的发展和失败,并分析了这次农民战争失败的原因,说明了单纯的农民战争
    不可能完成争取民族独立和人民解放的历史任务;
    \item \textbf{洋务运动的兴衰}:本节介绍洋务运动的兴办和失败,并分析了洋务运动失败
    的原因,说明了以洋务派为代表的封建的地主阶级不可能为中国摆脱贫弱找到出路,
    也不可能避免最终失败的命运;
    \item \textbf{戊戌维新运动的兴起和夭折};
\end{enumerate}

从实现近现代中国历史任务的角度,阐明农民阶级、地主阶级、资产阶级\textbf{对国家出路进行早期探索的历史价值及其内在关联性}是本章学习的重点;
运用马克思主义立场和观点,\textbf{深入剖析各个阶级的特点及其局限性},对其发动的政治运动做出客观评价是本章学习的难点。

\section{太平天国运动的起落}

\subsection{太平天国农民战争}

\subsubsection{金田起义和天平天国的建立}

太平天国农民起义爆发的原因: (1) 清政府加重了赋税的征收科派,农民的经济负担更为沉重;
(2) 由于西方资本主义的入侵,中国的农业和家庭手工业相结合的自然经济逐渐解体。

太平天国运动的性质:太平军所进行的战争,是一次反对清政府腐朽统治和地主阶级压迫、剥削的正义战争。

\subsubsection{《天朝田亩制度》和《资政新篇》}

\textbf{《天朝田亩制度》}是最能体现太平天国社会理想和这次农民起义特色的\textbf{纲领性文件}。
\begin{enumerate}[label=(\arabic*), itemsep=0pt]
    \item 确立了平均分配土地的方案,实际上是起义农民提出的一个以解决土地问题为中心的比较完整的社会改革
    方案。
    \item《天朝田亩制度》的主张,\textbf{否定了封建社会的基础即封建土地所有制}。
    \item 这部文件的局限性体现在,它并没有超出农民小生产者的狭隘眼界,这种社会理想,
    具有不切实际的空想的性质;
    \begin{remark}
        中国要实现共同富裕,但不是搞平均主义,而是要先把 “蛋糕” 做大,
        然后通过合理的制度安排把 “蛋糕” 分好,水涨船高、各得其所,
        让发展成果更多更公平惠及全体人民。
    \end{remark}
    \item 《天朝田亩制度》中的平分土地方案即使在太平军所占领地区也并未能付诸实施。
\end{enumerate}

\textbf{《资政新篇》}是\textbf{中国近代史上第一个具有资本主义色彩的方案}。
但限于当时的历史条件,未能付诸实施。

\subsubsection{从天京事变到太平天国败亡}

1856 年 9 月, 发生了自相残杀的天京事变, 成为太平天国由盛转衰的分水岭.

\subsection{农民斗争的意义和局限}

\subsubsection{太平天国农民起义的历史意义}

太平天国农民战争虽然失败了,但是其具有重大的历史意义:
\begin{enumerate}[label=(\arabic*), itemsep=0pt]
    \item 沉重打击了封建统治阶级,强烈撼动了清政府的统治根基;
    \item 是中国旧式农民战争的最高峰;
    \item 冲击了孔子和儒家经典的正统权威;
    \item 有力打击了外国侵略势力;
    \item 冲击了西方殖民主义者在亚洲的统治。
    \item 在 19 世纪中叶的亚洲民族解放运动中,太平天国农民战争是其中时间最久、
    规模最大、影响最深的一次。
\end{enumerate}

\begin{table}[H]
    \centering
    \caption{从任务、纲领、结果三个层面认识太平天国}
    \begin{tabular}{c|p{0.4\textwidth}}
        任务层面 & 反帝反封建 \\ 
        纲领层面 & 反封建, 但并没有明确的反帝纲领 \\ 
        结果层面 & 打击了帝国主义
    \end{tabular}
\end{table}

\subsubsection{太平天国农民起义失败的原因和教训}

太平天国农民战争动摇了清王朝封建统治的基础,有力地打击了西方资本主义侵略者,
显示了农民阶级的反抗精神和战争力量,但最终失败了。

\begin{enumerate}[label=(\arabic*), itemsep=0pt]
    \item 太平天国农民战争失败的根本原因是\textbf{农民阶级不是新的生产力和生产关系的代表},
    他们无法克服小生产者所固有的阶级局限性,这种阶级局限性主要表现在:
    \begin{enumerate}[label=\roman*., itemsep=0pt]
        \item 无法从根本上提出完整的、正确的政治纲领和社会改革方案;
        \item 无法制止和克服领导集团自身腐败现象的滋生;
        \item 无法长期保持领导集团的团结;
    \end{enumerate}
    \item 太平天国在军事战略上出现了重大失误;
    \item \textbf{没有科学理论的指导}:太平天国是以宗教来发动、组织群众的,但是,
    拜上帝教教义毕竟不是科学的思想理论,它不仅不能正确指导斗争,而且给农民战争带来了
    危害;
    \item 未能正确地对待儒学;
    \item 太平天国的领袖们虽然不承认不平等条约,但他们不能把西方国家的侵略者与
    人民群众区别开来,而是笼统地把信奉天父上帝的西方人都称为 “洋兄弟”,这说明
    他们对于西方资本主义侵略者还缺乏理性的认识。
\end{enumerate}

% \vspace*{0.5cm}

太平天国农民战争及其失败表明,在半殖民地半封建的中国,农民具有伟大的革命潜力;
但它自身不能担负起领导反帝反封建斗争取得胜利的重任。单纯的农民战争不可能完成争取
民族独立和人民解放的历史任务。

\begin{mdframed}[frametitle={太平天国运动:旧式农民起义的最高峰}]
    \begin{enumerate}[label=\textup{\arabic*}${}^\circ$, itemsep=0pt]
        \item 两个纲领性文件:
        \begin{enumerate}[itemsep=0pt]
            \item 《天朝田亩制度》最能体现太平天国社会理想和这次农民起义特色,
            确立了平均分配土地的方案,否定了封建社会的基础即封建土地所有制,
            但抱有不切实际的平均主义幻想。
            \item 《资政新篇》是\textbf{中国近代史上第一个具有资本主义色彩的方案}。
            但限于当时的历史条件,未能付诸实施。            
        \end{enumerate}
        \item 四点失败的原因:
        \begin{enumerate}[itemsep=0pt]
            \item 阶级局限性:农民阶级不是新的生产力和生产关系的代表;
            \item 军事战略上出现重大失误;
            \item 缺乏科学思想理论的指导;
            \item 未能正确地对待儒学;
            \item 对侵略者的认识不清。
        \end{enumerate}
    \end{enumerate}
\end{mdframed}

\subsection{复习参考题}

\begin{example}
    太平天国定都天京后,进行了一系列制度建设,并颁布了《天朝田亩制度》。该纲领
    将农民平均主义思想进行了制度化。但这篇纲领始终只是空想、幻想,这是因为:
    \begin{enumerate}[label=\Alph*., itemsep=0pt]
    \item 拜上帝教的教义并非科学的指导思想;
    \item 没有从根本上否定封建地主土地所有制;
    \item 提出发展资本主义的方案在中国行不通;
    \item 在分散的小农经济基础上不可能真正实现均贫富
    \end{enumerate}
\end{example}

\section{洋务运动的兴衰}

洋务运动是封建地主阶级引进近代工业,仿造洋枪洋炮,以期富国强兵,延续封建统治而兴起的一场运动。

洋务派兴办洋务事业的原因,首先是为了购买和制造洋枪洋炮以镇压农民起义,同时也有借此加强
海防、边防,并趁机发展本集团的政治、经济、军事实力的意图。

洋务运动的指导思想是\textbf{“中学为体,西学为用”},
就是仍然以中国封建伦理纲常所维护的统治秩序为主体,辅之以西方的近代工业和技术,
所以,\textbf{兴办洋务新政的主要目的是维护封建统治}。

\subsection{洋务事业的兴办}

\subsubsection{兴办近代企业}

% 2023 版教材 p56

洋务派首先兴办的是军用工业。这些军用工业的名称、创办的时间、筹办的官员和一些附加的
参考信息如下表所示,大家混个眼熟即可。

\begin{table}[H]
    \centering
    \captionof{table}{洋务派兴办的规模较大的军工企业}
    \begin{tabular}{c c c c c}
        \toprule
        名称 & 时间 & 官员 & 备注 \\
        \midrule
        安庆军械所 & 1861 & & 最早创办 \\
        上海江南制造总局 & 1865  & 曾国藩支持、李鸿章筹办 & 当时国内最大的兵工厂 \\ 
        金陵机器局(南京) & 1865  & 李鸿章 &  \\ 
        福州船政局 & 1866  & 左宗棠  \\ 
        天津机器局 & 1867  & 崇厚  \\
        湖北枪炮厂 & 1890 & 张之洞 \\   
        \bottomrule  
    \end{tabular}
\end{table}

洋务派在创办军事工业中遇到资金奇缺、原料和燃料供应不足,以及交通运输等困难,
因此还兴办了一些民用企业,这些企业除少数采取官办或官商合办的方式外,
多数都采取官督商办的方式。\textbf{这些官督商办的民用企业,虽然受官僚的控制,
发展受到很大限制,但基本上是资本主义性质的近代企业}。

\subsubsection{建立新式海陆军}
\subsubsection{创办新式学堂、派遣留学生}

新式学堂主要有三种:翻译学堂、工艺学堂、军事学堂。此外,还先后派遣赴美留学幼童
及官费赴欧留学生 200 多人。

\subsection{洋务运动的历史作用及失败}

\subsubsection{洋务运动的历史作用}

\begin{enumerate}[label=(\arabic*), itemsep=0pt]
    \item 洋务派发展近代军事工业和民用工业,在客观上对中国早期工业和民族资本主义的发展起到了某些
    促进作用;
    \item 洋务派兴办新式学堂,是中国近代教育的开始,还翻译了一批自然科学书籍,
    给当时的中国带来了新的知识,使人开阔了眼界;
    \item 社会风气和价值观念开始变化,工商业者的地位上升,这一切,都有利于资本主义经济的发展,
    也有利于社会风气的转变。
\end{enumerate}

\subsubsection{洋务运动失败的原因}

中日甲午战争一役,洋务派经营多年的北洋海军全军覆没,标志着以 “自强”、“求富” 为目标的洋务
运动的失败。洋务运动失败的原因如下:
\begin{enumerate}[label=(\arabic*), itemsep=0pt]
    \item \textbf{具有封建性};
    \item \textbf{对列强具有依赖性};
    \item \textbf{管理具有腐朽性}。
\end{enumerate}

\begin{mdframed}[frametitle={洋务运动:一次封建统治阶级的自救运动}]
    \begin{enumerate}[label=\textup{\arabic*}${}^\circ$, itemsep=0pt]
        \item 指导思想:中学为体、西学为用;
        \item 失败的原因:
            \begin{enumerate}[itemsep=0pt]
                \item 阶级局限性:\textbf{具有封建性};
                \item \textbf{对列强具有依赖性};
                \item \textbf{管理具有腐朽性}。
            \end{enumerate}
    \end{enumerate}
\end{mdframed}

\subsection{复习参考题}

\begin{example}
    第二次鸦片战争后,清朝统治集团内部一部分人震惊于列强的 “船坚炮利”,主张学习西方以求
    “自强”,洋务运动由此兴起,在洋务派兴办的洋务事业中,当时国内最大的兵工厂是 
    \underline{\qquad \qquad \qquad}。
    \newline
    \begin{tabular}{p{0.48\textwidth} p{0.45\textwidth}}
    A. 安庆内军械所; & B. 天津机器局; \\
    C. 福州船政局; & D. 上海江南制造总局 
    \end{tabular}
\end{example}

\begin{example}[2018 年,第 10 题]
    洋务运动时期,洋务派兴办了一些民用企业,这些企业除少数采取官办或官商合办的方式外,
    多数都采取官督商办的方式。这些官督商办的民用企业基本上是 \underline{\qquad \qquad \qquad}。
    \begin{tabular}{p{0.48\textwidth} p{0.45\textwidth}}
    A. 封建主义性质的企业; & B. 半封建性质的企业;  \\
    C. 资本主义性质的企业; & D. 资本主义性质的企业 
    \end{tabular}
    \begin{sol}
        这些官督商办的民用企业,虽然受官僚的控制,
        发展受到很大限制,但基本上是资本主义性质的近代企业。因此本题选 C。
    \end{sol}
\end{example}

\section{维新运动的兴起和夭折}

\subsection{戊戌维新运动的开展}

在内忧外患的冲击和中西文化的碰撞过程中,人们形成了一个共识:要救国,只有维新,
要维新,只有学外国。

\subsubsection{维新派倡导救亡和变法的活动}

\begin{enumerate}[label=(\arabic*), itemsep=0pt]
    \item 向皇帝上书:如著名的 “公车上书”。
    \item 著书立说:如《新学伪经考》、《孔子改制考》等。
    \item 介绍外国变法的经验教训。
    \item 办学会,如强学会、南学会、保国会等;
    \item 设学堂,如广州万木草堂、长沙时务学堂等;
    \item 办报纸,如《时务报》、《国闻报》、《湘报》等。
\end{enumerate}
维新派以各种方式宣传变法主张,制造维新舆论,培养变法骨干,组织革新力量,而重点则放在
争取光绪皇帝及其周围的帝党官员的支持上,希望通过他们\textbf{自上而下}地实行变法主张。

\subsubsection{维新派与守旧派的论战}

论战主要围绕以下三个问题展开:
\begin{enumerate}[label=(\arabic*), itemsep=0pt]
    \item 要不要变法;
    \item 要不要兴民权、设议院,实行君主立宪;
    \item 要不要废八股、改科举和兴西学。
\end{enumerate}
维新派和守旧派的这场论战,实质上是资产阶级思想与封建主义思想在中国的第一次正面交锋。
进一步开阔了新型知识分子的眼界,解放了人们长期受到束缚的思想,西方资产阶级社会
政治学说在中国得到进一步的传播,戊戌变法运动的帷幕随之拉开。




\begin{mdframed}[frametitle={救亡图存的各种尝试}]
    \begin{table}[H]
        \centering
        \begin{tabular}{c c c c}
            \toprule
            人物/历史事件 & 阶级属性 & 学习军事、科学、技术 & 学习思想、制度、文化 \\
            \midrule
            魏源、林则徐等人 & 封建地主阶级 & T & F \\
            洋务派/洋务运动 & 封建地主阶级 & T & F \\
            王韬/早期维新派 & 封建地主阶级 & T & T \\
            维新派/戊戌维新 & 资产阶级维新派 & 君主立宪制 & T \\
            辛亥革命 & 资产阶级革命派 & 民主共和制 & T \\
            \bottomrule
        \end{tabular}        
    \end{table}
\end{mdframed}

\subsubsection{昙花一现的百日维新}


“百日维新” 在政治方面的内容是:
\begin{enumerate}[label=(\arabic*), itemsep=0pt]
    \item 改革行政机构,裁撤闲散、重叠机构;
    \item 裁汰冗员,澄清吏治,提倡廉政;
    \item 提倡向皇帝上书言事;
    \item 准许旗人自谋生计,取消他们享受国家供养的特权。
\end{enumerate}
“百日维新” 期间颁布的政令使资产阶级享受一定程度的政治权利,促进了资本主义工商业
的发展。因此,戊戌维新运动是一场\textbf{资产阶级性质的改良运动}。但是,
在光绪皇帝发布的新政诏令中,并没有采纳维新派多次提出的开国会、定宪法等政治主张。
这些政令和措施\textbf{并未触及封建制度的根本},所要推行的是一种十分温和的不彻底的改良方案。

1898 年 9 月 21 日,“戊戌六君子” 惨遭杀害。“百日维新” 如昙花一现,只经历了
103 天就夭折了。除京师大学堂(今北京大学的前身)被保留下来以外,其余新政措施大多被废除。

\subsection{戊戌维新运动的意义和教训}

\subsubsection{戊戌维新运动的意义}

\begin{enumerate}[label=(\arabic*), itemsep=0pt]
    \item 戊戌维新运动是一次\textbf{爱国救亡运动};
    \item 戊戌维新运动是一场\textbf{资产阶级性质的政治改良运动};
    \item 戊戌维新运动是一场\textbf{思想启蒙运动}。
\end{enumerate}

\subsubsection{戊戌维新运动失败的原因和教训}

戊戌维新运动失败的原因主要是维新派自身的局限性(阶级局限性+客观原因)
和以慈禧太后为首的强大的守旧势力的反对。具体如下:
\begin{enumerate}[label=(\textup{\arabic*}), itemsep=0pt]
    \item 民族资产阶级力量弱小(客观原因);
    \item 维新派的局限性突出表现为以下三个方面:
    \begin{enumerate}[label=\roman*., itemsep=0pt]
         \item \textbf{不敢否定封建主义}:
            \begin{itemize}[itemsep=0pt]
                \item 政治上,不敢根本否定封建君主制度;
                \item 经济上,虽然要求发展民族资本主义,却未
                触及封建主义的经济基础——封建土地所有制;
                \item 思想上,虽然提倡学习西学,却仍要打折孔子的旗号,接古代圣贤之名
                “托古改制”。
            \end{itemize}
         \item \textbf{对帝国主义抱有幻想};
         \item \textbf{惧怕人民群众}:戊戌维新运动是一场走精英路线的运动。
    \end{enumerate}
\end{enumerate}
戊戌维新运动的失败说明了在半殖民地半封建的旧中国,企图通过统治者走自上而下的改良道路,
实现国家的独立、民主、富强是根本行不通的,
必须用革命的手段,推翻帝国主义、封建主义联合统治的半殖民地半封建的社会制度。
“戊戌六君子”血的教训,促使一部分人放弃了改良主张,开始走上了革命的道路。

\begin{mdframed}[frametitle={戊戌维新:中国民族资产阶级的第一舞}]
    \begin{enumerate}[label=\textup{\arabic*}${}^\circ$, itemsep=0pt]
        \item 历史意义:
        戊戌维新运动是一次爱国救亡运动、资产阶级性质的政治改良运动、思想启蒙运动;
        \item 资产阶级维新派的阶级局限性是戊戌维新运动失败的根本原因,其主要表现在:
        \begin{enumerate}[label=\roman*., itemsep=0pt]
            \item \textbf{不敢否定封建主义}:
            \item \textbf{对帝国主义抱有幻想};
            \item \textbf{惧怕人民群众}:戊戌维新运动是一场走精英路线的运动。
       \end{enumerate}
    \end{enumerate}
\end{mdframed}

\subsection{复习参考题}

\begin{example}
    中国民族资产阶级登上政治舞台的第一次表演是 \underline{\qquad \qquad \qquad}。
    \newline
    \begin{tabular}{p{0.48\textwidth} p{0.45\textwidth}}
    A. 洋务运动; & B. 戊戌维新运动; \\
    C. 辛亥革命; & D. 五四运动 
    \end{tabular}
\end{example}

\begin{example}
    (多选)在向西方学习的过程中,戊戌维新运动与洋务运动的不同点在于 \underline{\qquad \qquad \qquad}。
    \newline
    \begin{tabular}{p{0.48\textwidth} p{0.45\textwidth}}
        A. 学习西方的科学技术;& B. 学习西方的政治制度; \\ 
        C. 批判封建的伦理道德;& D. 主张采取君主立宪制
    \end{tabular}
\end{example}

% \section{拓展提高:晚清通史}

% \subsection{清代通商与外政制度}

% \subsubsection{通商与外政制度的概念意义}

% 首先,什么是外政制度?我们在此将\textbf{清朝固有的对外关系制度称为外政制度}。原因有二:
% 其一,清朝在成立外务部以前,并没有西方近代式的专责外交机构;其二,清朝与西方近代民族国家
% (nation-state)在构造上截然不同,因而无法透过西方近代式的政府外交机构或组织的视角来理解
% 清朝中国在对外关系上的构造、制度以及思想概念的运作情形。所以为了区别清朝固有的对外关系制度和
% 西方近代的外交制度的不同,我们将清朝固有对外关系制度称为外政制度。



% \newpage
% \subsection{十九世纪前期中西关系的演变}
% \subsection{近代的开端:鸦片战争}
% \subsection{条约制度的建立及其影响}
% \subsection{中华宗藩体系的挫败与转型}
% \subsection{太平天国的兴起与败亡}
% \subsection{洋务运动与早期现代化}
% \subsection{派系分合与晚清政局}
% \subsection{从甲午战争到戊戌变法}
% \subsection{义和团运动与二十世纪中国}
% \subsection{十年新政与清朝覆灭}

% \section{拓展提高:晚清专题史}

% \subsection{立宪运动与民间宪政诉求}
% \subsection{帝制面临的挑战:新政的制度困境和伦理转换}
% \subsection{现代经济的起步:晚清的经济发展}
% \subsection{悸动的农村与农民}
% \subsection{二十世纪初的收回利权运动}
% \subsection{清季人口与社会}
% \subsection{大变局下的生活世界:洋货流行与生活启蒙}
% \subsection{晚清士绅阶层的结构性变动}
% \subsection{中西学之争:从科举、学校到学堂}
% \subsection{晚清海防与塞防之争}
% \subsection{“过渡时代” 的脉动:晚清思想发展之轨迹}
% \subsection{天下、国家与价值重构:启蒙的历程}
% \subsection{族群、文化与国家:晚清的国族想象}
% \subsection{中国士人与西方政体类型知识 “概念工程” 的创造与转化}
% \subsection{译书与西学东渐}
% \subsection{晚清台湾的社会经济与文化发展}

\begin{thebibliography}{1}
    \addcontentsline{toc}{section}{参考文献}
    \bibitem{2023版教材}
    《马克思主义基本原理 (2023 版)》编写组. 马克思主义基本原理: 2023 版[M].
    北京: 高等教育出版社, 2023.
    \bibitem{核心考案}
    徐涛主编, 考研政治核心考案[M], 北京:中国政法大学出版社, 2023.1
    \bibitem{优题库}
    徐涛主编, 考研政治通关优题库[M], 北京:中国政法大学出版社, 2023.2
    \bibitem{真题库}
    徐涛主编, 考研政治必刷真题库[M], 北京:中国政法大学出版社, 2023.2
    \bibitem{冲刺背诵笔记}
    徐涛主编, 考研政治冲刺背诵笔记[M], 北京:中国政法大学出版社, 2023.9
    \bibitem{预测}
    徐涛主编, 考研政治预测 6 套卷[M], 北京:中国政法大学出版社, 2023.10
    \bibitem{预测}
    徐涛, 曲艺主编, 考研政治考前预测必备 20 题[M], 北京:中国政法大学出版社, 2023.11
\end{thebibliography}

\end{document}