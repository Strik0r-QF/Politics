\documentclass[10pt, UTF8]{book} %% ctexart

\title{\textbf{中国近现代史纲要}

学习参考与复习指导}
\author{钱锋\thanks{Email: strik0r.qf@gmail.com}${}^,$\thanks{
    西北工业大学软件学院, School of Software, Northwestern Polytechnical University, 西安 710072
}}

\usepackage{ctex}
\usepackage{graphicx}
\usepackage[toc]{multitoc} % 加载multitoc宏包
\usepackage{booktabs}
\usepackage{longtable}
\usepackage{amsthm, amssymb, amsmath, mathrsfs, mhchem}
\usepackage{tikz}
\usetikzlibrary{decorations.markings, angles, quotes}
\usepackage{pgfplots}
\usepackage{tikz-3dplot}
\usepackage{extpfeil}
\usepackage{diagbox}
\usepackage{multirow}
\usepackage{float}
\usepackage{hyperref}
\hypersetup{hidelinks,
    colorlinks = true,
    allcolors = black,
    pdfstartview = Fit,
    breaklinks = true}
\usepackage{caption}
% \captionsetup[table]{labelsep=space} % 表
% \captionsetup[figure]{labelsep=space} % 图
\usepackage{enumitem}
\usepackage{siunitx}
\usepackage{cuted}

\usepackage{fancyhdr} % 用于自定义页眉页脚


% 设置页眉页脚样式
\fancypagestyle{plain}{%
    \fancyhf{} % 清空页眉页脚
    \fancyhead[RO,LE]{·\thepage·} % 页眉显示页码, RO表示奇数页右侧, LE表示偶数页左侧
    \fancyhead[LO]{\nouppercase{\rightmark}} % 页眉显示小节标题, LO表示奇数页左侧
    \fancyhead[RE]{\nouppercase{\leftmark}} % 页眉显示章节标题, RE表示偶数页右侧
    \renewcommand{\headrulewidth}{0.4pt} % 设置页眉横线的宽度
    \renewcommand{\footrulewidth}{0pt} % 取消页脚横线
}

\renewcommand{\headrule}{\hrule width\textwidth height\headrulewidth\vskip-\headrulewidth}

% % 取消奇偶页的页眉偏移
% \fancyhfoffset[RO,LE]{0pt}

% % 取消奇偶页的页眉偏移
% \fancyhfoffset[RO,LE]{0pt}

% 定义取消页眉的命令
\newcommand{\cancelheader}{%
    \fancyhead{} % 清空页眉
    \renewcommand{\headrulewidth}{0pt} % 取消页眉横线
    \renewcommand{\footrulewidth}{0pt} % 设置页脚横线的宽度
}

\renewcommand{\chaptermark}[1]{\markboth{第 \thechapter 章 \hspace{1em} #1}{}}
\renewcommand{\sectionmark}[1]{\markright{\thesection \, #1}}
\usepackage{titlesec} % 定义标题样式

% 设置 chapter 标题样式
\titleformat{\chapter}[hang]{\centering\heiti\Large\bfseries}{第\,\thechapter\,章}{1em}{}

% 定义 section 标题格式
\titleformat{\section}[hang]{\heiti\centering\large\bfseries}{\thesection}{1em}{}

% 定义 subsection 标题格式
\titleformat{\subsection}[hang]{\heiti\bfseries}{\textbf{\thesubsection}}{1em}{}

% 定义 subsubsection 标题格式
\setcounter{secnumdepth}{3}
\renewcommand\thesubsubsection{\arabic{subsubsection}.}
\titleformat{\subsubsection}[hang]{\kaishu}{\quad\quad\thesubsubsection\,\,}{0em}{}

% % 重新定义 textbf
% \let\oldtextbf\textbf
% \renewcommand{\textbf}[1]{{\heiti\oldtextbf{#1}}}

% % 在导言区重新定义 \normalsize 命令
% \makeatletter
% \renewcommand\normalsize{%
%    \@setfontsize\normalsize{10.5pt}{12pt}%
%    \abovedisplayskip 8\p@ \@plus2\p@ \@minus5\p@
%    \abovedisplayshortskip \z@ \@plus3\p@
%    \belowdisplayshortskip 6\p@ \@plus3\p@ \@minus3\p@
%    \belowdisplayskip \abovedisplayskip
%    \let\@listi\@listI}
% \makeatother



% 设置页边距和对齐
% \usepackage[
%     paperwidth=185mm,
%     paperheight=260mm,
%     top=35mm,
%     bottom=25mm,
%     left=18mm,
%     right=18mm,
%     footskip=15mm % 通过这里的值来调整页脚与正文内容的垂直距离
% ]{geometry}

\usepackage[
    paperwidth=210mm,
    paperheight=297mm,
    top=40mm,
    bottom=31.8mm,
    left=25.4mm,
    right=25.4mm,
    footskip=15mm % 通过这里的值来调整页脚与正文内容的垂直距离
]{geometry}

% \usepackage[
%     paperwidth=195mm,
%     paperheight=270mm,
%     top=40mm,
%     bottom=25mm,
%     left=23.5mm,
%     right=23.5mm,
%     footskip=15mm % 通过这里的值来调整页脚与正文内容的垂直距离
% ]{geometry}
\usepackage{mdframed}
\mdfsetup{
  linewidth=0.4pt,
  frametitlebackgroundcolor=white, % 或者 transparent
  frametitlefont=\heiti\bfseries,
  frametitleaboveskip=10pt,
  frametitlebelowskip=5pt,
  frametitlealignment=\raggedright % 新增此行
}
\usepackage{fontspec}
% 设置 Menlo 字体
\setmonofont{Menlo}
\usepackage{fancyvrb}
\usepackage{xcolor}
\usepackage{listings}

% \definecolor{string}{HTML}{067D17}
% \definecolor{comment}{HTML}{8C8C8C}
% \definecolor{keyword}{HTML}{0033B3}
% \definecolor{class_field}{HTML}{871094}

\lstset{breaklines}
%这条命令可以让LaTeX自动将长的代码行换行排版
\lstset{extendedchars=false}
%这一条命令可以解决代码跨页时,章节标题,页眉等汉字不显示的问题
\lstset{escapeinside={(*}{*)}}

\lstset{
    basicstyle=\small\ttfamily\heiti,
    numbers=left,
    numberstyle=\scriptsize\fontspec{Menlo}, % 使用 Menlo 字体
    stepnumber=1,
    numbersep=8pt,
    frame=leftline,
    xleftmargin=2em, % 调整代码块的左边界
    framexleftmargin=0pt, % 调整边框的位置
    breaklines=true,
    % postbreak=\mbox{\textcolor{red}{$\hookrightarrow$}\space},
    % keywordstyle=\bfseries\color{keyword},          % keyword style
    % commentstyle=\heiti\color{comment},       % comment style
    % stringstyle=\color[HTML]{067D17},
    showstringspaces=false,
    % string literal style
    % escapeinside={\%*}{*)},            % if you want to add LaTeX within your code
    % morekeywords={}               % if you want to add more keywords to the set
}

\usepackage{smartdiagram}
\usepackage{tasks}

\begin{document}
\newtheoremstyle{mytheoremstyle}
    {1.5ex}                                         % Space above
    {1.5ex}                                         % Space below
    {}                                              % Font for body
    {}                                              % Indent amount
    {\bfseries}                                     % Font for head
    {}                                              % Punctuation after head
    {0.5em plus 0.2em minus 0.1em}                  % Space after head
    {\thmname{#1}\thmnumber{ #2}.\thmnote{ (#3).}}

\theoremstyle{mytheoremstyle}
\newtheorem{definition}{定义}[section]
\newtheorem{example}{例}[section]
\newtheorem{exercise}{习题}[section]
\newtheorem{code}{程序清单}[section]
\newtheorem*{result}{运行结果}

\newtheoremstyle{my2theoremstyle}
    {1.5ex}                                         % Space above
    {1.5ex}                                         % Space below
    {\kaishu}                                              % Font for body
    {}                                              % Indent amount
    {\bfseries}                                     % Font for head
    {}                                              % Punctuation after head
    {0.5em plus 0.2em minus 0.1em}                  % Space after head
    {\thmname{#1}\thmnumber{ #2}.\thmnote{ (#3).}}

\theoremstyle{my2theoremstyle}
\newtheorem{thm}{定理}[section]
\newtheorem{law}{定律}[section]
\newtheorem{educt}{推论}
\newtheorem{prop}{命题}
\newtheorem{lemma}{引理}
\newtheorem{axiom}{公理}
\newtheorem{property}{性质}

\newtheoremstyle{my4theoremstyle}
    {1.5ex}                                         % Space above
    {1.5ex}                                         % Space below
    {}                                              % Font for body
    {}                                              % Indent amount
    {\bfseries}                                     % Font for head
    {}                                              % Punctuation after head
    {0.5em plus 0.2em minus 0.1em}                  % Space after head
    {\thmname{#1}.}

\theoremstyle{my4theoremstyle} \newtheorem*{sol}{解}

\newtheoremstyle{my3theoremstyle}
    {1.5ex}                                         % Space above
    {1.5ex}                                         % Space below
    {}                                              % Font for body
    {}                                              % Indent amount
    {\kaishu}                                       % Font for head
    {}                                              % Punctuation after head
    {0.5em plus 0.2em minus 0.1em}                  % Space after head
    {\thmname{#1}\thmnumber{ #2}.\thmnote{ (#3).}}

\theoremstyle{my3theoremstyle} \newtheorem*{remark}{注}
\newtheorem*{cmt}{评注}
\pagestyle{empty}
\begin{titlepage}
    \thispagestyle{empty}
    \centering
        \vspace*{2cm}
        \includegraphics[width=0.5\textwidth]{pic/npu_2.png}\par
        \vspace{1em}
        \includegraphics[width=0.5\textwidth]{pic/npu_1.png}\par
    \vspace{1em}
        \begin{center}
            \Huge \heiti \textbf{中国近现代史纲要}

            An Outline of Modern Chinese History
        \end{center}
        \vspace{17em}
        \begin{center}
        \songti
        \kaishu 软件学院 \, \heiti\textbf{钱锋} \quad \songti 编
        \vspace{0.5em}

    \today
    \end{center}
\end{titlepage}
\cleardoublepage
\maketitle
\cleardoublepage

\frontmatter
\newpage
\pagestyle{plain}
\makeatother

% 设置目录页的页码格式
\pagenumbering{roman} % 切换回罗马数字页码
\addtocontents{toc}{\protect\thispagestyle{empty}}
\pagestyle{plain}
\tableofcontents
\newpage
\thispagestyle{empty}
\cleardoublepage % 确保正文从奇数页开始

\chapter*{导言}
\addcontentsline{toc}{chapter}{导言}
\markboth{中国近现代史纲要}{导言}
\thispagestyle{empty}
% 这里是绪论的内容

\quad\quad\textbf{中国近现代史,就其主流和本质来说,是中国人民为救亡图存和实现
中华民族伟大复兴而英勇奋斗,艰辛探索并不断取得伟大成就的历史。}
因此,实现中华民族的伟大复兴,是中国近现代史纲要课程的第一条,也是最鲜明的一条主线。

中国近现代史纲要课程的第二条主线是中国人民的历史选择。在中国近代史和中国现代史中,
中国人民一共做过四个大的历史选择:
\textbf{选择马克思主义、选择中国共产党、选择社会主义和选择改革开放}。
深刻地理解中国人民为何、如何做出这样的选择,以及这四个历史选择的历史意义,
是学好中国近现代史纲要课程的关键。

中国近现代史纲要课程的第三条主线是\textbf{马克思主义的中国化}。
一个民族要从被压迫走向解放、从贫困走向富裕,一个国家要从被侵略、被殖民走向独立,从弱小走向富强,
不能没有科学理论的指引。俄国的十月革命把马克思主义带到中国,在新民主主义革命开始后,
中国共产党人坚持把马克思主义基本原理同中国革命与建设具体实际相结合,在新民主主义革命时期、社会主义建设时期和中国特色社会主义进入新时代后取得了马克思主义中国化、时代化的三次飞跃。
马克思主义中国具体实际的第一次结合是为了寻找一条正确的革命道路,第二次结合则是为了寻找一条正确的社会主义建设道路。

\subsection*{如何学好历史?}
\setcounter{subsubsection}{0}

如何学好历史?笔者本人的体会是四个关键词:\textbf{唯物主义、时空观念、材料解读和家国情怀}。
在这四个关键词的指导下,我们不仅能掌握系统、全面的历史知识,也能锻炼出批判性思考的能力,受用终身。

\subsection*{《中国近现代史纲要》复习指南}
\setcounter{subsubsection}{0}

对于琐碎的历史事件具体的时间节点和历史细节,在考试中一般不做要求。考试重点考察的是历史事件的性质、意义、经验和教训。
因此在复习阶段,对于时间节点和历史细节的复习可以稍加淡化,通过习题和用历史事件来论述观点的形式来学习具体的历史事件。

\subsection*{如何使用本书}
\setcounter{subsubsection}{0}

这本书的主要目的是帮助你通过这门课程的期末考试和在考研政治的史纲部分拿到高分,所以我在
行文的过程当中会使用大段的文字来向你解读历史知识所代表的含义。一般地,用 \kaishu 楷体
\songti 写的内容往往是书上的原文,是知识点,需要你重点抓,重点掌握,而用宋体写出来的字
(比如说这些)则是我为了帮助你理解而写下的叙述和说明。在学习的时候,你要重点抓的内容
是楷体的内容当中打框的部分,而加粗的部分是需要你在脑海里留下一定的印象的,最少你要知道这是
一件什么事情,或者说你要能够用你自己的语言把打框部分里的加粗关键词串联起来。

\textbf{复习的重中之重,是打了框,并且标记了 “核心要点” 的部分,这部分的内容是考试的重点、难点,
有条件的同学应当加以记忆,记性不那么好的同学,最少最少也要加以理解}。剩下的所有内容,
都是围绕着这些核心要点展开的,是帮助你理解的内容,你只需要在复习的时候针对一些不太理解
的重难点选择性地阅读一下就可以了,或者在考完试之后时间充裕了,当作一个消磨时间的读物一样
浏览一遍就可以了。

除此之外,对于一些选择题考点,我还会以一些例题的形式给出来,大家在复习的时候对照着背一背,
有条件的同学可以自己用 Word 或者其他文字处理软件整理一下、抄一抄,这样的话有利于大家后面的
复习。当然了,有些同学是比较懒的,不想动手但是记忆力贼好,那么对于这一类同学,我专门在附录里
搞了一个\textbf{冲刺速成资料},大家考前拿着这个背也是可以的。

\newpage
\thispagestyle{empty}

% 设置章节标题页的页眉和页脚为空白页样式
\makeatletter
\let\ps@plain\ps@empty
\makeatother
\mainmatter

\chapter{进入近代后中华民族的磨难与抗争}
\thispagestyle{empty}

\quad\quad 本章的内容概括来讲就是 “落后就要挨打”、怎么挨打? 怎么反抗、反抗失败的经验和教训, 以及
我们反抗本身和反抗失败所导致的民族意识的觉醒. 本章的主要知识结构如下:
\begin{enumerate}[label=1.\arabic*, itemsep=0pt]
    \item 鸦片战争前后的中国与世界
    \item 西方列强对中国的侵略
    \item 外国武装侵略的斗争
    \item 反侵略战争的失败与民族意识的觉醒
\end{enumerate}
本章的学习目的是了解鸦片战争前后的中国与世界,认识近代中国社会的半殖民地半封建
社会的性质、主要社会矛盾和基本特征,并了解因近代中国社会主要矛盾的转变而产生的中华民族在近代
的两大历史任务和它们之间的相互关系,认识造成近代中国社会贫困落后的根本原因,并通过对近代中
国人民抵御外国侵略的斗争历史认识近代中国历次反侵略战争失败的根本原因,继承和发扬以爱国主义为
核心的民族精神。

其中,\textbf{中华民族在近代面临的两大历史任务及其相互关系和近代中国反侵略战争失败的原因是本章学习的重点};
\textbf{对鸦片战争前后的中国与世界的认知、理解和近代中国历次反侵略战争失败的原因和教训是本章学习的难点};
各种历史事件的意义、教训和成因是考察重点,对进入近代后的代表性历史事件的时间节点在考察中一般不作要求。

\section{鸦片战争前后的中国与世界}

这一块的内容可以总结为世界走向中国与中国走向世界。随着外国资本主义的入侵,
中国的封建社会逐步变成了半殖民地半封建社会,中国人民逐渐开始了反帝反封建的
资产阶级民主革命。虽然中国走向世界和世界走向中国的时间不同、方式不同、性质
不同、目的不同,但是从全球的观点来看,这个历史阶段最显著的变化是中国成为了
全球性国际社会中的一员,成为了世界的中国。中国外交从单向发展为双向,中国也
基本走向了以欧洲为主体的资本主义世界,成为了全球性国际社会的一员。

那么,中国的封建社会为什么会走向衰落?西方的资本—帝国主义为什么要来到东方?
鸦片战争对近代的中国社会造成了什么样的的改变?为什么鸦片战争是中国近代史
的起点呢?本节我们就要探讨这些问题。

\subsection{中国封建社会的衰落}

% \subsubsection{中国封建社会的主要特点}

自公元前 5 世纪的战国时代到 1840 年鸦片战争,中国的封建社会前后延续了两千多年。
中国封建社会的主要特点是:
\begin{itemize}[itemsep=0pt]
    \item 经济上,封建地主土地所有制经济占主导地位;
    \item 政治上,实行高度集权的封建君主专制制度;
    \item 文化上,中国封建社会的文化思想体系以儒家思想为核心;
    \item 中国封建社会的社会结构特点是族权和政权相结合的封建宗法等级制度,其核心是宗族家长制。
\end{itemize}
中国封建社会的经济、政治、文化、社会结构,一方面巩固和维系了中国封建制度的稳定和延续,
另一方面也使其前进缓慢甚至迟滞,并造成不可克服的周期性的政治经济危机。
\begin{remark}
    在这里需要简单说明的是,中国古代的封建君主制、封建君主专制和封建制度是不能划等号的。
    封建制度是一个很大的范畴,在政治方面,它的表现是封建君主专制;经济方面,它的表现是
    封建地主土地所有制,本质上就是土地剥削;而在文化方面,封建制度的体现是一整套的伦理纲常。
    中国的封建社会,就是一种处在这样的封建制度中的社会。
\end{remark}
\begin{remark}
    注意这里的这个 “周期性的政治危机”,以后还会出现。
\end{remark}

\subsection{世界资本主义的发展与殖民扩张}

\begin{itemize}[itemsep=0pt]
    \item 14 世纪至 15 世纪,在欧洲地中海沿岸的城市里,最早出现了资本主义的萌芽。
    \item 欧洲的文艺复兴冲破了中世纪神学蒙昧主义的精神束缚,
    为欧洲资本主义市场的产生做了思想上的准备;
    \item 而 15 世纪以来的地理大发现,更为欧洲开拓世界市场、发展海外贸易、
    推动殖民扩张提供了条件,加速了欧洲资本主义的兴起;
    \item 1640 年的英国资产阶级革命,标志着世界历史开始进入资本主义时代。
    \item 18 世纪,英国、美国、法国等先后通过资产阶级革命,建立了资产阶级政权,
    为资本主义的发展提供了政治上的前提和保证。
    \item 18 世纪中叶至 19 世纪中叶,从英国开始然后迅速推广到欧美各国的工业革命,
    使大机器生产取代了工场手工业,资本主义经济得到迅速发展。
    \item 19 世纪末,资本主义进入帝国主义阶段,资本输出成为殖民剥削的重要形式,
    并出现瓜分世界的狂潮。西方资本主义的发展及其向东方的殖民扩张,
    使古老的中国遇到了空前严重的挑战。
\end{itemize}

\subsection{鸦片战争的爆发}

\subsubsection{西方学界的主流观点:文化价值冲突论}

1793 年,George Macartney 率英国使团来华,当时的清朝并没有近代的外交理念,只有
宗藩观念。Macartney 抵华后,清廷一如既往地视为 “贡使”,朝见中国皇帝依例当行三跪九叩大礼,
Macartney 抗不遵从,经反复磋商,清廷同意屈一膝以为礼,并拒绝了 Macartney 提出的交涉要求。

Macartney 使团的来华经历成为了西方为发动侵略战争狡辩的借口,
1841 年 12 月,美国众议院外交委员会主席 J.Q.Adams 发表演说,“一般的看法都以为争执不过是
为了英国商人输入几箱鸦片,中国政府因其违法输入而予以查抄,但是我却认为这完全是错误的看法。
这只不过是争端中的一个偶然事故,而并不是战争的原因……战争的原因是磕头!”
Adams 的解释是:通过正常的外交途径无法构建平等的国家关系,只有诉诸战争。\cite{两岸晚清}

西方学界流行的 “文化价值冲突论”,虽有些道理,但不全面,没有把握到战争爆发最主要的根源,
反而更像是侵略者为自身邪恶行径所做的辩解。

\subsubsection{鸦片战争}

鸦片战争的发生是综合因素所致,但其中最重要的不是文化因素,而是经济原因,在于茶叶、白银、
鸦片等。
西方的资本主义社会具有一个主要矛盾,那就是生产的无限扩大于劳动人民具有支付能力的需求之间的
矛盾,这个矛盾决定了西方资本主义有向外扩张的趋势,他们需要找到一片新的市场作为商品的倾销地
和原料的供应地。因此,西方殖民主义势力来到东方,不是使包括中国在内的东方国家为了成为独立的
资本主义国家,而是为了把东方世界纳入资本主义的世界体系,成为殖民地、半殖民地,成为西方世界
经济上、政治上、文化上的附庸。

英国殖民者以走私毒品鸦片作为牟取暴利及改变贸易逆差的手段,强迫其殖民地印度种植鸦片,再由
东印度公司垄断收购、加工,然后走私到中国贩卖。
钦差大臣林则徐于 1839 年 6 月在广东虎门销毁所收缴鸦片的行动,完全是维护国家利益和民族尊严
的正义行动。

为了解决英国对华长期处于贸易入超状态的问题,和 1825 年、1847 年英国发生的两次资本主义经济危机,
并转移国内人民的视线,英国政府迫不及待地要发动一场侵略战争。
1840 年下午 4 月 4 日,英国政府将对华战争议案提交国会,遭
反对党质询,9 日下午,在辩论无果的情况下就反战质询案表决,赞成 262 人,反对 271 人。
“托利党的反战决议案只以五票之差被否决”。一个相当微弱的多数,却在某种程度上决定了对
世界上人口最多的国家的战争的动议得到通过。

鸦片战争以清政府的失败而告终,此后,在这段屈辱的历史时期内,中国与列强签订了一系列的不平等
条约:
\begin{table}[H]
    \centering
    \caption{中国与列强签订的不平等条约(不完全)}
    \label{table: 中国与列强签订的不平等条约(不完全)}
    \begin{tabular}{c c c p{6cm}}
        \hline
        条约名 & 签署国 & 签署时间 & 备注 \\ 
        \hline 
        《南京条约》 & 英国 & 1842 & 近代史上第一个不平等条约,标志着中国开始沦为半殖民地半封建社会 \\
        《虎门条约》 & 英国 & 1843 & \\ 
        《望厦条约》 & 美国 & 1844.7 & \\ 
        《黄埔条约》 & 法国 & 1844.10 & \\ 
        《马关条约》 & 日本 & 1895 & 中国沦为半殖民地半封建社会的程度大大加深了 \\ 
        《辛丑条约》 & 八国联军 & 1901& 标志着以慈禧太后为首的清政府已经彻底放弃
        了抵抗外国侵略者的念头,甘当 “洋人的朝廷”。 \\
        \hline
    \end{tabular}
\end{table}

通过这一系列的不平等条约,英国等西方列强在中国攫取了大量侵略特权。破坏了中国的主权的领土完整,
破坏了中国的领海、司法、关税主权……

随着外国资本主义的入侵,中国的社会性质开始发生质的变化。中国逐步沦为
半殖民地半封建国家。随着社会主要矛盾的变化,中国逐渐开始了
反帝反封建的资产阶级民主革命。正因为如此,鸦片战争成为中国近代史的起点。

\subsubsection{近代中国社会的半殖民地半封建性质}

从鸦片战争直到中华人民共和国成立前夕的中国社会,处于一种称为\textbf{半殖民地半封建社会}
的畸形社会形态。社会性质就是最大的国情,今后在讨论任何中国近代史的社会问题
的时候,都要从这个大前提、大背景出发。
中国半殖民地半封建社会及其特征,是随着资本—帝国主义侵略的扩大、资本—帝国主义与
中国封建势力结合的加深而逐渐形成的。
那么,为什么是半殖民地半封建社会呢?主要是两点原因:
\begin{itemize}[itemsep=0pt]
    \item \textbf{首先,独立的逐步中国变成了半殖民地的中国。}
    鸦片战争以后,中国已经丧失了完全独立的地位,在相当程度上被殖民地化了,
    但近代中国虽然在十几岁已经丧失拥有完整主权的独立国家的地位,但仍然维持着
    独立国家和政府的名义,还有一定的主权。由于它与连名义上的独立也没有而由
    殖民主义宗主国直接统治的殖民地尚有区别,因此被称作半殖民地。
    
    \item \textbf{其次,封建的中国逐步变成了半封建的中国。}
    资本—帝国主义列强用武力打开中国的门户,把中国卷入世界资本主义经济体系和世界市场之中,
    但资本—帝国主义列强并不愿意中国成为独立的资本主义国家。这样,中国的经济既不再是完全的
    封建经济,也不是完全的资本主义经济,成为了半殖民地半封建的经济。
\end{itemize}
简单地来说就是,列强用武力打开了中国的大门以后,我们的封建社会开始瓦解了,
我们的资本主义社会开始形成但没有完全形成,就这样 “卡” 在了中间,变成了
半殖民地半封建社会这种畸形的社会形态。

\begin{mdframed}
    \begin{thm}
        请论述中国半殖民地半封建社会的基本特征。(政治、经济、社会三个方面)
    \end{thm}
    \begin{enumerate}[label=(\arabic*), itemsep=0pt]
        \item 政治方面:
        \begin{enumerate}[label=$\textup{\alph*}$., itemsep=0pt]
            \item 资本—帝国主义日益成为支配中国的决定性力量;
            \item 封建势力日益成为资本—帝国主义压迫、奴役中国人民的社会基础和统治支柱;
        \end{enumerate}
        \item 经济方面:
        \begin{enumerate}[label=$\textup{\alph*}$., itemsep=0pt]
            \item 封建剥削制度的根基即封建地主的土地所有制依然在广大地区内保持着;
            \item 民族资本主义经济虽然已经产生,但发展很缓慢,力量很微弱;
        \end{enumerate}
        \item 社会方面:
        \begin{enumerate}[label=$\textup{\alph*}$., itemsep=0pt]
            \item \textbf{近代中国各地区经济、政治和文化的发展极不平衡};
            \item 广大人民尤其是农民日益贫困化以至大批破产,
            过着饥寒交迫和毫无政治权利的生活。
        \end{enumerate}
    \end{enumerate}
\end{mdframed}

\subsubsection{近代中国社会阶级关系的变动}

鸦片战争前,中国社会只有地主阶级和农民阶级两大阶级,鸦片战争后,中国的大门被迫打开,
中国社会的新阶级开始产生。19 世纪四五十年代,中国工人阶级最早出现于外国资本主义的在华企业中。

两个传统阶级:地主阶级、农民阶级;

两个新兴阶级:工人阶级、资产阶级。

% \begin{cmt}
%     各个新阶级的产生时间:
%     \begin{enumerate}[label=\textup{\arabic*}${}^\circ$]
%         \item 工人阶级:19 世纪四五十年代;
%         \item 民族资产阶级:19 世纪六七十年代;
%         \item 城市小资产阶级:19 世纪末;
%         \item 官僚买办资产阶级:20 世纪 20 年代末。
%     \end{enumerate}
% \end{cmt}

\textbf{近代中国诞生的工人阶级}是中国新生产力的代表。它深受帝国主义、封建势力、资产阶级三重压迫,
受剥削最深、革命性最强,\textbf{是近代中国最革命的阶级}。

\subsubsection{近代中国社会的主要矛盾和两大历史任务}

首先我们要讲的是国情、社会矛盾和历史任务的关系。一切从国情出发,而社会性质是最大的国情。
在讨论中国近代史的任何社会问题的时候,都要从半殖民地半封建社会这个大前提、大背景出发。
当时的我们处于半殖民地半封建社会,而造成这一社会性质产生的始作俑者是帝国主义侵略者和
长期封建统治中国的封建地主阶级。\textbf{国情决定社会矛盾,社会矛盾决定历史任务}。
在近代中国半殖民地半封建的错综复杂的社会矛盾中,占支配地位的主要矛盾,
是\textbf{帝国主义和中华民族的矛盾},\textbf{封建主义和人民大众的矛盾}。
中国近代社会的发展和演变,是上述两队主要矛盾相互交织和交替作用的结果。近代以来伟大的中国
革命,是在这些主要矛盾及其激化的基础知识发生和发展起来的。

中国近代社会的主要矛盾决定了,为了使中国在世界上站起来,
为了使中国人民过上幸福、富裕的生活,就必须推翻帝国主义、封建主义联合统治的半殖民地半封建
的社会制度,\textbf{争取民族独立、人民解放};就必须改变中国经济技术落后的面貌,
\textbf{实现国家富强、人民幸福}。这两个历史任务相互区别又相互紧密联系的。
完成前一项任务是为后一项任务扫清障碍、创造必要的前提。
只有通过革命取得民族独立、人民解放以后,中国人民才有可能集中力量进行现代化建设,
实现国家富强和人民富裕,
所以说,前一个任务是后一个任务的必要条件,后一个任务是前一个任务的最终目的和必然要求。
试想一下,不通过革命战争赶走侵略者、解放中国人民、创造一个好的执政环境和政治舞台,
怎么通过发展来实现国家的富强和人民的幸福呢?所以这两大历史任务的关系是不难理解的。

\begin{mdframed}
    \begin{cmt}
        鸦片战争对近代的中国社会造成了深远的改变,在鸦片战争后,中国社
        会的以下各个方面发生了巨大的变化:
        \begin{enumerate}[label=(\arabic*), itemsep=0pt]
            \item \textbf{中国的社会性质发生根本性变化},
            在鸦片战争后,独立的中国逐步变成了半殖民地的中国,
            封建的中国逐步变成了半封建的中国。
            中国由一个落后封闭的封建国家沦为一个\textbf{半殖民地半封建社会}。
            \item 中国社会阶级结构发生变化,
            除了原有的农民阶级和地主阶级外,出现了\textbf{两大新兴的阶级——资产
            阶级与无产阶级}。近代中国诞生的工人阶级是中国新生产力的代表,
            是近代中国最革命的阶级。
            \item \textbf{社会主要矛盾发生变化},
            主要矛盾由封建主义和人民大众的矛盾转化为\textbf{封建主义和人民大众
            的矛盾以及帝国主义与中华民族的矛盾}。其中,帝国主义与中华民
            族的矛盾是最主要的矛盾。
            \item \textbf{社会任务发生变化},
            近代以来中华民族面临的两大历史任务,就是\textbf{争取民族独立、人民解放
            和实现国家富强、人民幸福}。其中,前一个任务为后一个任务扫除障碍,创造必要
            的前提,后一个任务是前一个任务的最终目的和必然要求。
        \end{enumerate}
        随着社会主要矛盾的变化,\textbf{中国逐步开始了反帝反封建的资产
        阶级民主革命},正因如此,鸦片战争成为中国近代史的起点。
    \end{cmt}
\end{mdframed}

在这里扯一个题外话说明一下,本书勾划重点的方式主要有粗体、打框和标记 “核心要点” 三种方式。
其中,粗体的内容是一些关键词,大家只需要记住这些关键词然后能用自己的语言串联出这些关键词
之间的关系就可以了。打框的内容是选择题考点,这意味着你一定要知道有这样的表述,当这些内容
在选项中出现的时候你要能想得起来并且识别出来可能被偷换的用词,这些往往是命题的陷阱。打框
的内容会在行文过程中反复出现,因为优秀的课程设计在于重复,所以我在写书的时候决定用这种方式
来强迫你把一些重点的内容多重复几遍,不要嫌烦哦同学(嫌烦就对了!嫌烦意味着你已经熟练到一定
程度了)。“核心要点” 部分是我本人根据学习和做题的经验总结出来的一些内容,
这些内容往往在书上没有原文并且高度凝练,是一些在纵向上联系了本课程的各个章节,
又在横向上联系了其他四门思想政治理论课的综合性的表述,
对你的理解很有帮助。而\textbf{既打了框,又标注了 “核心要点” 的内容,就是全书的重中之重,
是你在复习备考的时候一定要过一遍的内容,记得越熟越好}。所以,请你再折回去,把上面既打了框
又是 “核心要点” 的这一段内容再看一遍。

\subsection{复习参考题}

\begin{example}
    中国从封建社会逐步沦为半殖民地半封建社会,中国社会的阶级关系也发生了深刻的变动。
    近代中国最早产生的新的阶级是 \underline{\qquad \qquad \qquad}。
    \begin{tasks}[label={\Alph*. }](4)
        \task 民族资产阶级
        \task 官僚买办资产阶级
        \task 工人阶级
        \task 城市小资产阶级
    \end{tasks}
\end{example}

\begin{mdframed}[frametitle={鸦片战争的影响:中国社会的阶级变化}]
    \begin{enumerate}[itemsep=0pt]
        \item 两个传统阶级:封建阶级和农民阶级。
        \item \textbf{两大新兴阶级}:资产阶级与无产阶级。近代中国诞生的工人阶级是中国新生产力的代表,
        是近代中国最革命的阶级。
    \end{enumerate}
\end{mdframed}

\begin{example}
    1840 年 6 月,英国舰队封锁了珠江海口和广州海面,鸦片战争正式爆发。下列关于
    鸦片战争的后果,说法错误的是:
    \begin{tasks}[label={\Alph*. }](1)
        \task 签订了中国近代史上第一个不平等条约
        \task 外国人在华不受中国法律管束
        \task 允许外国人在华开办工厂
        \task 帝国主义与中华民族的矛盾成为中国社会最主要的矛盾
    \end{tasks}
\end{example}

\begin{example}
    (多选)鸦片战争以后,西方列强通过发动侵略战争,强迫中国签订了一系列不平等条约,
    使中国沦为半殖民地半封建社会。中国逐步沦为半殖民地社会的原因包括:
    \begin{tasks}[label={\Alph*. }](1)
        \task 中国已经丧失了完全独立的地位
        \task 列强把中国卷入世界资本主义经济体系和世界市场之中
        \task 中国仍然维持着独立国家和政府的名义
        \task 西方列强不愿意中国成为独立的资本主义国家
    \end{tasks}
\end{example}

\begin{mdframed}[frametitle={为什么是半殖民地半封建社会?}]
    \begin{enumerate}[itemsep=0pt]
        \item 独立的中国逐步变成了半殖民地的中国:中国已经方式了完全独立的地位,但
        仍然维持着独立国家和政府的名义;
        \item 封建的中国逐步变成了半封建的中国:列强把中国卷入世界资本主义经济体系
        和世界市场之中,但不愿意中国成为独立的资本主义国家。
    \end{enumerate}
\end{mdframed}

\section{西方列强对中国的侵略}

本节的主要考点在于考察西方列强对中国侵略的具体表现,一般来讲,在选项中我们是能够
识别出具体侵略罪行所属的领域的。总而言之,在一般情况下,与领土主权有关的都是军事侵略,
与行政权力有关的都是政治控制,与钱有关的都是经济掠夺,与思想文化有关的都是文化渗透。
少数的特例(或者不易识别的选项)需要在平时解题过程中注意积累。

接下来我们把西方列强对中国的侵略行径不完全地罗列一下,你可以在闲着没事干的时候翻翻看看,
你没必要背,也可以为了备考而先跳过这张表格,但是请你一定要抽空回来看,你会直观的感受到
什么叫落后就要挨打,什么叫罄竹难书。
\begin{longtable*}{p{0.12\textwidth}|p{0.08\textwidth}|p{0.08\textwidth}|p{0.14\textwidth}|p{0.45\textwidth}}
    \toprule
    \textbf{侵略领域} & \textbf{列强} & \textbf{时间} 
    & \textbf{侵略行径} & \textbf{史实} \\
    \midrule
    \endhead
    军事侵略 & 英国 & 1840 & 发动侵略战争 & 鸦片战争 \\
    \hline
    军事侵略 & 英国 & 1842 & 侵占中国领土 & 强迫清政府签订《南京条约》,
    把香港岛割让给英国。 \\
    \hline
    军事侵略 & 英国 & 1845 & 划分势力范围 & 
    租得上海外滩附近 837 亩土地,设立上海英租界。 \\
    \hline
    军事侵略 & 英、法、美、德、日、俄、意、比、奥等国 & 1845 至 1911 & 划分势力范围 & 
    先后在上海、天津、汉口、广州、福州、重庆等 16 个城市,设立了 30 多个租界。 \\
    \hline
    军事侵略 & 葡萄牙 & 1849 & 侵占中国领土 & 武力强占澳门半岛。 \\
    \hline
    军事侵略 & 俄国 & 1858 & 侵占中国领土 & 
    利用英、法发动第二次鸦片战争之机,胁迫黑龙江将军奕山签订《瑷珲条约》,
    割去黑龙江以北 60 万平方公里领土。 \\
    \hline
    军事侵略 & 英国 & 1860 & 侵占中国领土 & 通过中英《北京条约》,
    割去香港岛对岸九龙半岛南端和昂船洲。 \\
    \hline
    军事侵略 & 俄国 & 1860 & 侵占中国领土 & 
    通过签订中俄《北京条约》,割去乌苏里江以东 40 万平方公里领土。 \\
    \hline
    军事侵略 & 俄国 & 1864 & 侵占中国领土 & 
    强迫清政府签订《勘分西北界约记》,割去中国西北 44 万平方公里领土。 \\
    \hline
    军事侵略 & 俄国 & 1881 & 侵占中国领土 & 
    通过《改订伊犁条约》和 5 个勘界议定书,割去中国西北 7 万多平方公里领土。
    通过这一系列不平等条约,俄国共侵占中国领土 150 多万平方公里。 \\
    \hline
    军事侵略 & 葡萄牙 & 1887 & 侵占中国领土 & 
    胁迫清政府订立《中葡和好通商条约》,允许葡萄牙 “永居管理澳门”。 \\
    \hline
    军事侵略 & 日本 & 1894.11 & 屠杀中国人民 
    & 日军在甲午战争中制造了旅顺大屠杀惨案,在 4 天内连续屠杀中国居民 2 万余人。 \\
    \hline
    军事侵略 & 日本 & 1895 & 侵占中国领土 
    & 日本强迫清政府签订《马关条约》,割去中国台湾全岛及其所有附属岛屿和澎湖列岛。 \\
    \hline
    军事侵略 & 德国 & 1898 & 划分势力范围 
    & 强租山东的胶州湾,把山东划为其势力范围。 \\
    \hline
    军事侵略 & 沙俄 & 1898 & 划分势力范围 
    & 强租辽东半岛的旅顺口、大连湾及其附近海面,以长城以北为其势力范围。 \\
    \hline
    军事侵略 & 英国 & 1898 & 划分势力范围 
    & 强租山东的威海卫和香港岛对岸的九龙半岛界限街以北、深圳河以南及附近的岛屿(新界),
    以长江流域为其势力范围。 \\
    \hline
    军事侵略 & 法国 & 1899 & 划分势力范围 
    & 法国强租广东的广州湾及其附近水面,把广东、广西、云南作为其势力范围。 \\
    \hline
    军事侵略 & 日本 & 1899 & 划分势力范围 
    & 声明把福建作为其势力范围。 \\
    \hline
    军事侵略 & 俄国 & 1900.7 & 屠杀中国人民 
    & 俄国入侵中国东北时,先后制造了海兰泡惨案和江东六十四屯惨案。沙俄军警
    把中国人居住的村庄烧光,把数千居民枪杀,或驱入黑龙江中活活淹死。 \\
    \hline
    军事侵略 & 八国联军 & 1900.8 & 屠杀中国人民 
    & 八国联军侵占北京后,仅在庄王府一处,就烧死和杀死义和团团民与平民 1700 多人。 \\
    \hline
    军事侵略 & 八国联军 & 1901 & 划分势力范围 
    & 《辛丑条约》规定,外国军队有权在北京使馆区和北京至大沽、山海关一线
    包括天津、唐山等 12 处 “留兵驻守”。 \\
    \hline
    军事侵略 & 日本 & 1901 & 划分势力范围 
    & 日俄战争后,日本从俄国手中攫得租自中国的旅顺口的大连湾、长春至旅顺口的铁路及
    其他有关权益,在旅顺设置 “关东总督府”,并派兵驻守上述地区及 “南满铁路” 沿线。
    这支军队后来被称作 “关东军”,成了日本侵略中国的突击队。 \\
    \hline
    \midrule
    政治控制 \\ 
    \midrule 
    经济掠夺 \\ 
    \midrule
    文化渗透 \\
    \bottomrule
\end{longtable*}

(未完待续,现在只列了不完全统计的 1/10 左右,任重而道远啊,罄竹难书,罄竹难书)

% \begin{table}[H]
%     \centering
%     \caption{细数西方列强的罪恶}
%     \begin{tabular}{c p{0.5\textwidth}}
%         \toprule
%         领域 & 列强国家 & 具体表现 \\ 
%         \midrule
%         军事侵略 & 英国 & 发动侵略战争,划分势力范围,勒索赔款\\
%         政治控制 &  & 控制内政外交,镇压人民反抗,扶植、收买代理人,把持海关行政权\\ 
%         经济掠夺 &  & 控制通商口岸,剥夺关税自主,商品倾销,资本输出,操控中国经济\\ 
%         文化渗透 &  & 假传教真侵略,炮制中国威胁论,办报纸,庚款兴学 \\ 
%         \bottomrule
%     \end{tabular}
% \end{table}

我们在这里针对一些容易混淆的问题简单说明一下,
勒索赔款的重点在勒索,试想一下,我拿一把刀架在你脖子上要你给我钱,你敢不给吗?
所以,西方列强在战后勒索一笔赔款,本质上还是对我们的军事侵略。

除此之外,对于海关权和关税的领域划分是一个容易混淆的问题。按照我们的原则,
与行政权力有关的都是政治控制,所以把持海关行政权本质上是政治控制,而与钱有关的都是
经济掠夺,所以剥夺关税自主本质上是经历掠夺。所以在海关问题上,西方列强的政治控制和经济掠夺是同时发生的。

还有一个,也是西工大期末考试比较爱考的一个,是庚款兴学。这里的 “庚款” 指的其实
就是《辛丑条约》签订以后的庚子赔款。1907 年,时任美国总统西奥多·罗斯福宣布将退还部分庚子赔款,
用于资助中国政府选派留学生赴美留学,史称 “庚款兴学”。从 1909 年第一批庚款留美学生至 1929 年,
整整持续 20 年之久,很多留学生成为美国思想的忠实信徒,甚至站在美国立场上为美国侵华行径做辩护。
毛泽东在《“友谊”,还是侵略?》一文中指出,\textbf{“庚款兴学” 的实质是帝国主义的文化侵略}。

总之,这一节的考察方法比较固定,但是出题点比较琐碎和细节,也有可能会考察一些教材上没有
出现过的史实。但总而言之,识别出西方列强到底在侵略我们的什么其实是不难的。

\begin{example}
    (多选)资本—帝国主义列强在对中国实行军事侵略、政治控制、经济掠夺的同时,还对中国进行
    文化渗透,具体包括:
    \begin{tasks}[label={\Alph*. }](2)
        \task 以传教为掩护进行间谍活动
        \task 创办《万国公报》
        \task 炮制了 “中国威胁论”
        \task 允许外国公使常驻北京
    \end{tasks}
\end{example}



% \subsection{军事侵略}
% \subsection{政治控制}
% \subsection{经济掠夺}
% \subsection{文化侵略}

\section{反抗外国武装侵略的斗争}

\subsection{抵御外来侵略的斗争历程}

这一块的内容在考题中多是以一些琐碎的史实作为选项出现。笔者建议各位读者在复习备考过程中
注意有关知识和有关内容的积累。一般来说,在选项中能够明显地看出这个人、或这群人是属于
人民群众还是爱国官兵。

然后在这里需要强调的是,阶级属性,或者说立场,和爱国是不一样的。例如我们即将谈到的
以邓世昌为代表的清朝的爱国将领,他们虽然属于腐朽的封建地主阶级,不是先进生产力和先进
生产关系的代表,但他们不可谓不爱国。
还有我们即将讲到的义和团运动,虽然他们有着笼统的排外主义错误,还存在着迷信、落后的倾向,
属于农民阶级,但他们也不可谓不爱国。
关于这一点,是各位读者需要了解的,所以在设问中
如果说他们是爱国的,这并不错,不能作为错误选项排除。

\subsubsection{人民群众的反侵略斗争}

1841 年 5 月,广州三元里人民的抗英斗争,是中国近代史上中国人民第一次大规模的反侵略武装
斗争,显示了中国人民不甘屈服和敢于斗争的英雄气概。

\subsubsection{爱国官兵的反侵略斗争}

中日甲午战争中,“致远” 舰受伤下沉,管带邓世昌下令向日军 “吉野” 号猛撞,不幸被鱼雷击中,
船体炸裂沉没,全舰 250 名官兵壮烈牺牲。

\subsection{义和团运动与列强瓜分中国图谋的破产}

\begin{mdframed}[frametitle={义和团运动失败的原因:爱国切不可学义和团}]
    \begin{enumerate}[label=\textup{\arabic*}${}^\circ$, itemsep=0pt]
        \item 对帝国主义的认识还停留在感性认识的阶段,存在着\textbf{笼统的排外主义错误};
        \item 认识不到帝国主义联合中国封建地主阶级压迫中国人民的实质,
        曾经\textbf{蒙受封建统治者的欺骗};
        \item 由于小生产者的局限性,义和团运动中还存在着\textbf{迷信、落后}的倾向。
    \end{enumerate}
\end{mdframed}

\subsubsection{边疆危机和瓜分危机}

帝国主义列强对中国的争夺和瓜分的图谋,在 1894 年中日甲午战争爆发后达到高潮。
中日《马关条约》签订后,俄国、法国、德国三国干涉还辽,迫使日本放弃了割占辽东半岛的
要求。1898 年至 1899 年,帝国主义列强竞相租借港湾和划分势力范围,掀起了瓜分中国的
狂潮。

\subsubsection{列强瓜分中国图谋的破产}

正是包括义和团在内的中华民族为反抗侵略所进行的前赴后继、视死如归的战斗,才粉碎了帝国主义
列强瓜分和灭亡中国的图谋。
\begin{remark}
    这里需要强调的是,义和团运动在粉碎帝国主义列强瓜分中国的斗争中发挥了重大的历史作用,
    \textbf{并不是说义和团运动粉碎了帝国主义列强瓜分中国的图谋}。所以,当选项中出现
    类似的错误表述的时候,千万不能选。不过一般来说,像这种错误,没复习过的同学不会犯,
    复习得比较全面的同学也不会犯,就是那些复习了但复习的不彻底、不仔细的同学,容易看到
    书上的那一句话然后就错误地把帝国主义列强瓜分中国图谋破产归功于义和团运动了。
\end{remark}

\begin{mdframed}
    \begin{cmt}
        瓜分中国图谋破产的原因:
        帝国主义列强之所以没有能够实现瓜分中国的图谋,
        一个重要的原因是\textbf{帝国主义列强之间的矛盾和相互
        制约}。而帝国主义列强不能灭亡和瓜分中国,最根本的原因在
        于\textbf{中华民族的不屈不挠的反侵略斗争}。
    \end{cmt}
\end{mdframed}


\subsection{复习参考题}

\begin{example}
    (多选)资本—帝国主义侵略、压迫中国人民的过程,同时也是中国人民反抗它们的侵略、压迫的过程。
    救亡图存,成了一代又一代中国人面临的神圣使命。其中,爱国官兵反抗外来侵略的斗争有:
    \begin{tasks}[label={\Alph*. }](2)
        \task 广州三元里抗英斗争
        \task 镇南关大捷
        \task 义和团反抗八国联军侵华
        \task 中日甲午海战
    \end{tasks}
\end{example}

\begin{example}
    (多选)1895 年签订的《马关条约》不仅割让台湾岛及其附属岛屿、澎湖列岛和辽东半岛
    给日本,使日本得到巨大的利益,还顺应了帝国主义各国向中国输出资本的愿望,标志着
    中日甲午战争的结束和洋务运动的破产。《马关条约》签订后,逼迫日本归还辽东半岛的
    国家有 \underline{\qquad \qquad \qquad}。
    \begin{tasks}[label={\Alph*. }](4)
        \task 俄国
        \task 德国
        \task 英国
        \task 法国
    \end{tasks}
\end{example}

\section{反侵略战争的失败与民族意识的觉醒}


\begin{mdframed}
    \subsection{反侵略战争的失败及其原因}
    
    \subsubsection{社会制度的腐败(根本原因)}

    正是社会制度的腐败,才使得经济技术落后的状况长期得不到改变。

    \subsubsection{经济技术的落后}
    19 世纪中叶,西方资本主义强国经过工业革命,经济和技术飞速发展,
    中国的封建统治者夜郎自大,闭关锁国,导致中国落后于时代发展步伐。

\end{mdframed}

\subsection{民族意识的觉醒}

% 这又是一个琐碎的考点,建议各位读者在复习备考过程中自行积累。

% \subsubsection{“师夷长技以制夷” 的主张和早期的维新思想}

% 林则徐是近代中国睁眼看世界的第一人。

% 在《海国图志》中,魏源提出了 “师夷长技以制夷” 的思想,主张学习外国先进的军事和科学技术,
% 以期富国强兵,抵御外国侵略。



% \subsubsection{救亡图存和振兴中华}

% 1895 年,严复写了《救亡决论》一文,响亮地喊出了 “救亡” 的口号。在甲午战争后,严复
% 翻译了《天演论》(1898 年正式出版)。他用 “物竞天择”、“适者生存” 的社会进化论思想,
% 为这种危机意识和民族意识提供了理论根据。《天演论》对中国人无疑是振聋发聩的警世钟。

% 绘制于 1898 年的《时局图》,形象地表现了当时中国面临的瓜分危局。
% 关于《时局图》,我们说明一下,根据有关的 Wikipedia 词条,
% 一般的说法认为《时局图》的作者是兴中会会员谢缵泰(1872年-1939年)。
% 其题词作者,则有三种不同说法:一是不知姓名的广东人;二是作者谢缵泰自己;
% 三是晚清政治家黄遵宪(1848年-1905年)。

% 孙中山 1894 年 11 月在创立革命团体兴中会时,喊出了 “振兴中华” 这个时代的最强音。

\begin{table}[H]
    \centering
    \caption{标志着中国人民的民族意识逐渐觉醒的人物、思想或事件}
    \begin{tabular}{p{0.2\textwidth} p{0.3\textwidth} p{0.4\textwidth}}
        \toprule
        人物 & 事件/作品 & 思想/警示 \\ 
        \midrule
        \multirow{2}{*}{林则徐} & 近代中国睁眼看世界的第一人 \\ 
        & 《四洲志》 \\
        魏源 & 《海国图志》 & 师夷长技以制夷 \\
        早期的维新派 & & 不仅要学习西方的科学技术,同时要求也要吸纳西方的政治、经济学说,具有一定程度的反对封建专制的民主思想 \\
        \multirow{2}{*}{严复} & 《救亡决论》 & 救亡 \\ 
        & 《天演论》 & 物竞天择、适者生存 \\
        谢缵泰 & 《时局图》 & 瓜分危局 \\
        孙中山 & 创立革命团体兴中会 & “振兴中华” 的时代最强音 \\
        \bottomrule
    \end{tabular}    
\end{table}

\textbf{中日甲午战争以后,当中华民族面临危机存亡的关头时,中国人才开始有了普
遍的民族意识的觉醒}。

\subsection{复习参考题}

\begin{example}
    (多选)魏源在其所著的《海国图志》一书中提出了 “师夷长技以制夷” 的思想,
    19 世纪 60 年代开始的洋务运动则提出了 “自强” “求富” 的口号。
    二者的相同点包括:
    \begin{tasks}[label={\Alph*. }](2)
        \task 有内在的一致性和继承性
        \task 都有抵御外来侵略的意图
        \task 主要体现了地主阶级的要求
        \task 意识到了中国落后挨打的根本原因
    \end{tasks}
\end{example}

\begin{example}
    1840 年鸦片战争以后,中国遭受西方列强“船坚炮利”的欺凌不断加深,中华民族面临生死存亡的形势也日益严峻,
    中国“睡狮”在西方列强的隆隆炮声中逐渐苏醒。促使中国人民的民族意识开始普遍觉醒的重大事件是 \underline{\qquad \qquad \qquad}。
    \begin{tasks}[label={\Alph*. }](2)
        \task 中法战争
        \task 中日甲午战争
        \task 八国联军侵华战争
        \task 日本全面侵华战争
    \end{tasks}
\end{example}

\begin{example}
    (多选)1894 年 7 月丰岛海战后,中日两国相互宣战,战至 1895 年 2 月北洋海军全军覆没,
    中日两国各自改革 30 年后的决战以清政府的惨败而告终,1895 年 4 月清政府被迫签署
    《马关条约》。这场战争对中国产生了极其严重的后果,表现在:
    \begin{tasks}[label={\Alph*. }](1)
        \task 清政府已经彻底沦为 “洋人的朝廷”;
        \task 中断了清政府通过洋务运动向近代化转型的努力;
        \task 中国付出了丧失领土主权的极大代价;
        \task 清政府损失了海军主力
    \end{tasks}
\end{example}

\begin{example}
    王韬在近代首次提出 “变法” 的主张,他在介绍西方国家的 “君主”、“民主”、“君民共主”
    这三种制度时,最早提出废除封建君主专制,建立 “与民众共政事并治天下” 的君主立宪制。
    该思想:
    \begin{tasks}[label={\Alph*. }](2)
        \task 最早提出发展资本主义
        \task 推动了维新变法的兴起
        \task 反思了当时中国近代化的问题
        \task 引入了社会进化论的思想
    \end{tasks}
    \begin{sol}
        19 世纪 70 年代以后,王韬、薛福成、马建忠、郑观应等人不仅主张学习西方的科学技术,
        同时也要求吸纳西方的政治、经济学说。它们的共同特点,就是具有比较强烈的反对外国侵略、
        追求中国独立富强的爱国思想,以及具有一定程度反对封建专制的民主思想。
        这些主张具有重要的思想启蒙的意义,我们现在称他们为 “早期的维新派”。
    \end{sol}
\end{example}

\begin{mdframed}[frametitle={救亡图存的各种尝试}]
    \begin{table}[H]
        \centering
        \begin{tabular}{c c c c}
            \toprule
            人物/历史事件 & 阶级属性 & 学习军事、科学、技术 & 学习思想、制度、文化 \\
            \midrule
            魏源、林则徐等人 & 封建地主阶级 & T & F \\
            洋务派/洋务运动 & 封建地主阶级 & T & F \\
            王韬/早期维新派 & 封建地主阶级 & T & T \\
            \bottomrule
        \end{tabular}        
    \end{table}
\end{mdframed}

% \subsubsection{中国历代历次反侵略战争失败的原因和教训是什么?}

% 中国近代历次反侵略战争失败的原因从中国内部因素来分析,主要有两个方面,
% 即\textbf{社会制度的腐败}和\textbf{经济技术的落后},而社会制度的腐败是更根本的原因。
% 正是由于社会制度的腐败,经济技术落后的状况才长期得不到改变。

% 中国近代历次反侵略战争失败的教训是:
% (1) 社会制度的腐败导致反侵略战争失败,落后就要挨打;
% (2) 必须建立先进的社会制度,要进行制度创新;
% (3) 必须大力发展先进生产力、富国强兵等。


% \subsubsection{民族意识的觉醒}

% 近代中国睁眼看世界的第一人是\textbf{林则徐}。
% 中国人民的民族意识开始普遍觉醒是在\textbf{中日甲午战争以后},当中华民族面临生死存亡的关头时。




% \newpage
% \section{拓展提高}

% \begin{cmt}
%     这一章中出现了许多琐碎的历史人物。在此总结如下:
%     \begin{enumerate}[itemsep=0pt]
%         \item 王韬,曾参与洋务运动,该经历及其对洋务运动的认识使他深刻体会到洋务运动的不足,
%         从而萌发出学习西方政治制度的思想主张。在近代首次提出 “变法” 的主张,在介绍西方国家
%         的 “民主”、“君主”、“君民共主” 这三种制度时,最早提出废除封建君主制,建立 “与民众共
%         政事并治天下” 的君主立宪制,该思想反思了当时中国近代化的问题。
%         \item 洪仁玕,近代中国最早提出发展资本主义主张。
%         \item 康有为、梁启超的维新思想推动了维新变法运动的兴起。
%         \item 西方社会进化论思想传入中国始于严复。
%     \end{enumerate}
% \end{cmt}



\clearpage
\thispagestyle{empty}

\chapter{不同社会力量对国家出路的早期探索}
\thispagestyle{empty}

\setcounter{example}{0}
\setcounter{thm}{0}

\quad\quad 这是一段近代中国仁人志士扛起救亡图存、振兴中华的爱国主义大旗,历经艰辛,
多方求索,挽救中华民族危亡的历史。
太平天国运动、洋务运动和维新运动的兴起和失败,说明了农民阶级、地主阶级洋务
派和资产阶级维新派都不可能是中国真正实现民族独立和国家富强,深刻地揭示了
无产阶级领导中国革命走向胜利的必然性。
本章的章节结构如下:
\begin{enumerate}[label=2.\arabic*,itemsep=0pt]
    \item \textbf{太平天国运动的起落}:介绍太平天国农民战争爆发的原因、经过,
    和这次农民起义的发展和失败,并分析了这次农民战争失败的原因,说明了单纯的农民战争
    不可能完成争取民族独立和人民解放的历史任务;
    \item \textbf{洋务运动的兴衰}:本节介绍洋务运动的兴办和失败,并分析了洋务运动失败
    的原因,说明了以洋务派为代表的封建的地主阶级不可能为中国摆脱贫弱找到出路,
    也不可能避免最终失败的命运;
    \item \textbf{戊戌维新运动的兴起和夭折};
\end{enumerate}

从实现近现代中国历史任务的角度,阐明农民阶级、地主阶级、资产阶级\textbf{对国家出路进行早期探索的历史价值及其内在关联性}是本章学习的重点;
运用马克思主义立场和观点,\textbf{深入剖析各个阶级的特点及其局限性},对其发动的政治运动做出客观评价是本章学习的难点。

\section{太平天国运动的起落}

\subsection{太平天国农民战争}

\subsubsection{金田起义和天平天国的建立}

太平天国农民起义爆发的原因: (1) 清政府加重了赋税的征收科派,农民的经济负担更为沉重;
(2) 由于西方资本主义的入侵,中国的农业和家庭手工业相结合的自然经济逐渐解体。

太平天国运动的性质:太平军所进行的战争,是一次反对清政府腐朽统治和地主阶级压迫、剥削的正义战争。

\subsubsection{《天朝田亩制度》和《资政新篇》}

\textbf{《天朝田亩制度》}是最能体现太平天国社会理想和这次农民起义特色的\textbf{纲领性文件}。
\begin{enumerate}[label=(\arabic*), itemsep=0pt]
    \item 确立了平均分配土地的方案,实际上是起义农民提出的一个以解决土地问题为中心的比较完整的社会改革
    方案。
    \item《天朝田亩制度》的主张,\textbf{否定了封建社会的基础即封建土地所有制}。
    \item 这部文件的局限性体现在,它并没有超出农民小生产者的狭隘眼界,这种社会理想,
    具有不切实际的空想的性质;
    \begin{remark}
        中国要实现共同富裕,但不是搞平均主义,而是要先把 “蛋糕” 做大,
        然后通过合理的制度安排把 “蛋糕” 分好,水涨船高、各得其所,
        让发展成果更多更公平惠及全体人民。
    \end{remark}
    \item 《天朝田亩制度》中的平分土地方案即使在太平军所占领地区也并未能付诸实施。
\end{enumerate}

\textbf{《资政新篇》}是\textbf{中国近代史上第一个具有资本主义色彩的方案}。
但限于当时的历史条件,未能付诸实施。

\subsubsection{从天京事变到太平天国败亡}

1856 年 9 月, 发生了自相残杀的天京事变, 成为太平天国由盛转衰的分水岭.

\subsection{农民斗争的意义和局限}

\subsubsection{太平天国农民起义的历史意义}

太平天国农民战争虽然失败了,但是其具有重大的历史意义:
\begin{enumerate}[label=(\arabic*), itemsep=0pt]
    \item 沉重打击了封建统治阶级,强烈撼动了清政府的统治根基;
    \item 是中国旧式农民战争的最高峰;
    \item 冲击了孔子和儒家经典的正统权威;
    \item 有力打击了外国侵略势力;
    \item 冲击了西方殖民主义者在亚洲的统治。
    \item 在 19 世纪中叶的亚洲民族解放运动中,太平天国农民战争是其中时间最久、
    规模最大、影响最深的一次。
\end{enumerate}

\begin{table}[H]
    \centering
    \caption{从任务、纲领、结果三个层面认识太平天国}
    \begin{tabular}{c|p{0.4\textwidth}}
        任务层面 & 反帝反封建 \\ 
        纲领层面 & 反封建, 但并没有明确的反帝纲领 \\ 
        结果层面 & 打击了帝国主义
    \end{tabular}
\end{table}

\subsubsection{太平天国农民起义失败的原因和教训}

太平天国农民战争动摇了清王朝封建统治的基础,有力地打击了西方资本主义侵略者,
显示了农民阶级的反抗精神和战争力量,但最终失败了。

\begin{enumerate}[label=(\arabic*), itemsep=0pt]
    \item 太平天国农民战争失败的根本原因是\textbf{农民阶级不是新的生产力和生产关系的代表},
    他们无法克服小生产者所固有的阶级局限性,这种阶级局限性主要表现在:
    \begin{enumerate}[label=\roman*., itemsep=0pt]
        \item 无法从根本上提出完整的、正确的政治纲领和社会改革方案;
        \item 无法制止和克服领导集团自身腐败现象的滋生;
        \item 无法长期保持领导集团的团结;
    \end{enumerate}
    \item 太平天国在军事战略上出现了重大失误;
    \item \textbf{没有科学理论的指导}:太平天国是以宗教来发动、组织群众的,但是,
    拜上帝教教义毕竟不是科学的思想理论,它不仅不能正确指导斗争,而且给农民战争带来了
    危害;
    \item 未能正确地对待儒学;
    \item 太平天国的领袖们虽然不承认不平等条约,但他们不能把西方国家的侵略者与
    人民群众区别开来,而是笼统地把信奉天父上帝的西方人都称为 “洋兄弟”,这说明
    他们对于西方资本主义侵略者还缺乏理性的认识。
\end{enumerate}

% \vspace*{0.5cm}

太平天国农民战争及其失败表明,在半殖民地半封建的中国,农民具有伟大的革命潜力;
但它自身不能担负起领导反帝反封建斗争取得胜利的重任。单纯的农民战争不可能完成争取
民族独立和人民解放的历史任务。

\begin{mdframed}[frametitle={太平天国运动:旧式农民起义的最高峰}]
    \begin{enumerate}[label=\textup{\arabic*}${}^\circ$, itemsep=0pt]
        \item 两个纲领性文件:
        \begin{enumerate}[itemsep=0pt]
            \item 《天朝田亩制度》最能体现太平天国社会理想和这次农民起义特色,
            确立了平均分配土地的方案,否定了封建社会的基础即封建土地所有制,
            但抱有不切实际的平均主义幻想。
            \item 《资政新篇》是\textbf{中国近代史上第一个具有资本主义色彩的方案}。
            但限于当时的历史条件,未能付诸实施。            
        \end{enumerate}
        \item 四点失败的原因:
        \begin{enumerate}[itemsep=0pt]
            \item 阶级局限性:农民阶级不是新的生产力和生产关系的代表;
            \item 军事战略上出现重大失误;
            \item 缺乏科学思想理论的指导;
            \item 未能正确地对待儒学;
            \item 对侵略者的认识不清。
        \end{enumerate}
    \end{enumerate}
\end{mdframed}

\subsection{复习参考题}

\begin{example}
    太平天国定都天京后,进行了一系列制度建设,并颁布了《天朝田亩制度》。该纲领
    将农民平均主义思想进行了制度化。但这篇纲领始终只是空想、幻想,这是因为:
    \begin{enumerate}[label=\Alph*., itemsep=0pt]
    \item 拜上帝教的教义并非科学的指导思想;
    \item 没有从根本上否定封建地主土地所有制;
    \item 提出发展资本主义的方案在中国行不通;
    \item 在分散的小农经济基础上不可能真正实现均贫富
    \end{enumerate}
\end{example}

\section{洋务运动的兴衰}

洋务运动是封建地主阶级引进近代工业,仿造洋枪洋炮,以期富国强兵,延续封建统治而兴起的一场运动。

洋务派兴办洋务事业的原因,首先是为了购买和制造洋枪洋炮以镇压农民起义,同时也有借此加强
海防、边防,并趁机发展本集团的政治、经济、军事实力的意图。

洋务运动的指导思想是\textbf{“中学为体,西学为用”},
就是仍然以中国封建伦理纲常所维护的统治秩序为主体,辅之以西方的近代工业和技术,
所以,\textbf{兴办洋务新政的主要目的是维护封建统治}。

\subsection{洋务事业的兴办}

\subsubsection{兴办近代企业}

% 2023 版教材 p56

洋务派首先兴办的是军用工业。这些军用工业的名称、创办的时间、筹办的官员和一些附加的
参考信息如下表所示,大家混个眼熟即可。

\begin{table}[H]
    \centering
    \captionof{table}{洋务派兴办的规模较大的军工企业}
    \begin{tabular}{c c c c c}
        \toprule
        名称 & 时间 & 官员 & 备注 \\
        \midrule
        安庆军械所 & 1861 & & 最早创办 \\
        上海江南制造总局 & 1865  & 曾国藩支持、李鸿章筹办 & 当时国内最大的兵工厂 \\ 
        金陵机器局(南京) & 1865  & 李鸿章 &  \\ 
        福州船政局 & 1866  & 左宗棠  \\ 
        天津机器局 & 1867  & 崇厚  \\
        湖北枪炮厂 & 1890 & 张之洞 \\   
        \bottomrule  
    \end{tabular}
\end{table}

洋务派在创办军事工业中遇到资金奇缺、原料和燃料供应不足,以及交通运输等困难,
因此还兴办了一些民用企业,这些企业除少数采取官办或官商合办的方式外,
多数都采取官督商办的方式。\textbf{这些官督商办的民用企业,虽然受官僚的控制,
发展受到很大限制,但基本上是资本主义性质的近代企业}。

\subsubsection{建立新式海陆军}
\subsubsection{创办新式学堂、派遣留学生}

新式学堂主要有三种:翻译学堂、工艺学堂、军事学堂。此外,还先后派遣赴美留学幼童
及官费赴欧留学生 200 多人。

\subsection{洋务运动的历史作用及失败}

\subsubsection{洋务运动的历史作用}

\begin{enumerate}[label=(\arabic*), itemsep=0pt]
    \item 洋务派发展近代军事工业和民用工业,在客观上对中国早期工业和民族资本主义的发展起到了某些
    促进作用;
    \item 洋务派兴办新式学堂,是中国近代教育的开始,还翻译了一批自然科学书籍,
    给当时的中国带来了新的知识,使人开阔了眼界;
    \item 社会风气和价值观念开始变化,工商业者的地位上升,这一切,都有利于资本主义经济的发展,
    也有利于社会风气的转变。
\end{enumerate}

\subsubsection{洋务运动失败的原因}

中日甲午战争一役,洋务派经营多年的北洋海军全军覆没,标志着以 “自强”、“求富” 为目标的洋务
运动的失败。洋务运动失败的原因如下:
\begin{enumerate}[label=(\arabic*), itemsep=0pt]
    \item \textbf{具有封建性};
    \item \textbf{对列强具有依赖性};
    \item \textbf{管理具有腐朽性}。
\end{enumerate}

\begin{mdframed}[frametitle={洋务运动:一次封建统治阶级的自救运动}]
    \begin{enumerate}[label=\textup{\arabic*}${}^\circ$, itemsep=0pt]
        \item 指导思想:中学为体、西学为用;
        \item 失败的原因:
            \begin{enumerate}[itemsep=0pt]
                \item 阶级局限性:\textbf{具有封建性};
                \item \textbf{对列强具有依赖性};
                \item \textbf{管理具有腐朽性}。
            \end{enumerate}
    \end{enumerate}
\end{mdframed}

\subsection{复习参考题}

\begin{example}
    第二次鸦片战争后,清朝统治集团内部一部分人震惊于列强的 “船坚炮利”,主张学习西方以求
    “自强”,洋务运动由此兴起,在洋务派兴办的洋务事业中,当时国内最大的兵工厂是 
    \underline{\qquad \qquad \qquad}。
    \newline
    \begin{tabular}{p{0.48\textwidth} p{0.45\textwidth}}
    A. 安庆内军械所; & B. 天津机器局; \\
    C. 福州船政局; & D. 上海江南制造总局 
    \end{tabular}
\end{example}

\begin{example}[2018 年,第 10 题]
    洋务运动时期,洋务派兴办了一些民用企业,这些企业除少数采取官办或官商合办的方式外,
    多数都采取官督商办的方式。这些官督商办的民用企业基本上是 \underline{\qquad \qquad \qquad}。
    \begin{tabular}{p{0.48\textwidth} p{0.45\textwidth}}
    A. 封建主义性质的企业; & B. 半封建性质的企业;  \\
    C. 资本主义性质的企业; & D. 资本主义性质的企业 
    \end{tabular}
    \begin{sol}
        这些官督商办的民用企业,虽然受官僚的控制,
        发展受到很大限制,但基本上是资本主义性质的近代企业。因此本题选 C。
    \end{sol}
\end{example}

\section{维新运动的兴起和夭折}

\subsection{戊戌维新运动的开展}

在内忧外患的冲击和中西文化的碰撞过程中,人们形成了一个共识:要救国,只有维新,
要维新,只有学外国。

\subsubsection{维新派倡导救亡和变法的活动}

\begin{enumerate}[label=(\arabic*), itemsep=0pt]
    \item 向皇帝上书:如著名的 “公车上书”。
    \item 著书立说:如《新学伪经考》、《孔子改制考》等。
    \item 介绍外国变法的经验教训。
    \item 办学会,如强学会、南学会、保国会等;
    \item 设学堂,如广州万木草堂、长沙时务学堂等;
    \item 办报纸,如《时务报》、《国闻报》、《湘报》等。
\end{enumerate}
维新派以各种方式宣传变法主张,制造维新舆论,培养变法骨干,组织革新力量,而重点则放在
争取光绪皇帝及其周围的帝党官员的支持上,希望通过他们\textbf{自上而下}地实行变法主张。

\subsubsection{维新派与守旧派的论战}

论战主要围绕以下三个问题展开:
\begin{enumerate}[label=(\arabic*), itemsep=0pt]
    \item 要不要变法;
    \item 要不要兴民权、设议院,实行君主立宪;
    \item 要不要废八股、改科举和兴西学。
\end{enumerate}
维新派和守旧派的这场论战,实质上是资产阶级思想与封建主义思想在中国的第一次正面交锋。
进一步开阔了新型知识分子的眼界,解放了人们长期受到束缚的思想,西方资产阶级社会
政治学说在中国得到进一步的传播,戊戌变法运动的帷幕随之拉开。




\begin{mdframed}[frametitle={救亡图存的各种尝试}]
    \begin{table}[H]
        \centering
        \begin{tabular}{c c c c}
            \toprule
            人物/历史事件 & 阶级属性 & 学习军事、科学、技术 & 学习思想、制度、文化 \\
            \midrule
            魏源、林则徐等人 & 封建地主阶级 & T & F \\
            洋务派/洋务运动 & 封建地主阶级 & T & F \\
            王韬/早期维新派 & 封建地主阶级 & T & T \\
            维新派/戊戌维新 & 资产阶级维新派 & 君主立宪制 & T \\
            辛亥革命 & 资产阶级革命派 & 民主共和制 & T \\
            \bottomrule
        \end{tabular}        
    \end{table}
\end{mdframed}

\subsubsection{昙花一现的百日维新}


“百日维新” 在政治方面的内容是:
\begin{enumerate}[label=(\arabic*), itemsep=0pt]
    \item 改革行政机构,裁撤闲散、重叠机构;
    \item 裁汰冗员,澄清吏治,提倡廉政;
    \item 提倡向皇帝上书言事;
    \item 准许旗人自谋生计,取消他们享受国家供养的特权。
\end{enumerate}
“百日维新” 期间颁布的政令使资产阶级享受一定程度的政治权利,促进了资本主义工商业
的发展。因此,戊戌维新运动是一场\textbf{资产阶级性质的改良运动}。但是,
在光绪皇帝发布的新政诏令中,并没有采纳维新派多次提出的开国会、定宪法等政治主张。
这些政令和措施\textbf{并未触及封建制度的根本},所要推行的是一种十分温和的不彻底的改良方案。

1898 年 9 月 21 日,“戊戌六君子” 惨遭杀害。“百日维新” 如昙花一现,只经历了
103 天就夭折了。除京师大学堂(今北京大学的前身)被保留下来以外,其余新政措施大多被废除。

\subsection{戊戌维新运动的意义和教训}

\subsubsection{戊戌维新运动的意义}

\begin{enumerate}[label=(\arabic*), itemsep=0pt]
    \item 戊戌维新运动是一次\textbf{爱国救亡运动};
    \item 戊戌维新运动是一场\textbf{资产阶级性质的政治改良运动};
    \item 戊戌维新运动是一场\textbf{思想启蒙运动}。
\end{enumerate}

\subsubsection{戊戌维新运动失败的原因和教训}

戊戌维新运动失败的原因主要是维新派自身的局限性(阶级局限性+客观原因)
和以慈禧太后为首的强大的守旧势力的反对。具体如下:
\begin{enumerate}[label=(\textup{\arabic*}), itemsep=0pt]
    \item 民族资产阶级力量弱小(客观原因);
    \item 维新派的局限性突出表现为以下三个方面:
    \begin{enumerate}[label=\roman*., itemsep=0pt]
         \item \textbf{不敢否定封建主义}:
            \begin{itemize}[itemsep=0pt]
                \item 政治上,不敢根本否定封建君主制度;
                \item 经济上,虽然要求发展民族资本主义,却未
                触及封建主义的经济基础——封建土地所有制;
                \item 思想上,虽然提倡学习西学,却仍要打折孔子的旗号,接古代圣贤之名
                “托古改制”。
            \end{itemize}
         \item \textbf{对帝国主义抱有幻想};
         \item \textbf{惧怕人民群众}:戊戌维新运动是一场走精英路线的运动。
    \end{enumerate}
\end{enumerate}
戊戌维新运动的失败说明了在半殖民地半封建的旧中国,企图通过统治者走自上而下的改良道路,
实现国家的独立、民主、富强是根本行不通的,
必须用革命的手段,推翻帝国主义、封建主义联合统治的半殖民地半封建的社会制度。
“戊戌六君子”血的教训,促使一部分人放弃了改良主张,开始走上了革命的道路。

\begin{mdframed}[frametitle={戊戌维新:中国民族资产阶级的第一舞}]
    \begin{enumerate}[label=\textup{\arabic*}${}^\circ$, itemsep=0pt]
        \item 历史意义:
        戊戌维新运动是一次爱国救亡运动、资产阶级性质的政治改良运动、思想启蒙运动;
        \item 资产阶级维新派的阶级局限性是戊戌维新运动失败的根本原因,其主要表现在:
        \begin{enumerate}[label=\roman*., itemsep=0pt]
            \item \textbf{不敢否定封建主义}:
            \item \textbf{对帝国主义抱有幻想};
            \item \textbf{惧怕人民群众}:戊戌维新运动是一场走精英路线的运动。
       \end{enumerate}
    \end{enumerate}
\end{mdframed}

\subsection{复习参考题}

\begin{example}
    中国民族资产阶级登上政治舞台的第一次表演是 \underline{\qquad \qquad \qquad}。
    \newline
    \begin{tabular}{p{0.48\textwidth} p{0.45\textwidth}}
    A. 洋务运动; & B. 戊戌维新运动; \\
    C. 辛亥革命; & D. 五四运动 
    \end{tabular}
\end{example}

\begin{example}
    (多选)在向西方学习的过程中,戊戌维新运动与洋务运动的不同点在于 \underline{\qquad \qquad \qquad}。
    \newline
    \begin{tabular}{p{0.48\textwidth} p{0.45\textwidth}}
        A. 学习西方的科学技术;& B. 学习西方的政治制度; \\ 
        C. 批判封建的伦理道德;& D. 主张采取君主立宪制
    \end{tabular}
\end{example}

% \section{拓展提高:晚清通史}

% \subsection{清代通商与外政制度}

% \subsubsection{通商与外政制度的概念意义}

% 首先,什么是外政制度?我们在此将\textbf{清朝固有的对外关系制度称为外政制度}。原因有二:
% 其一,清朝在成立外务部以前,并没有西方近代式的专责外交机构;其二,清朝与西方近代民族国家
% (nation-state)在构造上截然不同,因而无法透过西方近代式的政府外交机构或组织的视角来理解
% 清朝中国在对外关系上的构造、制度以及思想概念的运作情形。所以为了区别清朝固有的对外关系制度和
% 西方近代的外交制度的不同,我们将清朝固有对外关系制度称为外政制度。



% \newpage
% \subsection{十九世纪前期中西关系的演变}
% \subsection{近代的开端:鸦片战争}
% \subsection{条约制度的建立及其影响}
% \subsection{中华宗藩体系的挫败与转型}
% \subsection{太平天国的兴起与败亡}
% \subsection{洋务运动与早期现代化}
% \subsection{派系分合与晚清政局}
% \subsection{从甲午战争到戊戌变法}
% \subsection{义和团运动与二十世纪中国}
% \subsection{十年新政与清朝覆灭}

% \section{拓展提高:晚清专题史}

% \subsection{立宪运动与民间宪政诉求}
% \subsection{帝制面临的挑战:新政的制度困境和伦理转换}
% \subsection{现代经济的起步:晚清的经济发展}
% \subsection{悸动的农村与农民}
% \subsection{二十世纪初的收回利权运动}
% \subsection{清季人口与社会}
% \subsection{大变局下的生活世界:洋货流行与生活启蒙}
% \subsection{晚清士绅阶层的结构性变动}
% \subsection{中西学之争:从科举、学校到学堂}
% \subsection{晚清海防与塞防之争}
% \subsection{“过渡时代” 的脉动:晚清思想发展之轨迹}
% \subsection{天下、国家与价值重构:启蒙的历程}
% \subsection{族群、文化与国家:晚清的国族想象}
% \subsection{中国士人与西方政体类型知识 “概念工程” 的创造与转化}
% \subsection{译书与西学东渐}
% \subsection{晚清台湾的社会经济与文化发展}

\clearpage
\thispagestyle{empty}

\chapter{辛亥革命与君主专制制度的终结}
\thispagestyle{empty}

\quad\quad 
了解辛亥革命爆发的时代背景和阶级基础,了解三民主义纲领的基本内容、意义和局限性,
深刻认识辛亥革命的历史意义、失败的原因和经验教训是本章学习的重点。认识辛亥革命
的历史必然性,正确、客观地理解三民主义,深刻理解资产阶级共和国的方案在中国行不
通的根本原因是本章学习的难点。

\section{举起近代民族民主革命的旗帜}
\subsection{辛亥革命爆发的历史条件}

戊戌维新运动失败后,以孙中山为代表的资产阶级革命派在中国掀起了一场资产阶级
革命运动。这场革命的发生没事当时民族危机加深、社会矛盾激化的结果,具有历史的必然性。
\begin{itemize}[itemsep=0pt]
    \item \textbf{民族危机加深,社会矛盾激化}。随着晚清政局的演变,人民群众已经不能照旧
    生活下去了。
    \item \textbf{清末 “新政” 的破产}。
    清末新政的根本目的是延续其反动统治。
    清政府于 1906 年宣布 “预备仿行宪政”,并于 1908 年颁布了《钦定宪法大纲》,
    但又规定了 9 年的预备立宪期限。预备立宪并没有能够挽救清王朝,反而激化了社会矛盾,
    加重了危机。
    \item \textbf{资产阶级革命派的阶级基础和骨干力量的逐渐成熟}。资产阶级革命派的
    骨干是一批资产阶级、小资产阶级知识分子。
\end{itemize}
\begin{remark}
    从资产阶级革命派的骨干是一批资产阶级、小资产阶级知识分子可以看出,辛亥革命是一场
    走精英路线的运动,不能充分地发动人民群众,这是它最终被窃取革命果实的原因之一。
\end{remark}

\subsection{资产阶级革命派的活动}

\subsubsection{孙中山与资产阶级民主革命的开始}

1894 年 11 月,孙中山到檀香山组建了\textbf{第一个革命团体兴中会},立誓
“驱除鞑虏,恢复中国,创立合众政府”,以孙中山为首的资产阶级革命派在踏上革命道路之时,
就高举起民主革命的旗帜,并选择了以武装起义推翻清王朝统治的斗争方式。这是中国资产阶级革命派
与改良派的根本不同之处。




\subsubsection{资产阶级革命派的宣传与组织工作}

历史进入 20 世纪,随着一批新兴知识分子的产生,各种宣传革命的书籍报刊纷纷涌现,
民主革命思想得到广泛传播在资产阶级革命思想的传播过程中,资产阶级革命团体也在各地
次第成立。

\begin{table}[H]
    \centering
    \caption{资产阶级革命派的宣传与组织工作}
    \begin{tabular}{p{0.2\textwidth}|p{0.1\textwidth}|p{0.3\textwidth}|p{0.3\textwidth}}
        \toprule
        人物 & 时间 & 事件/作品 & 宣传的思想 \\ 
        \hline
        章炳麟 & \multirow{3}{*}{1903 年} & 《驳康有为论革命书》
        & \multirow{3}{0.3\textwidth}{号召人民奋起革命,推翻清政府这个 “洋人的朝廷”。}\\
        邹容 &  & 《革命军》&\\
        陈天华 & & 《警世钟》、《猛回头》&\\
        \hline
        & 1904 年 & 华兴会、科学补习所、光复会、岳王会等革命团体出现 &
        这些革命团体的成立为革命思想的传播及革命运动的发展提供了不可缺少的组织力。 \\
        \hline
        孙中山、黄兴、宋教仁等人 & 1905 年 8 月 20 日 & 在日本东京成立中国同盟会 
        & 这是近代中国第一个领导资产阶级
        革命的全国性政党,它的成立标志着中国资产阶级民主革命进入了一个新的阶段。 \\
        \bottomrule
    \end{tabular}    
\end{table}

\subsection{三民主义的提出}

同盟会的政治纲领是 “驱除鞑虏,恢复中华,创立民国,平均地权”。
1905 年 11 月,在同盟会机关报《民报》发刊词中,孙中山将同盟会的纲领概括为
三大主义, 即\textbf{民族}主义、\textbf{民权}主义、\textbf{民生}主义,
后被称为三民主义(本书为了与后文中将会提到的 “新三民主义” 作区分,此后将这里提出的
三民主义称为 “老三民主义” 或者 “旧三民主义”)。

\begin{table}[H]
    \centering
    \caption{老三民主义的内容及其局限性}
    \begin{tabular}{c p{0.4\textwidth} p{0.3\textwidth}}
        \toprule
        & 主要内容 & 局限性 \\
        \hline
        民族主义 & {\kaishu 驱除鞑虏,恢复中华}
        \newline 一是要以革命手段推翻清朝政府,改变它一贯推行的民族歧视和民族压迫政策;
        二是建立中华民族 “独立的国家” & 没有从正面鲜明地提出反对帝国主义的主张;
        放松了对汉族封建势力的警惕,从而给了这部分人后来从内部和外部破坏革命
        以可乘之机。简单地来说就是不反帝、反封不彻底(本质上是只反满)。\\
        \hline
        民权主义 & {\kaishu 创立民国}
        \newline 即推翻封建君主专制制度, 建立资产阶级民主共和国。
        & 民权主义归根到底是建立资产阶级专政的国家,却忽略了广大劳动群众在国家中的地位,
        因而人民的民主权利很难得到真正的保证。\\ 
        \hline 
        民生主义 & {\kaishu 平均地权}
        \newline 即核定全国土地的地价,其现有之地价,仍属原主;
        革命后的增价则归国家,为国民共享。
        \newline 这就是说,一块土地的价格
        \[ x \xrightarrow{\text{革命后}} x + \Delta x, \]
        其中原价 $x$ 归原地主,增量 $\Delta x$ 归国家为公民共享,
        这一主张存在一定的土地国有化的思想。
        & 没有正面触及封建土地所有制,不能满足广大农民的土地要求,在革命中难以成为
        发动广大工农群众的理论武器。\\ 
        \bottomrule
    \end{tabular}
\end{table}

% \begin{mdframed}[frametitle={旧三民主义的局限性:不反帝、反封建不彻底,难以保证人民的民主权利。}]
%     \begin{enumerate}
%         \item 民族主义:驱除鞑虏、恢复中华。\textbf{};
%         没有把汉族军阀、官僚、地主作为革命对象,
%         \item 民权主义:创立民国。虽然强调了要建立民主共和国,却忽略了广大劳动群众在
%         国家中的地位,\textbf{难以使人民的民主权利得到真正的保证}。
%         \item 民生主义:平均地权。\textbf{没有正面出击封建土地所有制},不能满足广大农民的土地要求。
%     \end{enumerate}
% \end{mdframed}

\subsection{关于革命与改良的辩论}

1905 至 1907 年间,围绕中国究竟是采用革命手段还是改良方式这个问题,
资产阶级革命派与改良派分别以《民报》《新民丛报》为主要舆论阵地展开了一场论战,
这场论战的焦点是要不要以革命手段推翻清王朝,革命派的主要观点如下:
\begin{itemize}[itemsep=0pt]
    \item \textbf{要不要以革命手段推翻清王朝:}清政府已经沦为帝国主义的 “鹰犬”,爱国必须革命。
    只有通过革命,才能避免 “瓜分之祸”,获得民族独立和进步。进行革命,固然会有牺牲。但是,\textbf{不进行革命,
    而容忍清王朝在中国的统治的话,中国人民将长期遭受痛苦和更大的牺牲}。人们在革命过程中所付出的努力,
    乃至做出的牺牲,是以换取历史进步为补偿的。革命本色正是为了建设,破坏与建设是革命的两个方面。
    因此,\textbf{必须以革命手段推翻清王朝}。
    \item \textbf{要不要推翻帝制,实行共和:}不是 “国民恶劣” 而是 “政府恶劣”。民主共和是大势所趋、人心所向。
    \textbf{拯救中国与建设中国都必须取法乎上,直接推行民主制度,而不能以国民素质低劣为借口,搞君主立宪或开明专制}。
    只有 “兴民权改民主”,才是中国的唯一出路。中国国民自有颠覆专制制度,建立民主共和的能力。
    因此,\textbf{必须推翻帝制,实行共和}。
    \item \textbf{要不要进行社会革命:}中国存在严重的 “地主强权” 和 “地权失平” 现象,必须通过平均地权以实现土地国有,
    在进行政治革命的同时实现社会革命,才能避免贫富不均等社会问题的出现。因此,\textbf{必须进行社会革命}。
\end{itemize}
这场论战划清了革命与改良的界限,传播了民主革命思想,促进了革命形势的发展,
但也暴露了革命派思想理论方面的弱点。这些理论和认识的局限不可避免地会影响辛亥革命的
进程和结局。

\subsection{复习参考题}

\begin{example}
    近代中国历史上的第一个革命团体是\textbf{兴中会},
    第一个领导资产阶级革命的全国性政党是\textbf{中国同盟会}。
\end{example}
\begin{example}
    兴中会的口号是\textbf{“驱除鞑虏,恢复中国,创立合众政府”},中国同盟会的政治纲领是
    \textbf{“驱除鞑虏,恢复中华,创立民国,平均地权”}。
\end{example}
\begin{example}
    资产阶级革命派与改良派的根本不同之处在于资产阶级革命派选择了
    以\textbf{武装起义}推翻清王朝统治的斗争方式。
\end{example}

\begin{example}
    1905 年至 1907 年间,围绕中国社会变革问题,资产阶级革命派与改良派分别以《民报》、
    《新民丛报》为主要的舆论阵地,展开了一场论战。这场论战 \underline{\qquad \qquad \qquad}。
    \begin{tasks}[label={\Alph*.}](1)
        \task 聚焦要不要以革命手段推翻清王朝;
        \task 实际上是资产阶级思想和封建主义思想的交锋;
        \task 传播了民主革命思想,促进了革命形势的发展;
        \task 暴露了革命派在思想理论方面的弱点。
    \end{tasks}
\end{example}

\section{辛亥革命与中华民国的建立}

\subsection{辛亥革命的爆发与清王朝的覆灭}
\subsection{中华民国的建立}
\subsection{辛亥革命的历史意义}
\subsection{复习参考题}

\begin{example}
    (2015 年)毛泽东在谈到辛亥革命时指出,辛亥革命有它胜利的地方,
    也有它失败的地方。“辛亥革命把皇帝赶跑,这不是胜利了吗?说它失败,
    是说辛亥革命只把一个皇帝赶跑”。毛泽东这里所说的 “只把一个皇帝赶跑”
    是指 \underline{\qquad \qquad \qquad}。
    \begin{tasks}[label={\Alph*.}](2)
        \task 没有推翻帝制
        \task 反帝反封建的革命任务没有完成
        \task 孙中山没有继续革命
        \task 袁世凯窃取了胜利果实
    \end{tasks}
    \begin{cmt}
        本题选择 B 选项。
    \end{cmt}
\end{example}

\newpage

\begin{mdframed}[frametitle={两次辩论}]
    
\end{mdframed}

\begin{mdframed}[frametitle={中华民国的建立与《中华民国临时约法》的局限性}]
    中国历史上第一部具有资产阶级共和国性质的法典是《中华民国临时约法》。这部临时约法的
    局限性体现在:
    \begin{enumerate}[itemsep=0pt]
        \item 企图用承认清政府与列强签订的一切不平等条约和清政府所欠的一切外债,
        来换取列强承认中华民国;
        \item 没有提出任何可以满足农民土地要求的政策和措施。
    \end{enumerate}[itemsep=0pt]
\end{mdframed}

\begin{mdframed}[frametitle={辛亥革命的历史意义}]
\begin{enumerate}[itemsep=0pt]
    \item 推翻了清王朝的统治,结束了中国持续两千多年的封建君主专制制度,建立了中国历史上第一个资产阶级共和政府,使民主共和的观念深入人心;
    \item 推动了中国人民的思想解放、推动了中国的社会变革;
    \item 在一定程度上打击了帝国主义的侵略势力,推动了亚洲各国民族解放运动的高涨。
\end{enumerate}
\end{mdframed}

\begin{mdframed}[frametitle={辛亥革命失败的原因和教训}]
    从根本上说,在帝国主义时代,在半殖民地半封建的中国,资本主义的建国方案是行不通的。
    从主观方面来说,辛亥革命之所以失败,在于它的领导者资产阶级革命派本身存在着许多弱点
    和错误:
\begin{enumerate}[itemsep=0pt]
    \item 没有提出彻底的反帝反封建革命纲领;
    \item 不能充分发动和依靠人民群众;
    \item 不能建立坚强的革命政党,作为团结一切革命力量的强有力的核心。
\end{enumerate}
\end{mdframed}

\begin{mdframed}[frametitle={民族资产阶级的两面性}]
    \begin{enumerate}[itemsep=0pt]
            \item 进步性、革命性;
            \item 软弱性、妥协性。
    \end{enumerate}
    \end{mdframed}



% \begin{mdframed}[frametitle={救亡图存的各种尝试}]
%     \begin{table}[H]
%         \centering
%         \begin{tabular}{c c c c}
%             \toprule
%             人物/历史事件 & 阶级属性 & 是否学习军事、科学、技术 & 是否学习思想、制度、文化 \\
%             \midrule
%             魏源、林则徐等人 & 封建地主阶级 & T & F \\
%             洋务派/洋务运动 & 封建地主阶级 & T & F \\
%             王韬/早期维新派 & 封建地主阶级 & T & T \\
%             戊戌维新 & 资产阶级维新派 & T & 君主立宪制 \\ 
%             辛亥革命 & 资产阶级革命派 & T & 民主共和制 \\
%             \bottomrule
%         \end{tabular}        
%     \end{table}
% \end{mdframed}

% \subsection{辛亥革命与中华民国的建立}
% \subsubsection{辛亥革命的爆发与清王朝的覆灭}
% \subsubsection{中华民国的建立}

% 辛亥革命作为中国近代史上的一次重大事件,具有重大的历史意义:
% (1) \textbf{推翻了清王朝的统治},沉重打击了中外反动势力,是反动统治者在政治上乱了阵脚;
% (2) \textbf{结束了中国持续两千多年的封建君主专制制度},建立了中国历史上第一个资产阶级共和政府,
% 使民主共和的观念开始深入人心;
% (3) \textbf{推动了中国人民的思想解放};
% (4) \textbf{推动了中国的社会变革},辛亥革命促使社会经济、思想习惯和社会风俗等方面发生了积极变化;
% (5) 不仅在一定程度上打击了帝国主义的侵略势力,而且\textbf{推动了亚洲各国民族解放运动的高涨}。

% 中国历史上第一部具有资产阶级共和国性质的法典是\textbf{《中华民国临时约法》}。
% 《中华民国临时约法》是革命的产物,带有鲜明的革命性、民主性。其革命性表现为:
% 以根本大法的形式废除了两千多年的封建君主专制制度,确认了资产阶级共和国的政治制度。
% 其民主性表现为:规定 “中华民国之主权属于国民全体”,而 “以参议院、临时大总统、国务员、法院行使其统治权”;
% 参议院为立法机关,行使立法权,参议院还有弹劾大总统和国务员的权利;中华民国国民一律平等。

% \subsection{北洋军阀统治与旧民主主义革命的失败}
% \subsubsection{封建军阀专制统治的形成}

% 毛泽东指出,辛亥革命有它胜利的地方,也有它失败的地方。“辛亥革命把皇帝赶跑,这不是胜利了吗?
% 说它失败,是说辛亥革命只把一个皇帝赶跑。” 毛泽东这里所说的 “只把一个皇帝赶跑”,是指
% 中国虽然结束了封建帝制,但仍旧在帝国主义和封建主义的压迫之下,反帝反封建的革命任务并没有完成。

% \subsubsection{旧民主主义革命的失败}

% 辛亥革命失败的原因是多方面的。从根本上说,在帝国主义时代,在半殖民地半封建的中国,
% \textbf{资本主义的建国方案是行不通的}。帝国主义和封建势力相互勾结扼杀了这场革命。
% 从主观方面来说,辛亥革命的领导者是资产阶级,而资产阶级革命派本身存在着许多弱点和错误:
% 如\textbf{没有提出彻底的反帝反封建的革命纲领、不能充分发动和依靠人民群众、不能建立坚强的革命政党,
% 作为团结一切革命力量的强有力的核心};等等。

% 辛亥革命的失败表明,\textbf{资产阶级共和国的方案不能够救中国},中国人民需要进行新的探索,为中国谋求新的出路。

% \section{章节自主测试题}

% \section{拓展提高}

% \subsection{辛亥革命:“低烈度” 与大业绩}
% \subsection{北洋政治的 “乱” 与 “治”}
% \subsection{北洋外交的成败}
% \subsection{训政框架下的国民政府}



\newpage
\thispagestyle{empty}

\chapter{中国共产党成立和中国革命新局面}
\thispagestyle{empty}

\quad\quad 新文化运动带来了思想解放的潮流,十月革命将马克思主义的春风吹到中国。
十月革命后,中国的先进知识分子经过曲折的探求选择了马克思主义。中国共产党的诞生,
是近代中国社会发展的必然结果,是马克思列宁主义和中国工人运动紧密结合的产物,是开天辟地的大事变。

\textbf{五四运动与新民主主义革命的开端},\textbf{中国共产党的成立及其意义},
\textbf{伟大建党精神的内涵}和\textbf{中国共产党与国民革命}是本章学习的重点内容;
\textbf{中国的先进分子对马克思主义的选择}是本章学习的难点。

% \section{内容精华}

\section{新文化运动和五四运动}

\subsection{新文化运动与思想解放的潮流}

\subsubsection{新文化运动的兴起及其意义}
\subsubsection{五四以前新文化运动的局限}

\subsection{十月革命与马克思主义在中国的初步传播}

\subsubsection{十月革命对中国先进分子的影响}
\subsubsection{李大钊率先在中国举起马克思主义旗帜}

{\captionof{table}{李大钊的文章} % 表格标题
\label{李大钊的文章} % 交叉引用标签
\begin{longtable}{p{6em}p{10em}p{14em}p{10em}}
    \hline
    % 表头
    \textbf{时间} & \textbf{文章} & \textbf{意义} & \textbf{备注} \\
    \hline
    \endhead
    \hline
    \endfoot

    % 表格内容
    1918 年 7 月 & 《法俄革命之比较观》
    & 指出 “俄罗斯之革命是二十世纪初期之革命,是立于社会主义之上的革命”,
    向中国人民第一次正确地阐述了十月革命的本质。 \\
    \hline
    1918 年 11 月 & 《庶民的胜利》 & \multirow{2}{14em}{
        深刻揭露了第一次世界大战的本质, 热情歌颂了十月革命和
        布尔什维主义的胜利,欢呼 “将来的环球,必是赤旗的世界”。
    }\\ 
    1918 年 12 月 & 《布尔什维主义的胜利》 & \\ \\
\end{longtable}}

% 早期马克思主义者的队伍主要由三部分组成:\textbf{新文化运动的精神领袖},
% \textbf{五四爱国运动的左翼骨干},以及\textbf{原中国同盟会会员、辛亥革命时期的活动家}。

\subsection{五四运动:新民主主义革命的开端}

\subsubsection{五四运动的爆发}
\subsubsection{五四运动的历史特点和意义}


\begin{mdframed}[frametitle={五四运动:中国新民主主义革命的开端}]
    \begin{enumerate}[itemsep=0pt]
        \item 五四运动的直接导火线是\textbf{巴黎和会上中国外交的失败}。
        \item 五四运动是中国旧民主主义革命走向新民主主义革命的转折点,
        在近代以来中华民族追求民族独立和发展进步的历史进程中具有里程碑意义;
        五四运动孕育了\textbf{以“爱国、进步、民主、科学”为主要内容的伟大的五四精神,其核心是爱国主义}。
    \end{enumerate}
\end{mdframed}

\begin{mdframed}[frametitle={伟大建党精神:中国共产党的精神之源}]
    \begin{enumerate}[itemsep=0pt]
        \item \textbf{坚持真理、坚守理想},就是坚持马克思主义的科学真理,坚守共产主义远大理想和中国特色社会主义共同理想。这是对中国共产党人理想信念和价值追求的集中表达。
        \item \textbf{践行初心、担当使命},就是坚持\textbf{为中国人民谋幸福、为中华民族谋复兴的初心和使命}。这是对中国共产党人历史责任和时代使命的集中表达。
        \item \textbf{不怕牺牲、英勇斗争},就是始终保持斗争精神、顽强意志、优良作风,毫无畏惧地面对一切困难和挑战,坚定不移地开辟新天地。这是对中国共产党人精神风范和意志品质的集中表达。
        \item \textbf{对党忠诚、不负人民},就是无条件地对党的信仰忠诚、对党组织忠诚、对党的理论和路线方针政策忠诚,始终坚持全心全意为人民服务的根本宗旨。
    \end{enumerate}
\end{mdframed}

% \subsubsection{中国先进知识分子为什么和怎样选择了马克思主义?}

% 新文化运动期间,先进知识分子中的一些人在宣传西方资产阶级民主主义时,就已经开始对它有所怀疑和保留。

% 先进知识分子在民主科学思想传播中经常遭遇挫折,如作为第一次世界大战战胜国的中国却要将山东权益让给日本,
% 这使得他们对资产阶级共和国方案在中国的可行性产生了极大的怀疑。

% 十月革命的推动使先进知识分子从中看到了民族解放的新希望,给予中国先进分子以新的革命方法的启示。

% 五四运动后,中国工人阶级登上历史舞台,这使中国先进知识分子对马克思主义运用于中国革命的前景产生了极大的希望。

% \subsection{马克思主义的广泛传播与中国共产党的诞生}

% \subsubsection{中国早期马克思主义思想运动}
% \subsubsection{马克思主义与中国工人运动的结合}
\subsection{复习参考题}

\begin{example}
    1921 年 7 月 23 日,中国共产党第一次全国代表大会在上海举行,
    从此,中国人民有了一个先进的坚强的政党作为凝聚自己力量的领导核心。
    下列关于中共一大的说法错误的是 \underline{\qquad \qquad \qquad}。
    \begin{tasks}[label={\Alph*.}](2)
        \task 决定首先集中精力组织工人
        \task 提出了反帝反封建的民主革命纲领
        \task 提出了以无产阶级军队推翻资产阶级
        \task 选举了由陈独秀、张国焘、李达组成的中央局
    \end{tasks}
    \begin{cmt}
        反帝反封建的民主革命纲领是在中共二大第一次提出的,不是在中共一大。
        因此本题选择 B 选项。
    \end{cmt}
\end{example}

% 中国共产党的宗旨是\textbf{全心全意为人民服务}。

% 中国共产党人的初心和使命是\textbf{为中国人民谋幸福,为中华民族谋复兴}。\textbf{“不忘初心,牢记使命”是
% 激励中国共产党人不断前进的根本动力}。唯有不忘初心,方可告慰历史、告慰先辈,方可赢得民心、赢得时代,
% 方可善作善成、一往无前。

% 中国共产党成立后,第一次\textbf{提出了反帝反封建的民主革命纲领};开始\textbf{采取群众路线}的方法;\textbf{实行国共合作},
% 掀起大革命的浪潮。

% 中国共产党的成立是开天辟地的大事变。中国共产党的成立,\textbf{使中国革命有了坚强的领导核心、科学的指导思想和新的革命方法}。
% \textbf{深刻改变了近代以来中华民族发展的方向和进程、中国人民和中华民族的前途和命运和世界发展的趋势和格局}。

% \subsubsection{如何理解伟大建党精神的深刻内涵?}

% 伟大建党精神:
    
% 坚持真理、坚守理想;

% 践行初心、担当使命;

% 不怕牺牲、英勇斗争;

% 对党忠诚、不负人民。

% \subsection{中国革命的新局面}

% \subsubsection{民主革命纲领的制定和工农运动的发动}
% \subsubsection{国共合作和大革命的进行}

% 国共合作的政治基础是\textbf{新三民主义}。

% 大革命是在中国共产党提出的\textbf{反对帝国主义、反对军阀的政治口号}下进行的;
% 大革命是在\textbf{以国共合作为基础的统一战线}的组织形式下进行的;
% 大革命是近代中国历史上\textbf{空前广泛而深刻的群众运动};大革命的主要斗争形式是\textbf{革命战争}。

% \subsubsection{大革命的失败及其教训}

% 国民革命的胜利和失败深刻地揭示了在革命统一战线中坚持无产阶级领导权的重要性。

% “农村包围城市、武装夺取政权”理论,是对 1927 年大革命失败后中国共产党领导的红军和根据地斗争经验
% 的科学概括,致命了中国革命胜利的唯一正确道路,标志着中国化的马克思主义即毛泽东思想的初步形成。

% \section{章节自主测试题}

% \section{拓展提高}

% \subsection{国民党的派系与内争}
% \subsection{国民革命军的制度与战力}
% \subsection{革命的底层动员:中共早期农民运动的动员·参与机制}

\chapter{中国革命的新道路}
\thispagestyle{empty}

% 国民革命结束后,中国民主革命的任务没有完成。中国革命具有长期性、曲折性和不平衡性,
% 中国革命道路新理论是中国共产党人把马克思主义普遍真理与中国革命具体实践相结合的光辉典范。

% 国民革命失败后,\textbf{国民党政权的性质}、\textbf{八七会议}的主要内容及其意义、
% 农村包围城市、武装夺取政权\textbf{革命新道路的探索历程}及其主要内容、\textbf{古田会议}的主要内容、
% \textbf{20 世纪二三十年代党内连续出现 “左” 倾错误的原因
% 及其严重后果}、\textbf{遵义会议}的主要内容及其意义和\textbf{红军长征}的胜利与\textbf{长征精神}是本章学习的重点。
% 对\textbf{农村包围城市道路}的理解是本章学习的难点。

% \section{内容精华}

\section{中国共产党对革命新道路的探索}

\subsection{国民党在全国统治的建立}

\subsubsection{南京国民政府的成立}

1927 年 4 月 18 日,南京国民政府成立。在以后的一段时间里,出现了南京国民政府
与武汉国民政府对峙的局面。经过几番周折,实现了 “宁汉合流”。在暂时统一的基础上,
1928 年 4 月,国民党蒋、桂、冯、阎四派新军阀联合举行了讨伐北洋军阀张作霖的战争。
1928 年 6 月,奉系首领张作霖在退回关外途中,被日本人预埋的炸药炸死。

张学良于 1928 年 12 月 29 日从东北发出通告,宣布 “遵守三民主义,服从国民政府,
改易旗帜”,史称 “东北易帜”。北洋军阀不再作为独立的政治力量继续存在,国民党就在全
国范围内建立了自己的统治。

\subsubsection{国民党政权的性质}

蒋介石、汪精卫先后叛变革命,实行 “清党分共” 政策后,国民党已经不再是工人、农民、城市
小资产阶级和民族资产阶级的革命联盟,而是变成了一个由代表地主阶级、买办性的大资产阶级的
反动利益的反动集团所控制的政党。

% 帝国主义列强一度对国民党政府作出过一两项表面上的让步。这个政权曾经在一个时期内,使得
% 一些人尤其是民族工商业者产生过幻想,以为中国可能由此走上独立发展资本主义的道路。
% 1928 年至 1929 年间,中国民族工业有过短暂的繁荣。但民族资产阶级并没有成为中国的统治阶级,
% 民族工商业也并没有得到自由的发展。
% 国民党实行的是代表地主阶级、买办性的大资产阶级利益的一党专政和军事独裁统治。其特点包括:
% \begin{itemize}[itemsep=0pt]
%     \item 为了镇压人民和消灭异己的力量,国民党建立了庞大的军队和全国性特务系统;
%     \item 为了控制人民,禁止革命活动,国民党还大力推行保甲制度;
%     \item 为了控制舆论,剥夺人民的言论和出版自由,国民党还厉行文化专制主义。
% \end{itemize}
% 国民党政府主要就是通过这些方法,来维护帝国主义、封建主义、官僚资本主义的利益,
% 巩固自身统治的。

国民党政府的统治依然是地主阶级和买办性的大资产阶级的统治,同北洋军阀的统治没有本质的区别。
中国仍然是一个处在帝国主义和封建主义统治之下的半殖民地半封建社会,中国革命的对象依然是
帝国主义和封建主义,中国革命的性质也依然是反帝反封建的资产阶级民主革命。中国仍然迫切需要
一个反帝反封建的资产阶级民主革命。正确地认识这一问题是中国人民继续进行民主革命和制定
正确路线、方针、政策的基本依据和战略出发点。

% \subsubsection{土地革命战争的兴起}

% \textbf{八七会议}确定了土地革命和武装反抗国民党反动统治的总方针。从组织上确立了党对组织的领导,
% 建设无产阶级领导的新型人民军队的重要开端是\textbf{三湾改编}。\textbf{南昌起义}打响了武装反抗
% 国民党反动统治的第一枪,这是中国共产党独立领导革命战争、创建人民军队和武装夺取政权的开端。
% \textbf{井冈山革命根据地}的建立是实行工农武装割据的开始。

% \subsubsection{农村包围城市、武装夺取政权道路的开辟}

% 大革命失败后,中国革命转入低潮,以毛泽东为主要代表的中国共产党人,逐渐探索、开辟了革命新道路。

% 毛泽东不仅在实践中把革命的方向指向了农村,领导和开创了\textbf{井冈山革命根据地,点燃了工农武装割据的星星之火},
% 为中国革命开辟出了农村包围城市、武装夺取政权这样一条前任没有走过的正确道路,而且在理论上阐述革命的新道路,
% 在同党内教条主义等作斗争的同时,科学地概括了红军和根据地斗争经验,系统提出了农村包围城市、武装夺取政权的理论。

% \subsubsection{20 世纪二三十年代,中国共产党为什么连续出现 “左” 倾错误?}

% 八七会议后,党内一直存在的浓厚的 “左” 倾情绪始终没有得到认真的清理;
% 共产锅里对中国革命的错误指导;全党的马克思主义理论准备不足,缺乏经验,
% 对中国的历史状况和社会状况、中国革命的特点、中国革命的规律不甚了解,
% 不善于把马克思列宁主义与中国实际全面地、正确地结合起来。

\section{中国革命在曲折中前进}

\subsection{土地革命战争的发展及其挫折}

\subsubsection{农村革命根据地的建设}

1931 年 11 月,中华苏维埃第一次全国工农兵代表大会在江西省瑞金县叶坪村举行,
大会通过了\textbf{《中华苏维埃共和国宪法大纲》}以及土地法令、劳动法等文件。
中华苏维埃共和国实行工农兵代表大会制度。
\begin{remark}
    实行工农兵代表大会制度,而不是人民代表大会制度,意味着在那一个历史时期,我党
    所代表的阶级还是比较狭隘的。
\end{remark}

\begin{mdframed}[frametitle={《中国近现代史纲要》课程中出现的宪法}]
    到现在为止,我们遇到了三部宪法:
    \begin{enumerate}[label={\arabic*.}, itemsep=0pt]
        \item 钦定宪法大纲;
        \item 中华民国临时约法;
        \item 中华苏维埃共和国宪法大纲。
    \end{enumerate}
\end{mdframed}

\subsubsection{土地革命战争的严重挫折}

\begin{table}[H]
    \centering
    \caption{“左” 倾错误先后三次在党中央取得统治地位}
    \begin{tabular}{c c c p{0.4\textwidth}}
        \hline 
        错误 & 时间 & 代表 & 主要表现 \\
        \hline 
        “左” 倾盲动错误 & 1927.11—1928.4 & & 认为革命形势在不断高涨,
        盲目要求 “创造总暴动的局面” \\ 
        “左” 倾冒险主义 & 1936.6—1936.9 & 李立三 & 
        错误地认为中国革命乃至世界革命进入高潮,盲目要求举行全国暴动和集中红军力量
        攻打中心城市 \\ 
        “左” 倾教条主义 & 1931.1—1935.1 & 陈绍禹(王明) &  \\ 
        \hline
    \end{tabular}
\end{table}

\begin{itemize}[itemsep=0pt]
    \item 第三次 “左” 倾错误, 即 “左” 倾教条主义错误的主要表现是:
    \begin{enumerate}[itemsep=0pt, label={(\arabic*)}]
        \item 在革命性质和统一战线问题上,混淆民主革命与社会主义革命的界限,将反帝反封建与
        与反资产阶级并列,将民族资产阶级视为中国革命最危险的敌人,一味排斥和打击中间势力;
        \item 在革命道路问题上,继续坚持以城市为中心;
        \item 在土地革命问题上,提出坚决打击富农和 “地主不分田,富农分坏田” 的主张;
        \item 在军事斗争问题上,实行进攻中的冒险主义、防御中的保守主义、退却中的逃跑主义;
        \item 在党内斗争和组织问题上,推行宗派主义和 “残酷斗争,无情打击” 的方针。
    \end{enumerate}
    \item 中国共产党内屡次出现严重的 “左” 倾错误的原因是:
    \begin{enumerate}[itemsep=0pt, label={(\arabic*)}]
        \item 八七会议以后党内一直存在着的浓厚的 “左” 倾情绪始终没有得到认真的清理;
        \item 共产国际对中国共产党内部事务的错误干预和瞎指挥;
        \item 不善于把马克思列宁主义与中国实际全面地、正确地结合起来。
    \end{enumerate}
\end{itemize}

\subsection{遵义会议与中国革命的历史性转折}

\subsubsection{遵义会议的结果}

遵义会议是一次伟大的历史转折,这次会议集中解决了当时具有决定意义的\textbf{军事和组织问题},
肯定了毛泽东的正确军事主张。
\begin{remark}
    遵义会议解决了当时具有决定意义的军事和组织问题,但没有解决政治问题和思想问题。
    所谓政治问题,是指中国共产党与其他各政党、各阶级的关系问题,
    主要是统一战线的问题,
    解决政治问题的会议是瓦窑堡会议,以及《论反对日本帝国主义的策略》一文;
    解决思想问题是在延安整风运动。
\end{remark}



\subsubsection{对遵义会议的评价}

遵义会议是党的历史上一个生死攸关的转折点。这次会议\textbf{开始确立以毛泽东为主要代表的
马克思主义正确路线在当重要的领导地位},
开始形成以毛泽东同志为核心的第一代中央领导集体,
从而在极其危急的情况下挽救了中国共产党、挽救了中国工农红军、挽救了中国革命。

遵义会议的鲜明特点是\textbf{坚持真理、修正错误},\textbf{确立党中央的正确领导},\textbf{创造性地制定和实施符合中国革命特点的战略策略}。
遵义会议\textbf{开启了中国共产党独立自主解决中国革命实际问题的新阶段}。

遵义会议是党的历史上一个生死攸关的转折点,标志着中国共产党\textbf{在政治上开始走向成熟}。

\subsection{红军长征的胜利}

红军的长征宣告了国民党反动派消灭中国共产党和红军的图谋彻底失败,
宣告了中国共产党和红军肩负着民族希望胜利实现了北上抗日的战略转移,
实现了中国共产党和中国革命事业从挫折走向胜利的伟大转折。
\begin{remark}
    该考点仅在部分有纪念意义的年份作为选择题重点考察。
\end{remark}

% \subsubsection{红军长征胜利和迎接全民族抗战}

% 长征精神是
% \textbf{把全国人民和中华民族的根本利益看得高于一切,坚定革命的理想和信念,坚信正义事业必然胜利}的精神;
% \textbf{为了救国救民,不怕任何艰难险阻,不惜付出一切牺牲}的精神;
% \textbf{坚持独立自主、实事求是,一切从实际出发}的精神;
% \textbf{顾全大局、严守纪律、紧密团结}的精神;
% \textbf{紧紧依靠人民群众、同人民群众生死相依、患难与共、艰苦奋斗}的精神。

\subsection{总结历史经验,迎接全民族抗日战争}

\begin{table}[H]
    \centering
    \caption{中国共产党的思想路线}
    \begin{tabular}{p{0.25\textwidth}|p{0.4\textwidth}p{0.25\textwidth}}
        \hline 
        \textbf{事件/会议/文件} & \textbf{标志} & \textbf{备注} \\ 
        \hline 
        《反对本本主义》 & 萌芽、初步形成 & \\ 
        \hline 
        《矛盾论》、《实践论》 & 科学、系统地阐明了我党的思想路线 \\ 
        \hline
    \end{tabular}
\end{table}

\subsection{复习参考题}

\begin{example}
    (多选)1935 年 1 月中共中央政治局在长征途中召开的遵义会议,是党的历史上一个生死攸关的转折点。
    这次会议 \underline{\qquad \qquad \qquad}。
    \begin{tasks}[label={\Alph*. }](1)
        \task 全面讨论了政治路线的是非问题
        \task 解决了当时党内所面临的最迫切的组织问题和军事问题
        \task 结束了 “左” 倾教条主义错误在中央的统治
        \task 开始确立毛泽东在中共中央和红军的领导地位
    \end{tasks}
    \begin{sol}
        遵义会议没有解决政治问题和思想问题,故不选择 A 项,其余三项均可以选择,
        因此本题选 BCD。
    \end{sol}
\end{example}

% \subsection{中国苏维埃革命的源流}
% \subsection{国共分和的背景 \quad 经过与原因}

\newpage
\thispagestyle{empty}

\chapter{中华民族的抗日战争}

\section{日本发动企图灭亡中国的侵略战争}

\subsection{日本灭亡中国的计划及其实施}

\subsubsection{抗日战争的开始}

1931 年 9 月 18 日,日本发动九一八事变,武装侵占中国东北。九一八事变,
是中国抗日战争的起点,揭开了世界反法西斯战争的序幕。每年的 9 月 18 日这一天,
全国各地都会响起警报,勿忘国耻,警钟长鸣。

\begin{remark}
    中华民族的抗日战争,从大的方面来说,分为六年局部抗战和八年全面抗战,所谓的 “十四年” 抗战
    的说法便是来源于此。九一八事变,是六点局部抗战的起点。
\end{remark}

\subsubsection{抗日战争前期国民政府的态度}

蒋介石在 1931 年 7 月提出 “攘外必先安内” 的方针。

\subsubsection{占东北、侵华北到全面侵华战争}

1937 年 7 月 7 日爆发卢沟桥事变,又称七七事变,日本全面侵华战争由此开始。卢沟桥事变是中华民族全面抗战(又称全民族抗战)
的开始,开辟了世界反法西斯战争的东方主战场。

\begin{table}[H]
    \centering
    \caption{即要了解中国,又要了解世界——中华民族的抗日战争与世界反法西斯战争}
    \begin{tabular}{c|p{0.3\textwidth}p{0.3\textwidth}}
        \hline
        & \textbf{抗日战争} & \textbf{世界反法西斯战争} \\
        \hline 
        九一八事变(1931.9.18) & 中国抗日战争的起点 & 揭开了世界反法西斯战争的序幕 \\ 
        \hline 
        卢沟桥事变(1937.7.7) & 中华民族全面抗战的开始 & 开辟了世界反法西斯战争的东方主战场 \\ 
        \hline 
    \end{tabular}
\end{table}
\begin{remark}
    九一八事变与卢沟桥事变在世界反法西斯战争中的地位和意义是比较容易混淆的考点,
    在选择题命题时常作为干扰选项给出,需要引起注意。
\end{remark}

\subsection{残暴的殖民统治和中华民族的深重灾难}

日本对中国进行大规模侵略和在中国部分地区实行殖民统治,犯下了空前严重的罪行,
给中华民族造成了极为深重的灾难。
1937 年 12 月,日军占领南京后,制造了震惊中外的 “南京大屠杀”。除了制造惨绝人寰的
大屠杀外,日军还疯狂掠夺中国的资源与财富,强制推行奴化教育,肆意摧残中国文化。

\section{中国人民奋起反抗日本侵略者}

\subsection{中国共产党举起武装抗日的旗帜}
% \subsubsection{抗日救亡运动的兴起}

% 促进了中华民族觉醒,标志着中国人民抗日救亡运动新高潮的到来的运动是一二·九运动。

% \subsubsection{抗日民族统一战线的建立与全民族抗战的开始}

% 瓦窑堡会议提出党的基本策略任务是建立广泛的抗日民族统一战线。

% \subsection{抗日战争的正面战场}

% 以国民党军队为主体的正面战场的代表性战役有\textbf{淞沪、忻口、徐州、武汉四大会战}。

% \subsubsection{战略防御阶段的正面战场}
% \subsubsection{战略相持阶段的正面战争}

% \subsection{抗日战争的中流砥柱}

% \subsubsection{全面抗战的路线和持久战的战略总方针}
% \subsubsection{敌后战场的开辟与游击战争的发展}
% \subsubsection{坚持抗战、团结、进步的方针}

% \subsection{抗日民主根据地的建设}

% 抗日民主根据地是认真贯彻和实现中国共产党全面抗战路线、坚持抗战和争取胜利的坚强阵地。

% \subsubsection{“三三制” 的民主政权建设}

% 加强政权建设,是抗日根据地建设的首要的、根本的任务。

% 中国共产党提出,根据地政权是共产党领导的抗日民族统一战线性质的政权,
% 是一切赞成抗日又赞成民主的人们的政权,是几个革命阶级联合起来对于汉奸和反动派的民主专政。

% 抗日民主政府在工作人员分配上实行“三三制”原则,即\textbf{
%     共产党员、非党的左派进步分子和不左不右的中间派各占 1/3}。
% 这样做\textbf{可以容纳各方面的代表,团结一切赞成抗日又赞成民主的各阶级、阶层}。抗日民主政权普遍采取民主集中制,
% 各级抗日民主政权机构的领导人都经过人民选举产生。抗日民主政权努力发扬政治民主,保障人民的民主自由权利。

% \subsubsection{减租减息、发展生产}

% 各地抗日民主政权十分重视根据地的经济建设。\textbf{停止实行没收地主土地的政策,普遍实行减租减息政策,
% 以减轻农民所受的封建剥削,提高他们的抗日和生产的积极性}。同时,根据地实行交租交息,以利于联合地主阶级抗日。
% 抗日民主政权发动农民组织劳动互助,动员农民开荒地,修水利,帮助农民改良耕作技术,推广优良品种。

% 为克服根据地面临的严重困难局面,毛泽东提出了\textbf{“发展经济、保障供给”的经济工作和财政工作总方针},
% 发出了\textbf{“自己动手,丰衣足食”的号召},领导根据地军民开展了大生产运动。根据地军民终于战胜了困难,
% 农业生产和工商业都得到恢复和发展,为坚持抗战、争取胜利奠定了物质基础。


% \subsubsection{文化建设与干部教育}

% 全民族抗日战争开始后,中共中央及时作出大量吸收知识分子的决定,
% 把发展抗日的革命文化运动提上重要议事日程,中国人民抗日军事政治大学、
% 鲁迅艺术学院等一批干部学校和专门学校创办起来。各根据地还创办了大量的中、小学校,吸收农民子女入学。
% 哲学社会科学和自然科学研究也得到重视。

% \subsubsection{大后方的抗日民主运动和进步文化工作}

% \subsection{中国共产党的自身建设}

% \subsubsection{马克思主义中国化命题的提出}

% 1938年9月至11月,中国共产党在延安举行了\textbf{扩大的六届六中全会,毛泽东明确提出了“马克思主义的中国化”这一命题}。

% 毛泽东强调:“离开中国特点来谈马克思主义,只是抽象的空洞的马克思主义。因此,马克思主义的中国化,使之在其每一表现中带着中国的特性,即是说,按照中国的特点去应用它,成为全党亟待了解并亟须解决的问题。”

% 为了推进马克思主义中国化的事业,毛泽东向全党提出了普遍地深入地学习马克思列宁主义的理论,
% 学习我们的历史遗产并给以批判的总结和调查研究当前运动的特点及其规律性的任务。

% \subsubsection{新民主主义理论的系统阐明过程}

% 在 20世纪30 年代后期和40年代前期,为了将丰富的中国革命实际经验马克思主义化,以便更好地指导抗日战争和中国革命,毛泽东撰写了《〈共产党人〉发刊词》、《中国革命和中国共产党》、《新民主主义论》等一批重要的理论著作。

% 毛泽东首先揭示了中国半殖民地半封建社会的性质和主要特征,近代中国社会的主要矛盾和中国革命发生及发展的原因,
% 阐明了中国共产党领导的整个中国革命运动,是包括民主主义革命和社会主义革命两个阶段在内的全部革命运动。
% 1919年五四运动以后的中国民主革命,已经是无产阶级领导的人民大众的反帝反封建的新民主主义革命,
% 它的前途是社会主义。

% 毛泽东还阐明了中国共产党在新民主主义革命阶段的基本纲领。
% 政治上,推翻帝国主义和封建主义的压迫,建立一个以无产阶级为领导、以工农联盟为基础的各革命阶级联合专政的新民主主义共和国。这个共和国的政体是民主集中制的人民代表大会制度。
% 经济上,没收操纵国计民生的大银行、大工业、大商业归新民主主义国家所有,建立国营经济;没收地主阶级的土地归农民所有,并引导个体农民发展合作经济;
% 允许民族资本主义经济的发展和富农经济的存在。
% 文化上,废除封建买办文化,发展无产阶级领导的人民大众的反帝反封建的中华民族的新文化,即民族的科学的大众的文化。

% 毛泽东总结了中国共产党成立以来的历史经验,指出统一战线和武装斗争,是战胜敌人的两个基本武器:党的组织,
% 是掌握统一战线和武装斗争这两个武器以实行对敌冲锋陷阵的英勇战士。
% \textbf{统一战线,武装斗争,党的建设,是中国共产党在中国革命中战胜敌人的三个主要的法宝}。

% 以毛泽东为主要代表的中国共产党人创立的新民主主义理论,是马克思主义基本原理同中国具体实际相结合的成果。\textbf{新民主主义理论的系统阐明,标志着毛泽东思想得到多方面展开而达到成熟}。

% \subsection{抗日战争的胜利及其意义}

% \subsubsection{抗日战争的胜利}
% \subsubsection{中国人民抗日战争在世界反法西斯战争中的地位}
% \subsubsection{抗日战争胜利的原因和意义}

% 第一,以爱国主义为核心的民族精神是中国人民抗日战争胜利的决定性因素;
% 第二,中国共产党的中流砥柱作用是中国人民抗日战争胜利的关键;
% 第三,全民族抗战是中国人民抗日战争胜利的重要法宝;
% 第四,中国人民抗日战争的胜利,同世界所有爱好和平和正义的国家和人民、国际组织以及各种反法西斯力量的同情和支持也是分不开的。

% \section{章节自主测试题}

% \section{拓展提高}

% \subsection{战前蒋介石与中共 \quad 日本之间的三角关系}
% \subsection{中华民族的抗日战争}
% \subsection{战时外交:从苦撑待变到大国擘画}
% \subsection{汪精卫政权登场与落幕}

% \chapter{为建立新中国而奋斗}

% 本章的内容是解放战争。在抗日战争胜利后,三种政治力量、三种建国方案、两种命运、两个前途互相较量,
% 最终以国民党政权的垮台和人民民主专政的新中国的建立结束。国民党政权陷入全民的包围中并迅速走向崩溃、
% 第三条道路的幻灭和中国革命胜利的原因、意义和基本经验深刻地指出,\textbf{中华人民共和国的建立、中国共产党
% 的执政地位是历史和人民的选择}。

% \textbf{深刻理解 “没有共产党,就没有新中国”},\textbf{深刻领悟中国革命胜利的原因和基本经验}是本章学习的重点;
% \textbf{对抗日战争后国民党政府陷入全民的包围中并迅速走向崩溃的原因的理解},和\textbf{对抗日战争胜利后中国面临
% 的三条道路和两种命运的认识和客观解读}是本章学习的难点。

% \section{内容精华}

% % 七届二中全会:由农业国转变为工业国、由新民主主义社会转变为社会主义社会

% \subsection{从争取和平民主到击退国民党的军事进攻}

% \subsubsection{中国共产党争取和平民主的斗争}

% 1945 年 8 月,中共中央在对时局的宣言中明确提出的口号是 “\textbf{和平、民主、团结}”。
% 1945 年 8 月 28 日,毛泽东、周恩来、王若飞赴\textbf{重庆}与国民党当局进行谈判。
% 10 月 10 日,双方签署《政府与中共代表会谈纪要》,即 “\textbf{双十协定}”,确认和平建国的基本方针,
% 同意 “\textbf{长期合作,坚决避免内战}”。

% \subsubsection{国民党发动全面内战和解放区军民的坚决反击}

% 人民解放战争分为三个阶段。

% \textbf{第一阶段是战略防御阶段}。1946 年 6 月 26 日,
% \textbf{国民党在美国的大力支持下,进攻我中原解放区},
% 全面内战爆发。国共两党的军队在中原地区爆发了大规模的武装冲突。

% \textbf{第二阶段是战略进攻阶段}。根据中共中央的政策和部署,刘伯承、邓小平率领的晋冀鲁豫野战军主力,
% 于 1947 年 6 月 30 日突破黄河天险,千里跃进大别山;陈毅、粟裕指挥的华东野战军主力为东路,
% 挺进苏鲁豫皖地区;陈赓、谢富治指挥的晋冀鲁豫野战军一部为主力,挺进豫西。
% \textbf{三路大军相互策应、机动歼敌,迫使国民党军处于被动地位。人民解放战争战略进攻的序幕由此揭开}。

% \textbf{第三阶段是战略决战阶段}。1948 年 9 月 16 日,粟裕华东野战军发动济南战役,标志着战略决战开始。
% 在毛泽东和中共中央军委的领导和指挥下,在人民群众的热烈支援下,中国人民解放军先后发动了\textbf{辽沈、淮海、平津三大战役},夺取革命政权。1949 年 4 月,解放军横渡长江,
% 解放南京,基本宣告了国民党统治的覆灭。

% \subsection{全面解放战争的发展和第二条战线的形成}

% \subsubsection{解放战争的胜利发展}

% 中国革命取得建设的三大法宝是\textbf{统一战线、武装斗争和党的建设}。

% \subsubsection{解放区的土地改革运动与农民的广泛发动}

% 在全国内战爆发的前夕,1946 年 5 月 4 日,
% 中共中央发出\textbf{《关于清算、减租及土地问题的指示》}(简称《五四指示》),
% 其基本内容是要坚决地支持和引导广大农民群众,采取各种适当方法,是地主阶级剥削农民而占有的土地
% 转移到农民手中,用一切方法吸收中农参加运动,绝不可侵犯中农土地;一般不变动富农土地,
% 对富农和地主有所区别;不可将农村中反对封建地主阶级的方法,运用于城市中反对工商业资产阶级的斗争。
% 将党在抗日战争时期实行的减租减息政策改变为实行 “\textbf{耕者有其田}” 的政策。

% 在人民解放军转入战略进攻之后,1947 年 7 月至 9 月,
% 中国共产党在河北省平山县召开全国土地会议,制定和通过了\textbf{《中国土地法大纲》}。
% 解放战争时期制定的《中国土地法大纲》规定 “\textbf{废除封建性及半封建性剥削的土地制度,
% 实现耕者有其田的土地制度}”,“乡村中一切地主的土地及公地,由乡村农会接收”,分配给无地或少地的农民。

% 1947 年 10 月 10 日,中国人民解放军总部发表宣言提出的口号是 “\textbf{打倒蒋介石,解放全中国}”。

% 解放战争时期反对内战、独裁的学生运动有 “五二零运动”。“五二零” 运动的口号是 “反饥饿、反内战、反迫害”。

% % \subsubsection{第二条战线的形成和发展}



% \subsection{中国共产党与民主党派的团结合作}

% 1949 年 6 月,毛泽东发表了\textbf{系统论述中国共产党建国主张的著作《论人民民主专政》}。

% \subsubsection{如何理解各民主党派的历史作用?}

% 抗日战争胜利后,民主党派在中国政治舞台上比较活跃。中国各民主党派的政纲不尽相同,
% 但都主张爱国、反对卖国,主张民主、反对独裁。在这些方面,\textbf{各民主党派同中国共产党
% 的新民主主义革命政纲基本上一致}。因此,它们大多从成立时起,就同中国共产党建立了
% 不同程度的合作关系,并在斗争实践中逐步发展了这种关系。

% 在战后进行国共谈判和召开政协会议时,民主党派作为“第三方面”,主要是同共产党一起,
% \textbf{反对国民党的内战、独裁政策,为和平民主而奔走呼号,为政协会议的成功作出了贡献,
% 并为维护政协协议进行过不懈努力}。在国民党当局撕毁政协协议、发动全面内战时,
% 尽管参加民盟的青年党、民主社会党跟随国民党跑了,民盟和其他民主党派的大多数人,
% 在拒绝参加国民党一手包办的伪“国民大会”和虚假的“多党政府”,
% 以及反对国民党炮制的伪“宪法”等一系列重大问题上,是同共产党站在一起的。
% 它们还\textbf{积极参加和支持国民党统治区的爱国民主运动,在第二条战线斗争中尽了一分力量}。

% % \subsubsection{中国共产党与民主党派的合作}
% \subsubsection{中国共产党领导的多党合作与政治协商格局的形成}

% 1949 年 1 月 22 日,李济深、沈钩儒等民主党派的领导人和著名的无党派民主人士 55 人联合发表\textbf{《对时局的意见》},
% 一致认定中国共产党提出的关于召开政治协商会议、成立联合政府的主张“符合于全国人民大众的要求”,
% 恳切表示“愿在中共领导下,献其绵薄,共策进行,以期中国人民民主革命之迅速成功,
% 独立、自由、和平、幸福的新中国之早日实现”。这个政治声明表明,\textbf{中国各民主党派和无党派民主人士自愿接受
% 中国共产党的领导,决心走人民革命的道路,拥护建立人民民主的新中国}。

% 同年春,毛泽东在同有关人士谈话时提出,民主党派应“积极参政,共同建设新中国”。这标志着\textbf{民主党派地位的根本
% 变化,即它们不再是旧中国反动政权下的在野党,而是将在中国共产党领导下,共同担负起管理国家和建设新中国的
% 历史重任。}中国共产党领导的多党合作政治格局,正是在这个基础上形成的。

% \subsection{建立人民民主专政的新中国}

% \subsubsection{南京国民党政权的覆灭}

% 抗日战争胜利后,国民党政府陷入全民的包围中并迅速走向崩溃的原因是多方面的。

% (1) 从根本上说,\textbf{国民党政府实行内战、独裁和卖国的政策,维护大地主大资产阶级的一党专政,
% 妄图把中国拉回到殖民地半殖民地的黑暗中,违背了历史发展规律,违背了全国人民的公意,必然遭到全国人民的反对}。

% (2) 国民党政府的\textbf{官员贪污腐败、大发国难财,严重丧失人心}。抗战后期在大后方已经民怨沸腾。抗战胜利时,
% 国民党政府派出的官员到原沦陷区接收时, 便已把接收变成了“劫收”,对国民党政府抱有很大希望的原沦陷区人民,
% 也很快对其感到极其失望。

% (3) 国民党政府\textbf{违背全国人民迫切要求休养生息、和平建国的意愿,
% 执行反人民的内战政策,公然发动全面内战,造成人心的背离}。

% (4) 国民党反动统治很快\textbf{在军事上、政治上、经济上出现全面危机,这使它迅速走向失败、崩溃}。
% 军事上,到全面内战的第三年,人民解放军发动战略决战,国民党军队主力基本被消灭。
% 政治上,蒋介石召开伪国大,制定伪宪法,玩弄“改组政府”把戏,并残酷镇压人民,制造多起流血惨案,
% 使其政治欺骗迅速破产。经济上,为了筹措内战经费,国民党政府横征暴敛、残酷掠夺,导致恶性通货膨胀,
% 民族工商业倒闭,农村经济破产,人民生活恶化。

% 综上所述,\textbf{国民党政府的倒行逆施迫使全国各阶层人民团结起来,同它做你死我活的斗争}。
% 伴随着国民党军事上、政治上、经济上危机四起,国统区的爱国民主运动迅速高涨,并形成反蒋的第二条战线,
% 有力地配合了人民解放军作战, 使国民党政府陷于全民的包围之中,其统治迅速崩溃。

% \subsubsection{人民政协与《共同纲领》}

% 1949 年 3 月,中共七届二中全会在河北省平山县西柏坡村召开,毛泽东在会上作了报告,
% \textbf{提出党的工作重心必须由乡村转移到城市}。

% 1949 年 9 月 21 日,中国人民政治协商会议第一届全体会议在北平隆重开幕,
% 会议通过了《中国人民政治协商会议共同纲领》。\textbf{《共同纲领》称为中国人民的大宪章,
% 在这一个时期内起着新中国临时宪法的作用}。

% \subsubsection{中国革命胜利的原因}

% 中国新民主主义革命取得胜利的原因是:
% (1) 拥有\textbf{广大人民和各界人士的广泛参加和大力支持};
% (2) 中国工人阶级的先锋队 —— \textbf{中国共产党的领导};
% (3) \textbf{国际无产阶级和人民群众的支持}。

% \subsubsection{中国新民主主义革命取得胜利的基本经验}

% 中国反帝反封建的革命,经历了资产阶级及其政党领导的旧民主主义革命和无产阶级及其政党领导的新民主主义革命
% 两个阶段。近代中国的历史经验表明,\textbf{没有无产阶级及其政党 —— 中国共产党的坚强领导,中国人民革命
% 是不可能胜利的}。

% 中国共产党之所以能够把革命引向胜利,一条重要的经验是:\textbf{必须坚持把马克思列宁主义的基本原理和中
% 国的具体实际结合起来,必须不断推进马克思主义中国化的事业}。正是在中国化的马克思主义理论 ——
% 毛泽东思想的指引下,中国共产党制定了正确的纲领、路线、方针和政策,并找到了适合本国国情的革命道路。

% 中国共产党在领导人民革命的过程中,积累了丰富的经验,锻造出了有效的克敌制胜的武器。毛泽东指出:
% “\textbf{统一战线、武装斗争、党的建设,是中国共产党在中国革命中破胜敌人的三个法宝,三个主要的法宝。}”

% 中国共产党正是遵循毛泽东建党学说,\textbf{在长期的斗争实践中,把自已锻炼成了一个有纪律的、有马克思列宁主义的
% 理论武装的、采取自我批评方法的、联系人民群众的党},成为掌握统一战线和武装斗争这两个武器以实行对敌神锋
% 陷阵的英勇战士,成为全国各族人民拥戴的领导核心。

% 革命的根本问题是国家政权问题。毛泽东在回顾中国共产党走过的历史道路时指出,总结我们的经验,集中到一点,
% 就是\textbf{工人阶级(经过共产党)领导的以工农联盟为基础的人民民主专政}。这个专政必须和国际革命力量
% 团结一致。

% 中国人民革命的胜利和人民民主专政的新中国的创建,彻底改变了近代以后一百多年中国积贫积弱、中国人民受人欺凌的悲惨命运,为实现中华民族伟大复兴创造了根本社会条件。

% \subsubsection{中国革命胜利的意义}

% (1) 中国革命的胜利,\textbf{结束了100多年来中华民族遭受资本-帝国主义侵略和
% 中国各族人民遭受资本-帝国主义同封建统治阶级联合压迫与剥削的历史},结束了国家战乱频仍、四分五裂的局面,
% \textbf{实现了中国人民梦寐以求的民族独立和人民解放}。

% (2) 中国革命的胜利,\textbf{从根本上改变了中国社会的发展方向,为实现由新民主主义到社会主义的转变和
% 建立社会主义制度、进行社会主义现代化建设,扫清了主要障碍},创造了政治前提;
% \textbf{为实现国家富强和人民幸福,实现中华民族伟大复兴,开辟了广阔道路}。

% (3) 中国革命的胜利,\textbf{是继十月社会主义革命和世界反法西斯战争胜利后世界历史中最重大的事件}。
% 它在一个人口占全人类近 1/4 的大国里,冲破帝国主义的东方战线,极大改变了世界的政治格局,
% 壮大了世界和平、民主和社会主义的力量,鼓舞了世界被压迫民族和被压迫人民争取解放的斗争,
% 受到世界人民的欢迎和支持。

% (4) 中国人民革命的胜利,是在马克思列宁主义的指导下取得的。\textbf{中国共产党创造性地运用
% 马克思列宁主义的基本原理,把它同中国革命具体实际结合起来,形成了伟大的毛泽东思想,
% 找到了夺取中国革命胜利的正确道路}。这对于马克思列宁主义的发展是一个重大的贡献。

% % 两个务必:务必使同志们继续地保持谦虚、谨慎、不骄、不躁的作风,务必使同志们继续地保持艰苦奋斗的作风。

% % 毛泽东思想活的灵魂:\textbf{独立自主、实事求是、群众路线}。

% \subsubsection{为什么说 “没有共产党,就没有新中国”?}

% 第一,从历史经验来看,中国共产党成立之前,各种拯救民族危亡的救国方案均以失败告终,\textbf{中国迫切需要
% 新的思想和新的力量组织凝聚革命力量}。只有在中国共产党成立后,中国革命的面貌才焕然一新,给近代饱受战乱,
% 灾难深重的中国人民送来了光明的希望,最终实现了民族独立与人民解放。

% 第二,\textbf{中国共产党之所以能够把革命引向胜利,一条根本性的经验就是把马克思列宁主义的基本原理和中
% 国的具体实际结合起来,不断推进马克思主义中国化}。正是在中国化的马克思主义理论 —— 毛泽东思想指引下,
% 中国共产党制定了正确纲领、路线、方针和政策,并找到了适合本国国情的革命道路。

% 第三,中国共产党在领导人民革命的过程中,积累了丰富的经验,锻造出了有效的克敌制胜的武器。毛泽东指出:
% “\textbf{统一战线、武装斗争、党的建设,是中国共产党在中国革命中战胜敌人的三个法宝,三个主要的法宝}。”

% \section{章节自主测试题}

% \section{拓展提高}

% \subsection{国名党统治的衰颓}
% \subsection{国民党大陆统治的瓦解及其退台}

% \chapter{中华人民共和国的成立与中国社会主义建设道路的探索}

% \section{内容精华}

% \subsection{中华人民共和国的成立与新生人民政权的巩固}

% 中国新民主主义革命基本胜利的标志是\textbf{中华人民共和国的成立}。


% 中华人民共和国的成立,\textbf{彻底结束了旧中国半殖民地半封建社会的历史},\textbf{彻底结束了旧
% 中国一盘散沙的局面},\textbf{彻底废除了列强强加给中国的不平等条约和帝国主义在中国的一
% 切特权},\textbf{实现了中国从几千年封建专制政治向人民民主的伟大飞跃},\textbf{实现了中国高度统
% 一和各民族空前团结}。

% 中国人从此站立起来了!中国人民从此把金运牢牢堂握在自己手中,成为国家、
% 社会和自己命运的主人!中华民族发展进步从此开启了新纪元!

% \subsubsection{我国新民主主义社会的性质和基本特征}

% \textbf{新民主主义社会不是独立的社会形态,而是属于社会主义体系井要逐步过渡到
% 社会主义社会的过渡性质的社会}。在新民主主义社会既有资本主义因素,又有
% 社会主义因素,而后者占主导地位。新民主主义社会是连接民主革命和社会主
% 义革命的纽带和桥梁,是中国社会发展不可逾越的一个阶段。

% 我国新民主主义社会的基本特征是:(1) 政治上:\textbf{实行工人阶级领导的,以工农联盟为基础的,
% 各革命阶级联合专政的政治制度}。(2) 经济上:实行\textbf{国营经济领导下的合作社经济、个体经济
% 、私人资本主义经济和国家资本主义经济五种经济成分井在的经济制度}。
% 在新民主主义向社会主义过渡时期,经济上处于领导地位的是\textbf{社会主义性质的国营经济}.
% (3) 文化上:发展以马克思主义指导下的新民主主义的文化。

% \subsubsection{新中国成立初期面临的考验}
% \subsubsection{巩固新政权的伟大斗争}

% 新中国的基本外交方针:\textbf{另起炉灶}、\textbf{一边倒}、\textbf{打扫干净屋子再请客}。

% “三反” 运动的基本内容是:反贪污、反浪费、反官僚主义,
% “五反” 运动的基本内容是:反行贿、反偷税漏税、反盗窃国家资财、反偷工减料、反盗窃国家经济情报。
% \textbf{延安整风运动}的基本内容是:\textbf{反主观主义}、\textbf{反宗派主义}、\textbf{反党八股}。

% 1950年至1953年的抗美援朝战争,战争的胜利打破了美国不可战胜的神话。
% 抗美援朝精神的内涵是:
% (1) \textbf{祖国和人民利益高于一切、为了祖国和民族的尊严而奋不顾身}的爱国主义精神;
% (2) \textbf{英勇顽强、舍生忘死}的革命英雄主义精神;
% (3) \textbf{不畏艰难困苦,始终保持高昂士气}的革命乐观主义精神;
% (4) \textbf{为完成祖国和人民赋予的使命、慷慨奉献自己一切}的革命忠诚精神;
% (5) \textbf{为了人类和平与正义事业而奋斗}的国际主义精神。

% \subsubsection{为社会主义改造创造条件}

% \subsection{党在过渡时期的总路线及其实施}

% \subsubsection{过渡时期总路线的提出}

% 1953年提出党在过渡时期的总路线:“党在这个过渡时期的总路线和总任务是\textbf{要在一个
% 相当长的时期内,逐步实现国家的社会主义工业化,井逐步实现国家对农业、对手工业
% 和对资本主义工商业的社会主义改造}。”(简称“一化三改”)

% 社会主义工业化与社会主义改造两者的关系:工业化是改造的基础和目的,改造是工业化的条件和手段。两者相互联系,相互促进,相互制约,体现了发展生产力和变革生产关系的有机统一。

% \subsubsection{过渡时期的总路线反映了历史的必然性。}

% (1) 社会主义工业化是国家独立富强的首要条件;
% (2) 资本主义经济力量弱小、发展困难,不可能成为中国工业起飞的基础;
% (3) 对个体农业进行社会主义改造,是保证工业发展、实现国家工业化的一个必要条件;
% (4) 当时的国际环境也促使中国选择社会主义。

% \subsubsection{社会主义工业化的起步}
% \subsubsection{个体农业和手工业的改造}
% \subsubsection{资本主义工商业的改造}

% 对民族资产阶级的改造采取和平赎买政策。

% \subsection{社会主义基本制度的确立}

% 1956 年,\textbf{社会主义改造的基本完成标志着我国社会主义基本制度的确立},
% 我国开始进入全面建设社会主义的历史阶段。

% \subsubsection{社会主义经济制度的确立}
% \subsubsection{社会主义政治制度的确立}
% \subsubsection{社会主义基本制度确立的伟大意义}

% 经济上,社会主义基本制度的确立\textbf{极大地提高了工人阶级和广大劳动人民的积极性、
% 创造性,为社会生产力的大发展开辟了广阔道路};
% 政治上,社会主义基本制度的确立\textbf{为当代中国的一切发展进步提供了基本的政治保障};
% 文化上,社会主义基本制度的确立\textbf{为社会主义先进文化的发展指明了前进方向}。

% \subsection{社会主义建设的良好开端}

% \subsubsection{全面建设社会主义的开始}
% \subsubsection{早期探索的积极进展}

% 《论十大关系》的历史意义:是以毛泽东为主要代表的中国共产党人开始探索中国自己的社会主义建设道路的标志;
% 在新的历史条件下从经济方面和政治方面提出了新的指导方针,为中共八大的召开作了理论准备。

% \subsection{社会主义建设道路的艰辛探索和曲折发展}

% \subsubsection{“大跃进” 及初步纠 “左” 的努力}
% \subsubsection{国民经济调整和 “四个现代化” 战略目标的提出}
% \subsubsection{“文化大革命” 内乱及其历史教训}
% \subsubsection{全面建设社会主义的成就}


% \section{章节自主测试题}

% \section{拓展提高}

\appendix
% 设置 chapter 标题样式
\titleformat{\chapter}[hang]{\centering\heiti\Large\bfseries}{附录\,\thechapter}{1em}{}
\renewcommand{\chaptermark}[1]{\markboth{附录 \thechapter\, #1}{}}

\chapter{《中国近现代史纲要》冲刺速成资料}

\chapter{零碎的考点}

\section{土地问题}

\begin{table}[H]
    \centering
    \caption{土地问题总结}
    \begin{tabular}{p{0.3\textwidth} c c}
        \toprule
        & 耕者有其田 & 否定封建土地剥削 \\
        \midrule 
        \hline
        太平天国运动 & T & T \\
        辛亥革命(旧三民主义)的资产阶级共和国方案 & F & F \\
        \bottomrule
    \end{tabular}
\end{table}

\section{中国共产党的会议}

\section{毛泽东的文章}

\subsection{大革命时期}

\begin{table}[H]
    \centering
    \caption{毛泽东在大革命时期的文章}
    \begin{tabular}{p{0.3\textwidth}|p{0.4\textwidth}p{0.2\textwidth}}
        \hline 
        \textbf{文章} & \textbf{主要内容} & \textbf{关键词} \\
        \hline
        《中国社会各阶级的分析》(1925)&
        \textbf{中国革命的首要问题是分清敌友}。以往的斗争之所以成效甚少,
        一个重要的原因,就在于不能团结真正的朋友,以攻击真正的敌人。
        & 中国革命的首要问题 \\ 
        \hline
    \end{tabular}
\end{table}

\subsection{土地革命战争后期至抗日战争前期}

\begin{table}[H]
    \centering
    \caption{毛泽东在大革命时期的文章}
    \begin{tabular}{p{0.3\textwidth}|p{0.4\textwidth}p{0.2\textwidth}}
        \hline 
        \textbf{文章} & \textbf{主要内容} & \textbf{关键词} \\
        \hline
        《论反对日本帝国主义的策略》(1935.12)&
        \textbf{阐明党的抗日民族统一战线的新政策},批判党内的关门主义和对于革命的急性病,
        系统地解决了党的\textbf{政治策略}上的诸问题。
        & 政治问题;统一战线 \\ 
        \hline
        《中国革命战争的战略问题》(1936.12) &
        总结土地革命战争中党在军事问题上的争论,系统地说明了有关中国革命战争
        战略方面的诸问题。
        & 军事问题 \\ 
        \hline 
        《实践论》、《矛盾论》(1937 年夏)
        & 从马克思主义认识论的高度,总结中国共产党的历史经验,揭露和批判党内的主观主义
        尤其是教条主义错误,深入论证马克思列宁主义基本原理同中国具体实际相结合的原则,
        科学地阐明了党的马克思主义的思想路线。
        & 主观主义;教条主义;马克思列宁主义基本原理同中国具体实际相结合;思想路线 \\
        \hline 
    \end{tabular}
\end{table}

\begin{example}
    中国革命的首要问题是 \textbf{分清敌友}。
\end{example}

\section{罢工问题}

为了支援上海人民五卅反帝爱国运动,在全国人民和广州国民政府的大力支援下,从 1925 年 6 月
到 1926 年 10 月,省港大罢工持续 16 个月之久,这在中国工人运动史上是空前的,
在世界工人运动史上也属罕见。

在中国共产党的领导下,从 1922 年 1 月的香港海员大罢工到 1922 年 9 月的安源路矿工人
大罢工,再到 1932 年 2 月的京汉铁路工人罢工,中国工人运动出现了第一次高潮。



% \onecolumn

\newpage
\thispagestyle{empty}
\begin{thebibliography}{1}
    \addcontentsline{toc}{chapter}{参考文献}
    \markboth{中国近现代史纲要}{参考文献}
    \bibitem{2023版}
    《中国近现代史纲要 (2023 版)》编写组编, 中国近现代史纲要 (2023 版)[M], 第 9 版, 
    北京:高等教育出版社, 2023.2
    \bibitem{教学辅导}
    赵海霞主编, 西北工业大学马克思主义学院组织编写, 中国近现代史纲要教学辅导[M], 第 3 版, 
    西安:西北工业大学出版社, 2019.8
    \bibitem{两岸晚清}
    王建朗, 黄克武主编, 两岸新编中国近代史·晚清卷[M], 2016.9
    \bibitem{两岸民国}
    王建朗, 黄克武主编, 两岸新编中国近代史·民国卷[M], 2016.6
    \bibitem{人类简史}
    (以色列) 尤瓦尔·赫拉利 (Yuval Noah Harari) 著. 人类简史: 从动物到上帝 (Sapiens: A Brief History
    of Humankind)[M]. 林俊宏译. 第 2 版.
    北京: 中信出版社.
    % \bibitem{口述史}
    % 钱锋, 改革开放口述史:改革开放与党的反腐败工作——
    % 改革开放以来党的有关反腐败的具体司法实践工作的变化, 2022.11
    \bibitem{核心考案}
    徐涛主编, 考研政治核心考案[M], 北京:中国政法大学出版社, 2023.1
    \bibitem{优题库}
    徐涛主编, 考研政治通关优题库[M], 北京:中国政法大学出版社, 2023.2
    \bibitem{真题库}
    徐涛主编, 考研政治必刷真题库[M], 北京:中国政法大学出版社, 2023.2
    \bibitem{冲刺背诵笔记}
    徐涛主编, 考研政治冲刺背诵笔记[M], 北京:中国政法大学出版社, 2023.9
    \bibitem{预测}
    徐涛主编, 考研政治预测 6 套卷[M], 北京:中国政法大学出版社, 2023.10
    \bibitem{预测}
    徐涛, 曲艺主编, 考研政治考前预测必备 20 题[M], 北京:中国政法大学出版社, 2023.11
\end{thebibliography}



% [序号]主要责任者.电子文献题名.电子文献出处[电子文献及载体类型标识].或可获得地址,发表或更新日期/引用日期。

\newpage
\thispagestyle{empty}

\newpage
\thispagestyle{empty}
\vspace*{5cm}
\begin{center}
    \includegraphics*[width=\textwidth]{../pic/i_love_npu.jpeg}
    \large
    公诚勇毅 \quad 永矢毋忘

    中华灿烂 \quad 工大无疆
\end{center}
\vspace*{13em}
\begin{center}
    \small
    本文档由\textbf{钱锋}编写, 钱锋保留一切权利.

    文档中出现的部分素材来源于网络, 笔者承诺这些素材仅供学习交流之用, 
    它们的原作者保留一切权利.

    2023 年 \quad 西北工业大学 \quad 中国西安 
\end{center}

\end{document}